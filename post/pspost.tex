
% $Id: pspost.tex,v 1.7 2003/08/19 09:47:49 georg Exp $

\documentclass{article}

\usepackage{shortvrb}

\newcommand{\psp}{{\tt pspost}}
\newcommand{\VERSION}{1.5}

\MakeShortVerb{\|}

%%%%%%%%%%%%%%%%%%%%%%%%%%%%%%%%%%%%%%%%%%%%%%%%%%%%%%%%% front matter

\title{%
        \psp
        \\Doing Graphics in PostScript
        \\with Fortran and C
        }

\author{%
        Georg Umgiesser
        \\ISMAR-CNR, S. Polo 1364
        \\30125 Venezia, Italy
        \vspace{1cm}
        \\Version \VERSION
        }

%\address{ISDGM-CNR}

%%%%%%%%%%%%%%%%%%%%%%%%%%%%%%%%%%%%%%%%%%%%%%%%%%%%%%%%% document

\begin{document}

\pagenumbering{roman}
\pagestyle{plain}

\maketitle

\begin{abstract}
A simple graphic plotting package is presented that can be used for
the creation of PostScript graphics. The routines are callable from
Fortran and C.

Only the basic plotting commands have been implemented. The library
allows you to plot lines and points, fill arbitrary shapes
with arbitrary color and write text with an arbitrary point size.
It also allows for producing more pages in one plot. Color can be
used as gray scale or through the RGB and HSB color spaces.
The coordinate system may be set to best adjust to the drawing.
Clipping graphics in a given rectangle is implemented.
\end{abstract}

\thispagestyle{empty}

\newpage

\tableofcontents

\newpage

\section*{Disclaimer}
\addcontentsline{toc}{section}{Disclaimer}


\begin{quotation}
 
   Copyright (c) 1992-2003 by Georg Umgiesser 
  
   Permission to use, copy, modify, and distribute this software 
   and its documentation for any purpose and without fee is hereby 
   granted, provided that the above copyright notice appear in all 
   copies and that both that copyright notice and this permission 
   notice appear in supporting documentation. 
  
   This file is provided AS IS with no warranties of any kind.
   The author shall have no liability with respect to the
   infringement of copyrights, trade secrets or any patents by
   this file or any part thereof.  In no event will the author
   be liable for any lost revenue or profits or other special,
   indirect and consequential damages.
  
   Comments and additions should be sent to the author:

        \begin{verbatim}
              Georg Umgiesser
              ISMAR-CNR
              S. Polo 1364
              30125 Venezia
              Italy
 
              Tel.   : ++39-041-5216875
              Fax    : ++39-041-2602340
              E-Mail : georg.umgiesser@ismar.cnr.it
        \end{verbatim}
\end{quotation}

\section*{Availability}
\addcontentsline{toc}{section}{Availability}


The library \psp{} is available free of charge bye anonymous ftp.
Connect to
|ftp.ve.ismar.cnr.it| and look in the directory |/pub/|\psp{}.
Please read the README file of the distribution. Only source code
is available.

The library should compile out of the box for nearly all Unix-like
systems. Please follow the instructions in the README file.

Please send bug reports to the author (|georg.umgiesser@ismar.cnr.it|).


\newpage

\pagenumbering{arabic}



\section{Introduction}


The plotting package \psp{} has been designed for Fortran and C
programmers to take advantage of the powerful PostScript language. With
\psp{} graphics can be easily generated and printed on any PostScript
printer and, with the aid of the program |ghostscript|, also on many
non PostScript printers. Moreover, with the programs |ghostview|
and |ghostscript|, the plot files may be viewed on the screen of your 
computer.

The \psp{} package has been kept voluntarily simple. There are no high
level plotting commands. Only the basic plotting commands are
implemented.  This was done in order to keep the code simple and
understandable.  However, it should be no problem to build higher level
graphics around the basic commands found in the \psp{} package.

The routines in the \psp{} package handle all the low level graphics,
such as switching to an other gray tone or color, plotting a line,
filling a rectangle with a color, setting the line width and writing
text in an arbitrary point size and font to the output device, mostly a
file. The routines are callable in Fortran and C.


\section{A Fortran Example}

\label{Example}

The following code shows a complete Fortran program
with the creation of two pages of output. The first
page contains a line, that runs across the page. The second line
contains the sentence ``Hello World!'', written in the current points
size and font.

\begin{verbatim}
        call qopen
        call qstart
        call qline(1.,1.,10.,10.)
        call qend
        call qstart
        call qtext(2.,2.,'Hello World!')
        call qend
        call qclose
        end
\end{verbatim}

Some comments can be made on this small program:
\begin{itemize}

\item   The whole code containing the graphics is bracketed by the
        sequence |call qopen| $\ldots$ |call qclose|. The call to
        |qopen| initializes the graphics library, |qclose| closes the
        file and does some cleaning up.

\item   Every page is started by a call to |qstart| and ended
        by |qend|.

\item   The call to |qline| plots a line from the point
        (1,1) to (10,10) using the actual color (black).  The
        coordinate system used is initially set to encompass the whole
        page (A4). The user coordinates are in centimeters. However,
        these values can be changed easily.

\item   The call to |qtext| writes the phrase
        ``Hello World!'' to the second page starting at the point
        (2,2). The default font used is ``Courier'' and the default
        point size is 12 pt.

\end{itemize}


\section{Description of the Plotting Package}

\label{Descrip}

In the following a description of the plotting package is given.  This
should put you in a position to use the package \psp{} with your own
routines.

Only the code for Fortran is shown. Here the typical usage of the
routines are shown. For the exact calling sequence please consult
section \ref{ForSeq}.

It is trivial to substitute the various Fortran routines with their
equivalent in C.  How this is done is shown in section \ref{CSeq}.

\subsection{Initialization}


All plotting code must be bracketed by a call to |qopen| and |qclose|.
The routine |qopen| initializes the plotting routines and opens the
file |plot.ps|. All output will then go to this file. The routine
|qclose| closes the graphic routines, does some cleaning up and closes
the file |plot.ps|.

% ?? how to change the plot file name

\subsection{Plotting Pages}


During one plotting session, more pages of output can be produced.
Every page must be opened, however, explicitly. You do this with a call
to |qstart|. After that call, all output goes to the actual page.
Every page has also to be closed explicitly with a call to |qend|.
This flushes the buffer, writes all remaining output to the file and
resets some parameters.

The structure of a plot is therefore the following:

\begin{verbatim}
        call qopen
        call qstart
         ...plotting commands for first page
        call qend
        call qstart
         ...plotting commands for second page
        call qend
         ...more pages
        call qclose
\end{verbatim}


\subsection{Coordinate System}


Two different coordinate systems are used for this graphic package.
One describes the physical position of the plot on the page (viewport
coordinates). The other coordinate system is used for the plotting
commands to actually plot the graphics (world coordinates).


\subsubsection{Viewport}

A viewport is a part of the actual page where the graphic routines may
place graphical marks, i.e., may plot. Outside the viewport,
nothing is plotted. If necessary, the lines and text will be clipped at
the boundary of the viewport.

At the beginning of each page, the viewport is described by a Cartesian
coordinate system with the origin in the lower left corner. The unit of
the coordinate system is in centimeters. For the paper size A4, the
usable portion of the page extends from 0 to 19 cm in $x$ direction and
from 0 to 28 cm in $y$ direction.

Three calls directly manipulate viewports. The first one is |qgetvp|,
which returns the dimension of the actual viewport in use.  On the
other hand, |qsetvp| sets the viewport to a new window on the current
page. Finally, |qclrvp| clears the viewport. This need be only called,
if the actual viewport overlapps with an old viewport that already has
put some graphics on the page, and if these graphical elements must be
deleted in the new viewport. In such a case, if you do not call
|qclrvp| the old graphics will remain on the page.

Viewport coordinates are used to place the plotting window on the
page.  However, they are not used for the plotting commands.  Note that
viewport coordinates are always in centimeters.


\subsubsection{World Coordinates}

To do the real plotting, the so called world coordinates are used.
These world coordinates can be adjusted so that they will fit to your
particular problem. World coordinates are established with a call to
|qworld|. The call takes 4 arguments, which are the coordinates of the
lower left and the upper right corner. These two points are made
coincident with the two corner points of the actual viewport.

An example will suffice:

\begin{verbatim}
        call qopen
        call qstart
        call qsetvp(4.,4.,10.,10.)
        call qworld(0.,0.,1000.,1000.)
          ...plotting commands
        call qend
        call qclose
\end{verbatim}

After initializing the graphics with |qopen| and opening the page with
|qstart|, the viewport window is set to a window of size $6\times6$ cm.
The lower left point of the viewport is at (4,4) cm and the upper right
point at (10,10) cm.  All plotting commands that place drawings outside
this window will be clipped.

The next call to |qworld| sets the world coordinates of the plotting
window.  After that call moving to point (0,0) will move the actual
position of the plotting pen to the coordinate (4,4) cm on the page,
and moving to (1000,1000) will move to (10,10) cm on the page.

More then one call can be made to |qworld| without changing the
viewport.  This corresponds to different plotting coordinate systems,
that all place their plotting marks into the window given by the
viewport.  Note that, if you never call the routine |qworld|, your
world coordinates are identical to the viewport coordinates. This is,
you can give your plotting commands in centimeters. In the above
example, without the call to |qworld|, you could plot in a window with
world coordinates between (4,4) and (10,10).


\subsubsection{Distortion}

If your scale factor, i.e., the ratio of the window size in $x$
and in $y$ direction, of the viewport window and the world window
differ, there will be a distortion of the drawing. Sometimes this is
just what is wanted. For example if a time series is to be plotted with
time on the $x$ axis and some arbitrary value on the $y$ axis the units
in $x$ and $y$ direction are normally different. However, if you are
plotting a horizontal map, you probably want the unit in $x$ and $y$
direction to be the same, e.g., kilometers. Now, if your scale
factor between the viewport and the world window is different, you will
have distortion.

This distortion can be avoided with a call to |qrcfy|, after you have
called |qworld|. This call adjusts the scaling factors in order for
your units to be the same in both direction. This frees you from
computing in your programs the scaling factor and adjust the world
coordinates for distortionless plotting.


\subsection{Plotting Commands}


In this part all plotting commands that actually plot something on the
page will be discussed. These are the line plotting commands, the
filling routines and the text routines.


\subsubsection{Line Plotting}

A line can be plotted with the routine |qline|. This routine takes 4
arguments, the $(x,y)$ coordinates of the starting point and the end
point. The line is plotted with the color and width that is actually in
use.

To plot a line you can also use the commands |qmove| and |qplot|.
|qmove| moves to a given point, called the actual position. |qplot|
plots a line from the actual position to the point given in the
arguments.

Therefore, the following statements are equivalent:

\begin{verbatim}
        call qline(x1,y1,x2,y2)
          ... is the same as
        call qmove(x1,y1)
        call qplot(x2,y2)
\end{verbatim}

The last routine is |qpoint| that plots a point at the specified
position. The size of the point is according to the actual line width.

Please note that these 4 routines are the only routines that actually
set the actual position. It is set to the endpoint of the line
plotted, or, in case of |qpoint|, to the coordinate of the point.
In order to use the call to |qplot|, one of
these routines must have been called before. Specifically, the filling
routines and the text routines do not reset the actual position.
Calling |qplot| after one of these routines will give unpredictable
results.

\subsubsection{Filling}

Two routines can be used for filling a shape. With |qafill| you can
fill an arbitrary shape with the actual color. You have to pass a
vector of $x$ and $y$ coordinates and the length of the vector.

If you have to fill a rectangle, you can directly use |qrfill|. Two
points, the lower left and the upper right point of the rectangle must
be specified. For a rectangle this call is easier then setting up a
vector of size 4 and passing it to |qafill|. Moreover, the call to
|qrfill| is optimized for speed and size of the plot file. 
If possible, you should use this routine for plotting rectangles.

\subsubsection{Text}

Text can be written to the page with a call to |qtext|. This routine
takes 3 arguments, an $(x,y)$ coordinate of the position where the text
should start, and a string that has to be written.

The text is written in the actual text size (default 12 points) and
font (Courier). However, you can change the default. A call to |qtxts|
changes the actual text size, whereas a call to |qfont| changes the
actual font.

\begin{verbatim}
        call qtext(3.,3.,'This is a test in Courier, 12 pt')
        call qtxts(20)
        call qfont('Times-Roman')
        call qtext(3.,5.,'This is Times-Roman, 20 pt')
\end{verbatim}

The call to |qtext| also takes two special arguments for the $(x,y)$
position. If one of the coordinates is exactly 999.0, then this
coordinate is not changed and the new text is appended to the last
coordinate. You can use this effect, if you want to continue a string
at the end of the old string. Here is an example:

\begin{verbatim}
        call qtxts(10)
        call qfont('Times-Roman')
        call qtext(1.,1.,'Just a test')
        call qtext(999.,999.,' continue after the old string')
        call qtext(999.,2.,' continue in x, but one cm higher')
        call qtext(15.,999.,' same y, but at x=15')
\end{verbatim}

Hopefully you will never need to write text exactly to the coordinate
999.0. If you have to, you are in a bad shape

The availability of certain fonts depends on the type of
your PostScript printer.
However, there are 13 fonts that are guaranteed to be available on
every PostScript interpreter. These fonts are

\begin{itemize}
  \item Times Family
  \begin{itemize}
        \item Times-Roman
        \item Times-Italic
        \item Times-Bold
        \item Times-BoldItalic
  \end{itemize}
  \item Helvetica Family
  \begin{itemize}
        \item Helvetica
        \item Helvetica-Oblique
        \item Helvetica-Bold
        \item Helvetica-BoldOblique
  \end{itemize}
  \item Courier Family
  \begin{itemize}
        \item Courier
        \item Courier-Oblique
        \item Courier-Bold
        \item Courier-BoldOblique
  \end{itemize}
  \item Symbol
\end{itemize}



\subsection{Attributes}


In this part the attributes of the plotting commands are discussed.
These are the color of the plotted lines, text and filled shapes, and
the line width of the lines.

\subsubsection{Color}


At the start of a page, the actual color is always set to black.  This
means, that when drawing a line, point, writing some text or filling
some shape, black is used as the drawing color.

There are more color spaces available for manipulating the color of a
plot. These color spaces are discussed in turn.  Please note that, if
you do not have a color printer, the color shows up in your block as
some sort of gray.

\paragraph{Gray Color Space}

The routine |qgray(gray)| sets the gray tone for the subsequent plots.
The call takes one argument |gray|, which is the gray tone to be used
for the actual color. The value must be between 0 and 1, where 0 means
black, and 1 means white. Intermediate values result in various shades
of gray.

\paragraph{RGB Color Space}

Color in the RGB (Red, Green, Blue) color space may be set with a call
to |qrgb(red,green,blue)|, where |red,green,blue| are the fraction of red,
green and blue color that are used to create the actual color.  Values
of these arguments are between 0 and 1, where 0 means no such color,
and 1 means full color. So a call |qrgb(1.,0.,0.)| will give you a
bright red, |qrgb(0.,1.,1.)| will give you bright cyan and
|qrgb(0.5,0.5,0.5)| will give you an intermediate gray tone.

\paragraph{HSB Color Space}

Color in the HSB (Hue, Saturation, Brightness) color space may be set
with a call to |qhsb(hue,sat,bri)|. Values for the arguments are again
between 0 and 1. If you just need to build a simple color scale, then
you can also use a call to |qhue(hue)|. This adjusts only the value of
the hue (type of color), setting implicitly the values for saturation
and brightness to 1. You can then use this color scale as you would use
a simple gray scale with calls to |qgray(gray)|. But, instead of gray,
you will get color.

\paragraph{Plotting with White}

If you use the gray color space, plotting with the white color 
(|call qgray(1.)|) is inhibited. This means that all plotting commands that
use white as the actual color are not actually executed. This done in
order to avoid unnecessary big plot files.

If you need to plot with white, because of some special effects or some
over-plotting, you must call routine |qwhite| with the argument set to
1, so: |call qwhite(1)|. From then on, plotting with white is enabled.
If you want to disable white plotting, you must |call qwhite(0)|
again.


\subsubsection{Line Width}

You can change the line width with a call to |qlwidth(width)|.  Here
|width| is the width in centimeters. It defaults to 0.01 cm.


\subsection{Various}

For debugging reasons, you can include in your plot file some comments
that explain what you are doing. These comments show up in the plot
file, but not when you print the graphics.

You place your comments with |call qcomm('This is a comment')|. Note
that comments are completely optional for the graphics, but may be
useful for you.


\section{Fortran Subroutines}

\label{ForSeq}


In this section all available Fortran routines are listed, with their 
arguments. For their usage please refer to section \ref{Descrip}.


\subsection{Initialization and Page Commands}


\begin{verbatim}
        subroutine qopen
c       opens output file and initializes graphics
        end
\end{verbatim}

\begin{verbatim}
        subroutine qclose
c       closes output file and graphics routines
        end
\end{verbatim}

\begin{verbatim}
        subroutine qstart
c       opens new page
        end
\end{verbatim}

\begin{verbatim}
        subroutine qend
c       closes actual page and flushes buffer to file
        end
\end{verbatim}


\subsection{Coordinate System}


\begin{verbatim}
        subroutine qgetvp(xmin,ymin,xmax,ymax)
c       gets viewport dimensions (in cm)
c       ... note that xmin,ymin,xmax,ymax are returned values
        real xmin,ymin   !lower left coordinates of viewport
        real xmax,ymax   !upper right coordinates of viewport
        end
\end{verbatim}

\begin{verbatim}
        subroutine qsetvp(xmin,ymin,xmax,ymax)
c       sets viewport dimensions (in cm)
        real xmin,ymin   !lower left coordinates of viewport
        real xmax,ymax   !upper right coordinates of viewport
        end
\end{verbatim}

\begin{verbatim}
        subroutine qclrvp
c       clears actual viewport
        end
\end{verbatim}

\begin{verbatim}
        subroutine qworld(xmin,ymin,xmax,ymax)
c       sets world coordinates
        real xmin,ymin   !lower left values of world coordinates
        real xmax,ymax   !upper right values of world coordinates
        end
\end{verbatim}

\begin{verbatim}
        subroutine qrcfy
c       adjusts scale factor
        end
\end{verbatim}


\subsection{Plotting Commands}


\begin{verbatim}
        subroutine qline(x1,y1,x2,y2)
c       plots line between two points
        real x1,y1     !starting point of line
        real x2,y2     !ending point of line
        end
\end{verbatim}

\begin{verbatim}
        subroutine qmove(x,y)
c       moves to point
        real x,y       !coordinates of point to move to
        end
\end{verbatim}

\begin{verbatim}
        subroutine qplot(x,y)
c       plots to point
        real x,y       !coordinates of point to draw line to
        end
\end{verbatim}

\begin{verbatim}
        subroutine qpoint(x,y)
c       plots point
        real x,y       !coordinates of point to plot
        end
\end{verbatim}

\begin{verbatim}
        subroutine qafill(n,x,y)
c       fills arbitrary shape
        integer n      !total number of points given
        real x,y       !vector of points defining shape
        end
\end{verbatim}

\begin{verbatim}
        subroutine qrfill(x1,y1,x2,y2)
c       fills rectangle
        real xmin,ymin   !lower left values of rectangle
        real xmax,ymax   !upper right values of rectangle
        end
\end{verbatim}


\subsection{Text Commands}


\begin{verbatim}
        subroutine qtxts(ip)
c       sets text size in points
        integer ip      !text size in points
        end
\end{verbatim}

\begin{verbatim}
        subroutine qfont(font)
c       sets font for text
        character*(*) font      !font name
        end
\end{verbatim}

\begin{verbatim}
        subroutine qtext(x,y,s)
c       writes text s at (x,y)
        real x,y          !coordinates to start writing text
        character*(*) s   !string of text to write
        end
\end{verbatim}

\begin{verbatim}
        subroutine qtsize(s,w,h)
c       computes width and height of string
        character*(*) s   !string of text
        real w,h          !computed width and height of string s
        end
\end{verbatim}


\subsection{Color Management}


\begin{verbatim}
        subroutine qgray(gray)
c       sets actual color to gray tone
        real gray         !shade of gray [0,1] (0=black, 1=white)
        end
\end{verbatim}

\begin{verbatim}
        subroutine qrgb(red,green,blue)
c       sets actual color to rgb value
        real red,green,blue    !value for colors rgb [0,1]
        end
\end{verbatim}

\begin{verbatim}
        subroutine qhsb(hue,sat,bri)
c       sets actual color to hsb value
        real hue,sat,bri       !value for colors hsb [0,1]
        end
\end{verbatim}

\begin{verbatim}
        subroutine qhue(hue)
c       sets hue of actual color (bri=sat=1)
        real hue               !value for hue [0,1]
        end
\end{verbatim}

\begin{verbatim}
        subroutine qwhite(ipaint)
c       en/dis-ables plotting with white
        integer ipaint         !0=disable/1=enable white plotting
        end
\end{verbatim}


\subsection{Color Table}


\begin{verbatim}
        subroutine qinitct(n)
c       initializes color table
        integer n               !size of color table
        end
\end{verbatim}

\begin{verbatim}
        subroutine qsetct(i,itype,c1,c2,c3)
c       sets color i in color table
        integer i               !color to set [1-n]
        integer itype           !type of color: 0=HSB  1=RGB
        real c1,c2,c3           !color value (either hsb or rgb) [0-1]
        end
\end{verbatim}

\begin{verbatim}
        subroutine qsetcpen(i)
c       sets actual color from color table
        integer i               !color to use [1-n]
        end
\end{verbatim}

\begin{verbatim}
        subroutine qsetcrange(c)
c       sets actual color from color table
        real c                  !color to use [0-1]
        end
\end{verbatim}


\subsection{Attributes}


\begin{verbatim}
        subroutine qlwidth(width)
c       sets line width
        real width             !line width
        end
\end{verbatim}

\begin{verbatim}
        subroutine qpsize(size)
c       sets point size
        real size              !point size
        end
\end{verbatim}


\subsection{Various}


\begin{verbatim}
        subroutine qcomm(s)
c       writes comment to output file
        character*(*) s   !string of comment
        end
\end{verbatim}


\section{C Routines}

\label{CSeq}

In this section the prototypes of the C routines are given.
The C routines are completely equivalent to their Fortran counterpart.
For the meaning of the C routines please refer to section \ref{Descrip}
and \ref{ForSeq}. We also include a short table that shows the 
correspondence between the C and Fortran routines.

\DeleteShortVerb{\|}

\begin{table}
\begin{center}
\begin{tabular}{|l|l|} \hline

Fortran routine & C routine \\ \hline \hline

qopen   & PsGraphOpen           \\ \hline
qclose  & PsGraphClose          \\ \hline
qstart  & PsStartPage           \\ \hline
qend    & PsEndPage             \\ \hline
                                   \hline
qgetvp  & PsGetViewport         \\ \hline
qsetvp  & PsSetViewport         \\ \hline
qclrvp  & PsClearViewport       \\ \hline
qworld  & PsSetWorld            \\ \hline
qrcfy   & PsRectifyScale        \\ \hline
                                   \hline
qline   & PsLine                \\ \hline
qmove   & PsMove                \\ \hline
qplot   & PsPlot                \\ \hline
qpoint  & PsPoint               \\ \hline
                                   \hline
qafill  & PsAreaFill            \\ \hline
qrfill  & PsRectFill            \\ \hline
                                   \hline
qtxts   & PsTextPointSize       \\ \hline
qfont   & PsTextFont            \\ \hline
qtext   & PsText                \\ \hline
qtsize  & PsTextDimensions      \\ \hline
                                   \hline
qgray   & PsSetGray             \\ \hline
qrgb    & PsSetRGB              \\ \hline
qhsb    & PsSetHSB              \\ \hline
qhue    & PsSetHue              \\ \hline
qwhite  & PsPaintWhite          \\ \hline
                                   \hline
qinittc    & PsInitColorTable   \\ \hline
qsettc     & PsSetColorTable    \\ \hline
qsetcpen   & PsSetColorPen      \\ \hline
qsetcrange & PsSetColorRange    \\ \hline
                                   \hline
qlwidth & PsSetLineWidth        \\ \hline
qpsize  & PsSetPointSize        \\ \hline
                                   \hline
qcomm   & PsComment             \\ \hline

\end{tabular}
\end{center}
\caption{Corresponding calls for Fortran and C}
\end{table}

\MakeShortVerb{\|}

For reasons of completeness we first give the example of section
\ref{Example} in C.

\begin{verbatim}
#include "pspost.h"

void main( void )

{
        PsGraphOpen();
        PsStartPage();
        PsLine(1.,1.,10.,10.);
        PsEndPage();
        PsStartPage();
        PsText(2.,2.,"Hello World!");
        PsEndPage();
        PsGraphClose();
}
\end{verbatim}

This example generates exactly the same output as does the
Fortran example.  The C prototypes are the following.

\begin{verbatim}

void PsGraphOpen(void);
void PsGraphClose(void);
void PsStartPage(void);
void PsEndPage(void);

void PsGetViewport(float *left,float *bottom
                             ,float *right,float *top);
void PsSetViewport(float left,float bottom
                             ,float right,float top);
void PsClearViewport(void);
void PsSetWorld(float xmin,float ymin,float xmax,float ymax);
void PsRectifyScale(void);

void PsLine(float x1,float y1,float x2,float y2);
void PsMove(float x,float y);
void PsPlot(float x,float y);
void PsPoint(float x,float y);

void PsAreaFill(int ndim,float *x,float *y);
void PsRectFill(float x1,float y1,float x2,float y2);

void PsTextPointSize(int size);
void PsTextFont(char *font);
void PsText(float x,float y,char *s);
void PsTextDimensions(char *s, float *width, float *height);

void PsSetGray(float gray);
void PsSetRGB(float red,float green,float blue);
void PsSetHSB(float hue,float sat,float bri);
void PsSetHue(float hue);
void PsPaintWhite(int ipaint);

void PsSetLineWidth(float width);
void PsSetPointSize(float size);

void PsComment(char *s);

\end{verbatim}

\DeleteShortVerb{\|}


\end{document}
