The sediment transport module calculates sediment transport for currents 
only or combined waves and currents over either cohesive and non-cohesive
sediments. The core of this module is derived by the SEDTRANS05 
sediment transport model, which is here coupled with SHYFEM. The sediment 
transport module computes the erosion and deposition rates at every mesh 
node and determines the sediment volume that is injected into the water 
column. After this step the sediments are advected with the transport and 
diffusion module described above. The module update every time step the 
characteristcs of the bottom in terms of grainsize composition and 
sediment density.

The sediment transport module is activated by setting |isedi| = 1 
and |sedgrs| in the section |sedtr|. 
For more details about the parameters see the Appendix C.

For more information about the sediment transport module please refer 
to Neumeier et al. \cite{urs:sedtrans05} and Ferrarin et al. 
\cite{ferrarin:morpho08}.

\subsubsection{Sediment transport formulation}
For the non-cohesive sediment there are two transport mechanisms, the 
bedload and the suspended transport, while for the cohesive sediment 
is assumed to exist only the suspended transport. Then we use the 
sediment continuity equation for the non-cohesive bedload transport 
and the transport and diffusion equation to describe the transport
of the suspended sediment, both cohesive and non-cohesive.
For the bedload component a direct advection scheme is used. 
Five methods are proposed to predict sediment transport for non-cohesive
sediments. The methods of Brown \cite{brown:engi}, Yalin \cite{yalin:bedload} 
and Van Rijn \cite{vanrijn93:prin} predict the bedload transport. The 
methods of Engelund and Hansen \cite{eh:momo} and Bagnold 
\cite{bagnolds:ma-sed} predict the total load transport.

Different approaches have been used to compute the sediment flux 
between water column and bottom for cohesive and non-cohesive sediment.
For cohesive sediments the model computes erosion and deposition rates,
while for non-cohesive the net sediment flux between bottom and water 
column s computed as the difference between the equilibrium
concentration and the existing concentration in the lower level.
Here the source (erosion occurs, flux from the bottom to the water 
column) term has been taken as explicit, whereas the sink term as 
semi-implicit (deposition occurs, flux from the water column to 
the bottom). This approach permits to avoid negative concentration 
due to deposition higher then the sediment mass present in the water 
column.

The vertical mixing coefficient has been calculated  using analytical
expressions given by \cite{vanrijn93:prin} for both the case of current
related and wave related turbulent mixing.

\subsubsection{Bed representation}
The bed module is designed to have spatially different characteristics, such 
as grainsize composition, sediment density and critical stress for erosion.
The bed is subdivided in several layers and levels. Each layer is considered
homogeneous, well mixed and characterized by its own grainsize distribution
(fraction of each class of sediment considered). On the level are defined
the dry bulk density $\rho_{dry}$ and the critical stress for erosion
$\tau_{ce}$; it is assumed that these variables vary linearly between two
levels. These characteristics could vary spatially in the domain.

At each location the uppermost layer has to be always greater or equal to
the surficial active, or mixed, layer that is available for suspension.
Sediment below the active layer is unavailable resuspension until the 
active layer moves downward either because erosion has occurred, or it 
has thickened due to increase shear stress. As active layer is considered 
the bottom roughness height considered as the sum of the grain roughness, 
the bedload roughness and the bedform (ripple) roughness.

Multiple sand grain size classes are considered to behave independently. At
each location the average grain size (based on the sediment fractions) is
used to compute the bed roughness and critical shear velocities.
Modification of the bed elevation and to the grain size distribution are
updated at each time step based on the net erosion and deposition.
For each size class, the volume of sediment removed from the bed during any
time step is limited by the amount available in the active layer. In this
way the model takes into account time-dependent and spatial sediment
distribution and bed armouring.

\subsubsection{Sand-mud mixture}
The morphological behaviour of estuaries and lagoons often depend on
non-cohesive as well as cohesive sediments. Prediction the distribution
of sediments in these environments, characterized by zonation of sand,
mud and mixed deposits, is of crucial importance for sustainable
management and development of such systems.

Based on laboratory and field experiments several researchers identified a
transition from non-cohesive to cohesive behaviour at increasing mud content
in a sand bed. A sand bed with small amount of mud already shows increased
resistance against erosion. Above a critical mud content (\% $<$ 0.063 mm), 
the bed behaved cohesively. The critical mud content depend on the history 
of the bed and on the geochemical properties of the sand and mud and could 
varies from 3-20 \%. Such a wide range clearly demonstrate that the 
parameters governing the erosion behaviour of sand-mud mixture, are
 not fully understood yet. For this reason is crucial to estimate 
the critical mud or clay content from laboratory and field experiments.

Below the critical mud content the sediments particles are eroded as
non-cohesive sediments. It is assumed that the sediments in suspension
always behave independently, either if flocculation processes of the
thinner particles could trap sand grains into the floc.

Moreover sand increases the binding between the clay particles and results
in a more compact and dense matrix which is more resistant to erosion. For
non-consolidated cohesive bed adding sand to mud increase the erosion
resistance because of the increased density and consolidation rate.
The improved critical stress for erosion due to sand particles is taken
into consideration increasing the dry density with the sand fraction.

\subsubsection{Sediment model output}
The sediment transport model writes the following output:
\begin{itemize}
\item erosion/deposition [m], 2D variable 80 in file SED
\item average grainsize of first bottom layer [m], 2D variable 81 in file SED 
\item bed shear stress [Pa] of first bottom layer, 2D variable 82 in file SED
\item updated node depth [m], 2D variable 83 in file SED
\item total bedload transport [kg/ms], 2D variable 84 in file SED
\item total suspended concentration [kg/m3], 3D variable 85 in file SCO
\end{itemize}

The time step and start time for writing to file SED and SCO
are defined by the parameters |idtcon| and |itmcon| in the |para|
section. These parameter are the same used for writing tracer
concentration, salinity and water temperature. If |idtcon| is not
defined, then the sediment model does not write any results.


