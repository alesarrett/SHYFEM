
% $Id$

% todo:
%
% windos users -> cygwin

\documentclass{report}

\usepackage{a4}
\usepackage{shortvrb}
\usepackage{pslatex}

\usepackage{verbatim}
\usepackage{alltt}		% as verbatim, but interpret \ { }

\newenvironment{code}{\verbatim}{\endverbatim}
\newenvironment{codem}{\alltt}{\endalltt}	%interpret \
\newcommand{\ttt}[1]{{\tt #1}}

\newcommand{\shy}{{\tt SHYFEM}}
\newcommand{\psp}{{\tt SHYFEM}}

\input{P_version.tex}
\newcommand{\shyname}{shyfem-\version}
\newcommand{\shydist}{\shyname.tar.gz}
\newcommand{\basedir}{/home/model}
\newcommand{\shydir}{\basedir/shyfem-\version}

\newcommand{\descrpsep}{\vspace{0.2cm}}
\newcommand{\descrpitem}[1]{\descrpsep\parbox[t]{2cm}{#1}}
\newcommand{\descrptext}[1]{\parbox[t]{10cm}{#1}\descrpsep}
\newcommand{\densityunit}{kg\,m${}^{-3}$}
\newcommand{\accelunit}{m\,s${}^{-2}$}
\newcommand{\maccelunit}{m${}^{4}$\,s${}^{-2}$}
\newcommand{\dischargeunit}{m${}^3$\,s${}^{-1}$}

\newcommand{\ten}[1]{$\cdot 10^{#1}$}
\newcommand{\degrees}{${}^o$}

\newcommand{\figref}[1]{\ref{fig:#1}}
\newcommand{\Fig}{Fig.~}

\newcommand{\todo}[1]{This section still has to be written by #1}

\parindent 0cm

\newcommand{\importstr}[2]{%
\begin{figure}
\begin{alltt}
\input{#1.str}
\end{alltt}
\caption{#2}
\label{fig:#1}
\end{figure}
}

\MakeShortVerb{\|}

\hyphenation{ ba-thy-me-try }

%%%%%%%%%%%%%%%%%%%%%%%%%%%%%%%%%%%%%%%%%%%%%%%%%%%%%%%%% front matter

\title{%
	\shy{} 
	\\Finite Element Model for Coastal Seas
	\\~
	\\User Manual
	}

\author{%
	The SHYFEM Group
	\\Georg Umgiesser
	\\Oceanography, ISMAR-CNR
	\\Arsenale Tesa 104, Castello 2737/F
	\\30122 Venezia, Italy
	\vspace{0.5cm}
	\\georg.umgiesser@ismar.cnr.it
	\vspace{1cm}
	\\Version \VERSION
	}

%\address{ISDGM-CNR}

%\date{}		%uncomment for no date

%%%%%%%%%%%%%%%%%%%%%%%%%%%%%%%%%%%%%%%%%%%%%%%%%%%%%%%%% document

\begin{document}

\bibliographystyle{plain}

\pagenumbering{roman}
\pagestyle{plain}

\maketitle

%\begin{abstract}
% to be written
%\end{abstract}

\thispagestyle{empty}

\newpage
\tableofcontents
\newpage

%%%%%%%%%%%%%%%%%%%%%%%%%%%%%%%%%%%%%%%%%%%%%%%%%%%%%%%%%%%%%%%%%%%%%%%%
%%%%%%%%%%%%%%%%%%%%%%%%%%%%%%%%%%%%%%%%%%%%%%%%%%%%%%%%%%%%%%%%%%%%%%%%
%%%%%%%%%%%%%%%%%%%%%%%%%%%%%%%%%%%%%%%%%%%%%%%%%%%%%%%%%%%%%%%%%%%%%%%%

\chapter*{Disclaimer}
	\addcontentsline{toc}{chapter}{Disclaimer}

	
\begin{quotation}

   Copyright (c) 1992-2012 by Georg Umgiesser

   Permission to use, copy, modify, and distribute this software
   and its documentation for any purpose and without fee is hereby
   granted, provided that the above copyright notice appear in all
   copies and that both that copyright notice and this permission
   notice appear in supporting documentation.

   This file is provided AS IS with no warranties of any kind.
   The author shall have no liability with respect to the
   infringement of copyrights, trade secrets or any patents by
   this file or any part thereof.  In no event will the author
   be liable for any lost revenue or profits or other special,
   indirect and consequential damages.

   Comments and additions should be sent to the author:

        \begin{verbatim}
                      Georg Umgiesser                                  
                      Oceanography, ISMAR-CNR
                      Arsenale Tesa 104, Castello 2737/F
                      30122 Venezia
                      Italy

                      Tel.   : ++39-041-2407943
                      Fax    : ++39-041-2407940
                      E-Mail : georg.umgiesser@ismar.cnr.it
        \end{verbatim}
\end{quotation}



\newpage
\pagenumbering{arabic}

%%%%%%%%%%%%%%%%%%%%%%%%%%%%%%%%%%%%%%%%%%%%%%%%%%%%%%%%%%%%%%%%%%%%%%%%
%%%%%%%%%%%%%%%%%%%%%%%%%%%%%%%%%%%%%%%%%%%%%%%%%%%%%%%%%%%%%%%%%%%%%%%%
%%%%%%%%%%%%%%%%%%%%%%%%%%%%%%%%%%%%%%%%%%%%%%%%%%%%%%%%%%%%%%%%%%%%%%%%

% external software
% contributors

\chapter{Overview}

	\section{What is it}
	

The finite element program \shy{} is a program package that can be used
to resolve the hydrodynamic equations in lagoons, coastal seas,
estuaries and lakes. The program uses finite elements for the
resolution of the hydrodynamic equations. These finite elements,
together with an effective semi-implicit time resolution algorithm,
makes this program especially suitable for application to a complicated
geometry and bathymetry.

This version of the program \shy{} resolves the depth integrated
shallow water equations. It is therefore recommended for the
application of very shallow basins or well mixed estuaries. Storm surge
phenomena can be investigated also.  This two-dimensional version of
the program is not suited for the application to baroclinic driven
flows or large scale flows where the the Coriolis acceleration is
important.

Finite elements are superior to finite differences when dealing with
complex bathymetric situations and geometries. Finite differences are
limited to a regular outlay of their grids. This will be a problem if
only parts of a basin need high resolution.  The finite element method
has an advantage in this case allowing more flexibility with its
subdivision of the system in triangles varying in form and size.

This model is especially adapted to run in very shallow basins. It is
possible to simulate shallow water flats, i.e., tidal marshes that in a
tidal cycle may be covered with water during high tide and then fall
dry during ebb tide. This phenomenon is handled by the model in a mass
conserving way.

Finite element methods have been introduced into hydrodynamics since
1973 and have been extensively applied to shallow water equations by
numerous authors \cite{Grotkop73, Taylor75, Herrling77, Herrling78, Holz82}.

The model presented here \cite{Umgies86, Umgies93} uses the mathematical
formulation of the semi-implicit algorithm that decouples the solution
of the water levels and velocity components from each other leading to
smaller systems to solve. Models of this type have been presented from
1971 on by many authors \cite{Kwizak71, Duwe82, Backhaus83}.




	%\section{Why use it}
	%\todo{Georg}

	%\section{How to use it}
	%\todo{Georg}

	%\section{Where to get it}
	%\todo{Georg}

	%\section{Features}
	%\todo{Georg}

	%\section{License}
	%\todo{Georg}

%%%%%%%%%%%%%%%%%%%%%%%%%%%%%%%%%%%%%%%%%%%%%%%%%%%%%%%%%%%%%%%%%%%%%%%%
%%%%%%%%%%%%%%%%%%%%%%%%%%%%%%%%%%%%%%%%%%%%%%%%%%%%%%%%%%%%%%%%%%%%%%%%
%%%%%%%%%%%%%%%%%%%%%%%%%%%%%%%%%%%%%%%%%%%%%%%%%%%%%%%%%%%%%%%%%%%%%%%%

\chapter{Installing SHYFEM}

	\section{Downloading and unpacking}
	
The source code of the model is provided in a file named \ttt{\shydist}
or similar, depending on the version of the code. In this case the
version is \version.  The file can be downloaded from the SHYFEM
web-site\footnote{http://www.ismar.cnr.it/shyfem/}, or by contacting
directly its authors.

Once you have downloaded the model distribution, move the file to
the directory in which you want to install the model and unpack the
distribution. In the following we will assume that the file is in your
home directory and your home directory is called \ttt{\basedir}. However,
any other directory works as well. To unpack the distribution
in your home directory, move there and run the command:

\begin{codem}
    cd \basedir
    tar -xzvf \shydist
\end{codem}

At this point a new folder named \ttt{\shydir} has been created. 
This directory is the root of the SHYFEM model. All other commands
given in this chapter assume that you are in this directory. 
Therefore, before reading on, please move into this directory:

\begin{codem}
    cd \shydir
\end{codem}






	\section{Needed software}
	
The source code is composed mainly of Fortran 77 files, but files written
in C, Fortran 90, Perl and Shell scripts are also present.

In order to use the model you have to compile it in a Linux Operating
System. Several software products must be present in order to be able
to compile the model. Please refer to the documentation of your Linux
distribution for installing these programs.

\begin{itemize}

\item The package |make| is required for compilation.

\item The |perl| interpreter, the |bash| shell and the |gcc| C compiler
are necessary for compiling.

\item A Fortran 77 and 90 compiler. Supported compilers are the Gnu |g77|
and |g90| compilers, the new Gnu compiler |gfortran|, the Intel Fortran
compiler |ifort| or the Portland group |pgf90| Fortran compiler.

\end{itemize}

Please note that you might already have everything available in your
Linux distribution, with the exception maybe of the Fortran compiler.

To find out what software is installed on your computer and what you
still have to install you can run the following command:

\begin{code}
    make check_software
\end{code}

If you get something like |bash: make: command not found|, then you do
not have make installed. Please first install the |make| command and
then run the command again.

The output of the command will show you what software you will still have
to install. The software is divided into different sections. The first
section is needed software, which you will not be able to do without. The
next section is recommended software, which you really should install,
but for compilation and running you will not necessarily need it. The
last section is software which is optional, but which makes life easier.

You can always run |make check_software| again to check if the software
had been successfully installed. When you are satisfied with the output
you can go to the next section.

Please note that you have to carry out the steps in this section only
the first time you install the model. If you install a new version of
SHYFEM software you can skip these steps.






	\section{Installation}
	
\newcommand{\sysfiles}{.bashrc .bash\_profile .profile}

Before compiling it is advisable to install some files for a simpler
usage of the model. As long as you only want to tun a simulation, this
step is not strictly necessary. But if you will run some scripts of the
distribution, these scripts will not work properly if you do not install
the model.

In order to install the model, you should run

\begin{code}
    make install
\end{code}

This command will do the following:

\begin{itemize}

\item It hardcodes the installation directory in all scripts of the
model so only programs of the installed version will be executed.

\item It inserts a symbolic link |syhfem| from the home directory to
the root of the SHYFEM installation.

\item It inserts a small snippet of code into the initialization files
\ttt{\sysfiles} that are in your home directory. This will adjust your
path to point to the SHYFEM directory and gives you access to some
administrative commands.

\end{itemize}

After this command you will find the original files that have been changed
in your home directory saved with a trailing number (e.g., |.profile.35624|
or similar).  If you encounter problems, just substitute back these files.

In order that your new settings will take effect you will have to log
out and log in again. Alternatively, you will have to execute one of
the files \ttt{\sysfiles} that have actually changed. If |.bashrc| has been
changed, then run |. ~/.bashrc|.

If you do not want to run the installation routine, you should at least
manually insert a symbolic link to the root of the SHYFEM model.
Otherwise some of the commands and shell scripts will not work
properly. This can be done with the command

\begin{codem}
    cd
    ln -s \shydir shyfem
\end{codem}

from your home directory. If there is already such a link existing you
will first have to delete it (|rm shyfem|).

You should also think about the possibility to add the fembin path to your
default paths to have the main utility commands always available. To do
this open your |.bashrc| file in the home directory and add the following
lines at the end of the file:

\begin{verbatim}
SHYFEMDIR=$HOME/fem
PATH=$PATH:$SHYFEMDIR/fembin
export SHYFEMDIR PATH
\end{verbatim}

However, the same effect can be achieved more easily by using the
abovementioned command |make install|.

If you ever want to uninstall the model, you can do it with the command
|make uninstall|. This will delete the symbolic link, cancel the hard
links in the model scripts and restore the systemfiles \ttt{\sysfiles}
to their original content.

Please note that you will still have to delete manually the model
directory. This can be done with the command \ttt{rm -rf \shydir}). In
this way, however, changes to the code you have made will be lost.



	\section{Compilation}
	
In order to compile the model you will first have to adjust some settings
in the |Rules.make| file. Assuming that you are already in the SHYFEM
root directory (in our case it would be \ttt{\shydir}), open the file
|Rules.make| with a text editor.  In this file the following options
can be set:

\begin{itemize}

\item |Parameters|. In this section you have to set the maximum number
of nodes (|NKNDIM|) and elements (|NELDIM|) used by your grids. You
might also have to set also the maximum number of elements attached to
a node (|NGRDIM|) and the maximum bandwidth (|MBWDIM|) of the z-level
matrix. You can find these numbers when you create the basin file with
|vpgrd|. Finally you have to specify the maximum number of vertical levels
(|NLVDIM|).  It is advisable to set this value close to the desired number
of vertical levels, since it affects the model speed performance. So,
if you want to run the model in 2D mode, please set |NLVDIM| to 1.
For all other possible parameter settings please have a look at |param.h|.
However, you should never directly change this file. Always make changes
to the parameters in the |Rules.make| file.

\item |Compiler|. Set the compiler you want to use. Please see also
the section on needed software and the one on compatibility problems to
learn more about this choice.

\item |Parallel compilation|. Some parts of the code are parallelized
with OpenMP statements. Here you can set if you want to use it or not.
All supported compilers (except |g77|) accept OpenMP statements.

\item |Solver for matrix solution|. There are three
different solvers implemented.  The |GAUSS| solver is the
most robust and best tested solver, but it is quite slow. The
|PARDISO| solver needs an external library available at the Intel
web-site\footnote{http://software.intel.com/en-us/articles/code-downloads/},
that can be freely downloaded for non-commercial use.  The Pardiso
solver is parallelized, but it seems to be a little slower than the
|SPARSKIT| solver.  The |SPARSKIT| solver is the recommended solver,
since it seems to be the fastest one. However, if you are ever in doubt
about your results you might want to revert back to the |GAUSS| solver
and check the results.

\item |NetCDF library|. If you want output files in NetCDF format
you need the NetCDF library.

\item |GOTM library|. The GOTM turbulence model is already included in
the code. However, a newer and better tested version is available as an
external module. In order to use it please set this variable to true. This
is the recommended choice. You will need a Fortran 90 compiler to enable
this choice.

\item |Ecological module|. This option allows for the inclusion of an
ecological module into the code. Choices are between |EUTRO|, |ERSEM|
and |AQUABC|. Please refer to information given somewhere else on how
to run these programs.

\item |Compiler options|. Here several sections are present, one for
each supported compiler. Normally it should not be necessary to change
anything beyond this point.

\end{itemize}

Once you have set all these options you can start compilation with

\begin{code}
    make clean
    make fem
\end{code}

This should compile everything. In case of a compilation error you will
find some messages during compilation and also at the bottom of the output,
where a check is run to see, if the main routines have been compiled.

Please remember that you will always have to run the commands above
when you change settings in the |Rules.make| file. If you only change
something in the code, or if you only change dimension parameters, it
might be enough to run only |make fem|, which only compiles the necessary
files. However, if you are in doubt, it is always a good idea to run
|make clean| or |make cleanall| before compiling, in order to start from
a clean state.



	\section{Compatibility problems}
	
The SHYFEM program is designed to work with most of the compilers
that are available. Normally there should be no problems with
compatibility. However you have to keep in mind some points that are
listed below.

The supported compilers for SHYFEM are

\begin{itemize}

\item {\bf g77} This the old GNU compiler. It is Fortran 77 only. It
is also a little outdated and is not supported anymore by the major
distributions. If you do not need software that relies on Fortran 90 it
is still possible to use it. However, GOTM and some ecological modules
are relying on Fortran 90 and cannot be used with the g77 compiler.

\item {\bf gfortran} This is the actual GNU compiler. It also
supports Fortran 90. This is the compiler that you will find in recent
distributions.

\item {\bf ifort} This is the INTEL compiler. You will have to download
and install it on your own. It is very efficient and normally faster
than the GNU compiler.

\item {\bf pgf90} This is the Portland group compiler. It is a commercial
compiler. It creates very efficient code.

\end{itemize}

Whatever compiler you choose to use, you have to set your choice in
the |Rules.make| file. Even if not desirable, different compilers can
give you slightly different results in the computations. This is due to
the different optimizations enabled and maybe a different treatment of
accuracy and round off. Other compatibility issues are the following.

\begin{itemize}

\item With |g77| and |ifort| it is possible to open the same file in
read only mode more than once. This is useful, e.g., if you have two
open boundaries, but you want to prescribe the same value on these
two boundaries. With |gfortran| or |pgf90| you cannot do this. A file,
even in read only mode, can be opened only once. In the above example
you therefore have to copy the input file to a new name (duplicate it)
and then prescribe the two different files as boundary conditions.

\item With |gfortran| it is very difficult to decide if a file is
formatted or unformatted. Some modules allow the use of either formatted
or unformatted input files, where the check on the file type is made
via software. In case of |gfortran| this may not work reliably. The only
solution to this problem is to specify the file type directly in the code.

\item Objects generated during compilation and libraries used in linking
are normally not compatible between compilers. What this means is that,
when you switch compiler, you will have to recompile everything with
|make cleanall; make fem|. Otherwise you will encounter errors during
the linkage process.

\item Unformatted files are normally not portable between different
compilers. You normally cannot use a basin file created with programs
compiled with one compiler together with a program compiled with another
compiler. The same is true for unformatted data files (initial conditions,
wind and meteo forcing, etc.).

If you have problems reading a basin file, try |basinf|. If this is
not working chances are high that you have the problem described above.
In case of unformatted data files the diagnosis is not so easy. In any
case, you can solve the problem recompiling all programs with the commands
|make cleantotal; make fem| and then re-creating all unformatted files
with the newly compiled programs. In case of the basin file, you will
have to run the pre-processor on the grid again.

If you have obtained unformatted data files from others, then there is
really no easy solution to this problem. Exchanging unformatted files
between different computers and compilers is never a good idea.

\item A similar problem exists if you switch files between different
architectures (32 bit and 64 bits), even if created with the same
compiler. These files are normally not portable.

\item Nan values (Not a Number) are treated differently between different
compilers. Nan values are created if a not well defined operation is
executed (divide by 0 or square root of a negative number). All compilers
above (except |pgf90|) treat Nans to be not comparable to any number.
This means that a logical expression |a.eq.a| is always false if |a|
is a Nan. However the |pgf90| compiler treats Nans to be comparable
to any other number. So, an expression like |a.ne.a| will evaluate to
true. SHYFEM includes code to handle these problems gracefully, but
incompatibilities might still show up.

\item In parallel execution you might get a segmentation fault during
execution. This is normally due to limited stack size. You can change
the behavior by increasing the stack size (|ulimit -s unlimited|)
on the console before running the program. Compilers may behave
differently. Please see also the section on parallel execution in the
file |Rules.make|.

\end{itemize}








	\section{Summary}
	
In summary, the following steps have to be carried out before you will
be able to run the model:

\begin{itemize}

\item Get the distribution and unpack it in a place of your choice.

\item Move to the root of the distribution (\ttt{e.g., cd \shydir}).

\item Check if all software is available (|make check_software|). This
step has to be done only the first time you install SHYFEM on a computer.

\item Install the model (|make install|). This has to be done every time
you get and install a new version of the model.

\item Adjust options in |Rules.make|. This has to be done every time you
change options (compiler, parallel execution, etc.). After this you have
to run also |make cleanall|.

\item Adjust dimension parameters in |Rules.make|. This has to be done the
first time and every time you change application (basin, etc.) to adapt
the dimensions to the new problem. You might also run |make cleanall|
after this step, but it is not required.

\item Compile the programs with |make fem| and have a look at the error
messages.

\end{itemize}

Moreover, below a summary of administrative commands is given that are
available in SHYFEM. \vspace{0.5cm}

\begin{center}
\begin{tabular}{ l p{7cm} }
|make version|		&	shows version of distribution \\
|make clean|		&	deletes objects and executables from a previous
                        	compilation \\
|make cleanall|		&	same as |make clean| but also deletes 
				compiled libraries \\
|make fem|		&	compiles SHYFEM \\
|make doc|		&	makes this manual (|femdoc/shyfem.pdf|) \\
|make check_software|	&	checks the availability of installed software \\
|make check_compilation|&	checks if all programs have been compiled \\
|make changed|		&	finds files that are changed with respect to the
				original distribution \\
|make changed_zip|	&	zips files that are changed with respect to the
				original distribution to the file 
				|changed_zip.zip| \\
|make install|		&	installs SHYFEM \\
|make uninstall|	&	uninstalls SHYFEM \\
\end{tabular}
\end{center}

\vspace{0.5cm}
Finally, if you have installed the model with |make install|, 
the following utility commands are available \vspace{0.5cm}

\begin{center}
\begin{tabular}{ l l }
|shyfemdir|		&	shows information about actual SHYFEM
				settings \\
|shyfemdir fem_dir|	&	sets |fem_dir| to be the new default 
				SHYFEM version \\
|shyfeminstall|		&	shows information about original SHYFEM 
				installation \\
|shyfemcd|		&	moves into root of actual SHYFEM directory \\
\end{tabular}
\end{center}


%%%%%%%%%%%%%%%%%%%%%%%%%%%%%%%%%%%%%%%%%%%%%%%%%%%%%%%%%%%%%%%%%%%%%%%%
%%%%%%%%%%%%%%%%%%%%%%%%%%%%%%%%%%%%%%%%%%%%%%%%%%%%%%%%%%%%%%%%%%%%%%%%
%%%%%%%%%%%%%%%%%%%%%%%%%%%%%%%%%%%%%%%%%%%%%%%%%%%%%%%%%%%%%%%%%%%%%%%%

\chapter{Preprocessing: The numerical grid}

	\section{Overview}
	
Before you can start using the model you have to create a numerical grid.
Ths step is more difficult for models that work on unstructured grids
(like finite element models) than for finite difference models, where
often it is enough to have a regular gridded bathymetry to start running
simulations.

This chapter describes the steps that you have to take in order to be able
to create a numerical grid for SHYFEM. If you are in the happy position
to already have a numerical finite element grid, then you can jump ahead
to the section on how to transform to/from other grid formats. You might
still have to interpolate a bathymetry onto the numerical grid, so you
will have to refer to the section on interpolating bathymetry. In any
case, at the end you will have to create a (unformatted) basin file,
so that SHYFEM is able to read in the information.

The steps that have to be carried out to create a numerical grid are

\begin{enumerate}

\item obtain raw digital data of the coastline and the bathymetry

\item convert the digital data into a format GRD that is read by the
	provided routines

\item prepare the coastline

\item create the numerical grid with an automatic grid generator

\item regularize the grid

\item interpolate bathymetry onto the created grid

\item create the basin file (the finite element version of the grid file)

\end{enumerate}

As mentioned above, some of the steps can be skipped if you already
have a finite element grid. If you already have a grid with bathymetry
information you can jump to point 7. However, you will have to convert
your grid into the |GRD| format used by SHYFEM. The |GRD| format is
documented in the appendix. For some of the most common unstructured grid
formats routines are available to convert between these formats and the
|GRD| format. In any case, the |GRD| format is quite easy to parse and
write, so you might be able to write a transformation routine yourself.

In the following a description is given what you have to do if you start
from scratch. Please refer to the section on other programs to create
a grid for conversion routines.




	\section{Converting to {\tt GRD} format}
	
Data files with boundary line and bathymetry should be given. These
files have to be transformed into |GRD| files, that can be read and
manipolated with the programs |mesh| and |grid|. Examples of how to do
so can be found in coast.pl and ldb.pl for the coastline and bathy.pl
for the bathymetry points.

\begin{code}
    coast.pl mpcoast.dat > coast.grd
    bathy.pl mpbathy.dat > depth.grd
\end{code}

Please note that the coordinates for the GRD files should be always in
meters. Therefore, if you have your coordinates in other units, you have
to adjust the conversion routines in order to create the new coordinates
in meters.

Please note that UTM coordinates are in meters, so UTM coordinates are
fine. However, since UTM coordinates are normally huge numbers, there
might be an accuracy problem when you try to create the grid. If this
happens, you should first shift your UTM coordinates so that the origin
of your new coordinate system coincides with the central point of your
grid. This translation can be done using the program |grd_transl.pl|.

Other transformation routines are:

\begin{itemize}

\item |dxf2grd.pl|  Transforms a grid from |DXF| (Autocad) to |GRD|
format. This is still experimental.

\item |kml2grd.pl|  Transforms a grid from the Google Earth format |KML|
to |GRD| format.

\item |xyz2grd.pl|  Transforms a simple list of nodes to |GRD|
format. Every line contains 3 values $(x,y,z)$ or two values $(x,y)$,
when the information on depth is missing.

\end{itemize}

Please note that for SHYFEM depth values have to be positive. If your
files have depth values as negative numbers, you will have to invert
them. You can use the command

\begin{code}
    grd_modify.pl -depth_invert grd-file
\end{code}

to achieve this task.



	\section{Boundary line: smoothing and reducing}
	

With the routine |grid| the coastline can be viewed.
However, normally the line needs some post-processing.
It might either have resolution which is too high, island
might show up as open lines etc..

It is important that there is one closed boundary line that
defines the whole domain of the computation. If you have an
open coastline, please close the line with the routine |grid|
at the places where you want your open boundary to be.

Once this domain boundary line has been defined, care has
to be taken that the lines inside this domain, which denote
islands, are closed.

Finally the resolution of the boundary lines (coast and islands)
have to be adjusted. If the cosatline is left as it is you might
have a much too high resolution along the boundaries. This is due
to the fact that the meshing algorithm does not discard any points
given to it. This means that all boundary nodes are used for the meshing.
Therefore, if you have a very high resolution boundary line, you will
get many elements along the boundary amd reletively little elements
(depending on the number of internal points) in the inside of the
basin.

Smoothing and reduction of the boundary lines can be done with the
routine |reduce|. The command is

\begin{code}
    reduce -s sigma -r reduct coast.grd
\end{code}

Here |sigma| specifies the langth scall for the smoothing operator
and |reduct| is the length scale below which points may be deleted.
Both values have to be given in the same units of the coordinates
of the file |coast.grd|, so normally meters.
The smoothed file can be found in |smooth.grd| and the subsequently
reduced file in |reduct.grd|. 

If there are some points in the boundary line that should not be smoothed
they can be given a depth value of -1. This is a flag that indicates
that the position of these points will not be touched.




	\section{Other information}
	
\begin{verbatim}

Construct a background grid
===========================

If you want a grid with a uniform solution all over, then
you are already in a position to run the meshing algorithm.
You just say: "mesh -I2000 coast-new.grd" and then 
the constructed mesh will be in final.grd. The number 2000
means that you want aprox. 2000 internal points in the domain.
You may adjust this number to your needs.

However, you will normally want to have different resolution
in the domain (high at the inlets of lagoons, at interesting
sites like harbours etc..). Then you have to construct a
background grid that gives an indication to the meshing
algorithm what kind of resolution is need in what area.

You open the coastline with grid and construct elements
that cover the parts or all of the domain. The areas where
no background grid exists will use the (constant) resolution of the
domain computed by the routine mesh using the total number of
internal nodes (2000 in this example).

Where a background grid exists the model uses the depth values at the
element vertices (nodes) to compute a new value for this resolution.
The depth value acts like a factor that multiplies the constant
overall resolution to obtain a local resolution. So, for example,
constructing a background grid and setting all depth values to 1
would not change the resolution at all from a situation without
background grid. A factor higher than 1 increases the resolution
and one smaller than 1 decreses it. Therefore, in areas where
resolution should be higher than average you can set it to
2 or higher, and in other areas, where you want lower resolution,
you can set it to 0.5 or lower. All nodes of the background grid
need to have a depth (resolution) value. Inside each background
element the resolution is interpolated between the three nodes
(vertices).

In order to distinguish the background grid from the elements
that are constructed by the meshing routine, they must become 
a unique element type. You can set it to a value that is not
used for other elements (99 is a good choice). All elements
of the background grid must have this element type.

Please extract the background grid from the grid file you just
have constructed by running exgrd: "exgrd -LS coast-new.grd".
The file "new.grd" contains only the background grid. Rename it to
something more useful (mv new.grd bck1.grd). You are then
ready to start the meshing algorithm.

	manually construct background grid using coast-new.grd
	delete coastline (leave only background elements in file)
		exgrd -LS coast-new.grd
	set depth at nodes for resolution
	set type in elements to 99
	rename to bck1.grd
		mv new.grd bck1.grd


Meshing of the basin
====================

The meshing algorithm is called mesh. Please see "mesh -h" for
help of the command line options. The most important are:

	-I2000		use aprox. 2000 internal nodes for the domain
	-g99		element type of background grid is 99

With this parameters the call to mesh would be
	"mesh -I2000 -g99 coast-new bck1".
The created mesh can be found in final.grd.

Please note that you can specify more than one file for the coast line,
so you could keep the domain line and the island lines in seperate files.
You can also have different background grid for different areas in
different files. So a call like this is also possible:
	"mesh -I2000 -g99 coast island1 island2 bck1 bck2 bck3".

After the meshing please have a look at the result (final.grd).
If you need more overall resolution, increase the number of internal
points (here 2000). If you need more resolution in the background grid,
open the background file and increase the factor (depth) value where needed.
You might also need other areas with a background grid. Once you
are satisfied with the result please save it to a more meaningful name.

	mesh -I2000 -g99 coast-new bck1
	mv final.grd mesh1.grd


Adjust elements for regularity
==============================

After the creation of the mesh, the grid is still not good enough
for usage in a finite element model. This is due to the fact that
the grid is too irregular. Therefore a program has to be applied
that regularizes the grid. 

The program is called regularize. It must be given the input grid file
(irregular) and creates a new one with much more regular characteristics.
The program has to be called as:
	"regularize mesh1.grd mesh2.grd".
In this case the new regular grid is in mesh2.grd.


Interpolate bathymetry
======================

To interpolate bathymetry, a grd file with single points containig
depth values has to be available. This file, together with the basin
onto which the bathymetry has to be interpolated, has to be specified
for the program basbathy. The simplest call is:

	basbathy mesh2 bathy

where bathy.grd is the grd file with the bathymetry values and
mesh2 is the basin for which to interpolate the bathymetry.
Different types of interpolation can be used. Please run
"basbathy -h" for more options.

The new grd file will be in "new.grd".

	basbathy mesh2 bathy
	mv new.grd mesh3.grd


Create basin for FEM model (bandwidth optimization)
===================================================

Before proceeding to the simulations we must first create a 
representation of the basin suitable for the finite element model. 

In order to create the finite element reppresentation of the
grid, please run "vpgrd mesh3". This creates a file mesh3.bas.
This is a binary file suitable for being read by the finite
element model.

        vpgrd mesh3

\end{verbatim}















%	\section{Manipulating nodes, lines and elements: grid}
%	\todo{Francesca}

%	\section{Creating the mesh: mesh}
%	\todo{Francesca}

%	\section{Other programs to create a mesh}
%	\todo{Marco o Christian}

%	\section{Interpolating the bathymetry}
%	\todo{Francesca}

	\section{Creating the basin file}
	



The pre-processing routine |vp| is used to generate an
optimized version of the file that describes the basin
where the main program is to be run. In the following a
short introduction in using this program is given.

\subsection{The pre-processing routine {\tt vp}}

The main routine |hp| reads the basin file generated by
the pre-processing routine |vp| and uses it as the description
of the domain where the hydrodynamic equations have to be
solved.

The program |vp| is started by typing |vp| on the command line.
From this point on the program is interactive, asking you about
the basin file name and other options. Please follow the online
instructions.

The routine |vp| reads a file of type GRD. This type of file
can be generated and manipulated by the program |grid| which
is not described here. In short, the file GRD consists of
nodes and elements that describe the geometrical layout
of the basin. Moreover, the elements have a type and a depth.

The depth is needed by the main program |hp| to run the model.
The type of the element is used by |hp| to determine
the friction parameter on the bottom, since this parameter
may be assigned differently, depending on the various situations
of the bottom roughness.

This file GRD is read by |vp| and transformed into an
unformatted file BAS. It is this file that is then read
by the main routine |hp|. Therefore, if the name of the
basin is |lagoon|, then the file GRD is called |lagoon.grd|
and the output of the pre-processing routine |vp| is
called |lagoon.bas|.

The program |vp| normally uses the depths assigned to the
elements in the file GRD to determine the depth of the
finite elements to use in the program |hp|. In the case
that these depth values are not complete, and that all nodes
have depths assigned in the GRD file, the nodal values of the
depths are used and interpolated onto the elements. However,
if also these nodal depth values are incomplete or are missing
altogether, the program terminates with an error.

\subsection{Optimization of the bandwidth}

The main task of routine |vp| is the optimization of the 
internal numbering of the nodes and elements.
Re-numbering the elements is just a mere convenience. When
assembling the system matrix the contribution of
one element after the other has to be added to the system matrix.
If the elements are numbered in terms of lowest node numbers,
then the access of the nodal pointers is more regular in 
computer memory and paging is more likely to be inhibited.

However, re-numbering the nodes is absolutely necessary.
The system matrix to be solved is of band-matrix type.
I.e., non-zero entries are all concentrated along the
main diagonal in a more or less narrow band. The larger this
band is, the larger the amount of cpu time spent to
solve the system. The time to solve a band matrix
is of order $n \cdot m^2$, where $n$ is the size of the
matrix and $m$ is the bandwidth. Note that $m$ is normally
much smaller than $n$.

If the nodes are left with the original numbering, it is very likely
that the bandwidth is very high, unless the nodes in the
file GRD are by chance already optimized. Since the bandwidth $m$
is entering the above formula quadratically, the amount
of time spent solving the matrix will be prohibitive.
E.g., halving the bandwidth will speed up computations by
a factor of 4.

The bandwidth is equal to the maximum difference of node numbers
in one element. It is therefore important to re-number the
nodes in order to minimize this number. However, there exist
only heuristic algorithms for the minimization of this number.

The two main algorithms used in the routine |vp| are
the Cuthill McGee algorithm and the algorithm of Rosen. The first
one, starting from one node, tries to number all neighbors in
a greedy way, optimizing only this step. From the points
numbered in this step, the next neighbors are numbered.

This procedure is tried from more than one node, possibly
from all boundary nodes. The numbering resulting from this
algorithm is normally very good and needs only slight
enhancements to be optimum.

Once all nodes are numbered, the Rosen algorithm tries to
exchange these node numbers, where the highest difference
can be found. This normally gives only a slight improvement
of the bandwidth. It has been seen, however, that, if the
node numbers coming out from the Cuthill McGee algorithm
are reversed, before feeding them into the Rosen algorithm, 
the results tend to be slightly better. This step is also
performed by the program.

All these steps are performed by the program without
intervention by the operator, if the automatic optimization
of bandwidth is chosen in the program |vp|. The choices
are to not perform the bandwidth optimization at all
(GRD file has already optimized node numbering), perform
it automatically or perform it manually. It is suggested
to always perform automatic optimization of the bandwidth.
This choice will lead to a nearly optimum numbering of the
nodes and will be by all means good results.

If, however, you decide to do a manual optimization, please
follow the online instructions in the program.

\subsection{Internal and external node numbering}

As explained above, the elements and nodes of the basin are re-numbered 
in order to optimize the bandwidth of the system matrix and so
the execution speed of the program. 

However, this re-numbering of the node and elements is transparent
to the user. The program keeps pointers from the original numbering
(external numbers) to the optimized numbering (internal numbers).
The user has to deal only with external numbers, even if the 
program uses internally the other number system.

Moreover, the internal numbers are generated consecutively.
Therefore, if there are a total of 4000 nodes in the system, the internal
nodes run from 1 to 4000. The external node numbers,
on the other side, can be anything the user likes. They just must be
unique. This allows for insertion and deletion of nodes without
having to re-number over and over again the basin.

The nodes that have to be specified in the input parameter file
use again external numbers. In this way, changing the structure of
the basin does not at all change the node and element numbers in the
input parameter file. Except in the case, where modifications
actually touch nodes and elements that are specified in the 
parameter file.



%%%%%%%%%%%%%%%%%%%%%%%%%%%%%%%%%%%%%%%%%%%%%%%%%%%%%%%%%%%%%%%%%%%%%%%%
%%%%%%%%%%%%%%%%%%%%%%%%%%%%%%%%%%%%%%%%%%%%%%%%%%%%%%%%%%%%%%%%%%%%%%%%
%%%%%%%%%%%%%%%%%%%%%%%%%%%%%%%%%%%%%%%%%%%%%%%%%%%%%%%%%%%%%%%%%%%%%%%%

\chapter{Running SHYFEM}

	
In the following an overview is given on running the model
\shy{}. The model needs a parameter input file that is read
on standard input. Moreover, it needs some external files that
are specified in this parameter input file. The model produces
several external files with the results of the simulation. Again,
the name of this files can be influenced by the parameter input file



	\section{How to run: the Parameter Input File (STR)}
	

The model reads one input file that determines the behavior of the
simulation. All possible parameters can and must be set in this file.
If other data files are to be read, here is the place where to specify
them.

The model reads this parameter file from standard input. Thus, if
the model binary is called |hp| and the parameter file |param.str|, 
then the following line starts the simulation
\begin{verbatim}
	ht < param.str
\end{verbatim}
and runs the model.

\subsection{The General Structure of the Parameter Input File}

The input parameter file is the file that guides program
performance. It contains all necessary information for the main routine
to execute the model. Nearly all parameters that can
be given have a default value which is used when the parameter
is not listed in the file. Only some time parameters are compulsory
and must be present in the file.

The format of the file looks very like a namelist format, but is
not dependent on the compiler used. Values of parameters are given
in the form :  
|name = value|  or  |name = 'text'|.  If |name|
is an array the following format is used : 
\begin{verbatim}
          name = value1 , value2, ... valueN
\end{verbatim}
The list can continue on the following lines. Blanks before and after
the equal sign are ignored. More then one parameter can be present
on one line. As separator blank, tab and comma can be used.

Parameters, arrays and data must be given in between certain sections.
A section starts with the character {\tt \$} followed by a keyword and
ends with {\tt \$end}. The {\tt \$keyword} and {\tt \$end} must not
contain any blank characters and must be the first non blank characters
in the line. Other characters following the keyword on the same line
separated by a valid separator are ignored.

Several sections of data may be present in the input parameter file.
Further ahead all sections are presented and the possible
parameters that can be specified are explained. The sequence in
which the sections appear is of no importance. However, the first 
section must always be section |\$title|, the section that
determines the name of simulation and the basin file to use and
gives a one line description of the simulation.

Lines outside of the sections are ignored. This gives
the possibility to comment the parameter input file.

Figure \ref{fig:example} shows an example of a typical input
parameter file and the use of the sections and definition of
parameters.

\importstr{example}
{Example of a parameter input file ({\tt STR} file)}



%	\section{Adjusting to the basin}
%	\todo{Georg}

%	\section{Basic file formats}
%	\todo{Georg}

	\section{Basic usage}

		
This section explains typical usage of the model. It will show how
the model can be run doing basic 2D hydrodynamic simulations, simulate
a passive tracer, compute T/S, use the Coriolis force and apply wind
forcing. More advanced usages of the model, like 3D simulations and the
use of the turbulence module will be presented later. This section is
conceived as a simple HOWTO document. For the exact meaning and usage
of the single parameters, please see the section on input parameters.

To run a simulation, two things are needed. The first is the description
of the basin and the numerical grid, which must be prepared beforehand and
then must be compiled in a form that the model can use. How this is been
done has already been described in the chapter dealing with preprocessing.

The second thing that is needed is a description of the simulation and the
forcings that have to be applied. This is done through a parameter input
file. Here we call it |STR| file, because historically these files always
ended with an extension of |.str|. However, any extension can be used.



		\subsection{Minimal simulation}
		
\importstr{basic}
{Example of a basic parameter input file ({\tt STR} file)}

A basic version of an |STR| file can be found in \ref{fig:basic}. In
fact, it is so basic, it really does not do anything. Here only the
compulsory parameters have been inserted. These are:

\begin{itemize}

\item An introductory section |$title| where on three lines the following
information is given:

\begin{enumerate}
\item A description of the run. This can be any text that fits on one line.
\item The name of the simulation. This name is used for all files that 
the simulation produces. These files differ from each other only by 
their extension.
\item The name of the basin. This is the basin file without the extension
|.bas|.
\end{enumerate}

\item A section |$para| that contains all necessary parameters for the
simulation to be run. The only compulsory parameters are the ones that
specify the start of the simulation |itanf|, its end |itend| and its 
time step |idt|.

\end{itemize}

In order to be more helpful, some more information must be added to the
|STR| file. As an example let's have a look on \figref{example}. Here
we have added two parameters that deal with the type of friction
to be used. |ireib| specifies the bottom friction formulation, here
through a simple quadratic bulk formula. (For the exact meaning of the
parameters, please refer to the appendix where all parameters
are listed.) The parameter |czdef| specifies the value to use for the
bottom drag coefficient.



%		
This section explains typical usage of the model. It will show how the
model can be run doing basic 2D simulations, compute T/S, do 3D simulations,
set up the turbulence module etc. This section is conceived as
a simple HOWTO document. For the exact meaning and usage of the single
parameters, please see the section on input parameters.

\subsubsection{2D Hydrodynamic Simulation}

To run a simulation, two things are needed. The first is the description
of the basin and the numerical grid, which must be prepared beforehand
and then must be compiled in a form that the model can use. This is typically
done by the routine |vp| that, starting from a file |.grd| creates a file
|.bas|. This will be called the basin file from now on.

The second thing that is needed is a description of the simulation and
the forcings that have to be applied. This is done through a 
input parameter description file. Here we call it a |STR| file, because
historically these files always ended with an extension of |.str|. However,
any extension can be used.

\begin{figure}
\begin{alltt}
\input{basic.str}
\end{alltt}
\caption{Example of a basic parameter input file ({\tt STR} file)}
\label{fig:str_basic}
\end{figure}

A basic version of an |STR| file can be found in \ref{fig:str_basic}. In
fact, it is so basic, it really does not do anything. Here only the
compulsory parameters have been inserted. These are:

\begin{itemize}

\item An introductory section |$title| where on three lines the following information is given:

\begin{enumerate}
\item A description of the run. This can be any text that fits on one line.
\item The name of the simulation. This name is used for all files that 
the simulation produces. These files differ from each other only by 
their extension.
\item The name of the basin. This is the basin file without the extension
|.ext|.
\end{enumerate}

\item A section |$para| that contains all necessary parameters for the
simulation to be run. The only compulsory parameters are the ones that
specify the start of the simulation |itanf|, its end |itend| and its 
time step |idt|.

\end{itemize}

In order to be more helpful, some more information must be added to the
|STR| file. As an example let's have a look on \ref{fig:str_example}. Here
we have added two parameters that deal with the type of friction
to be used. |ireib| specifies the bottom friction formulation, here
through a simple quadratic bulk formula. (For the exact meaning of the
parameters, please refer to the section lateron where all parameters
are listed.) The parameter |czdef| specifies the value to use for the
bottom drag coefficient.

The lats parameter in the |$para| section is |dragco| which is the
drag coefficient to use for the wind file specified later. If n

ggugguggu

do with 
	%??

		\subsection{Boundary conditions}
		

In order to have a more meaningfull simulation, we need to specify
boundary conditions. In this section we will deal with the open boundary
conditions, e.g., the conditions at the place where the basin comunicates
with other water bodies.  For lagoons it would be the inlets.

For every boundary condition one section |$bound| must be specified. Since
you can have more than one open boundary you must specify also the number
of your boundary, e.g., |$bound1|, |$bound2| etc. Inside every section
you can then specify the various parameters that characterize your boundary.

Basically there two types of open boundary conditions. Either the water
level or the discharges (fluxes) can be specified. The parameter that
decides the type of boundary is |ibtyp|. A value of one indicates water
levels, instead a value of 2 or 3 indicates fluxes. If you specify
discharges entering at the border of the domain, |ibtyp| = 2 should be
specified. Otherwise, if there are internal sources in the basin then
|ibtyp| = 3 must be used. If you do not define this parameter, a value of 1
will be used and water levels will be specified.

The only compulsary parameter in this section is the list of boundary
nodes.  You do this with the parameter |kbound|. 
In the case of |ibtype| 1 or 2 at least two nodes must be
specified, in order to give an extension of the boundary. The numeration
of the boundary nodes must be consecutive and with the basin on its
left side when going along the boundary nodes.  In the case of |ibtyp|
= 3 even a single point can be given.

The boundary values you want to give are normally specified through 
a a file with a time series. You give the name of the file that contains
the time series with the parameter |boundn|. 
An example with two boundaries can again be found in 
\Fig\figref{example}. Here water levels are prescribed and the values
for the water levels are read from a file |levels1.dat|.

If the values on the boundary
you want to impose can be described through a simple sinus function, you
can also give the bounadry values specifying the parameters for the
sinus function. An example of a water level boundary with a tide of
$\pm 70 cm$ and a period of 12 hours (semi-diurnal) is given in
\Fig\figref{bound}. Note thet |zref| gives the average water level of the
boundary. If you specify |ampli|=0 you get a constant boundary value
of |zref|.

\begin{figure}[ht]
\begin{verbatim}
$bound1
      ibtyp = 1   kbound = 23 25 28
      ampli = 0.70  period = 43200  phase = 10800  zref = 0.
$end
\end{verbatim}
\caption{Example of a boundary with regular sinusoidal water levels.
The pahse of 10800 (3 hours) makes sure that the simulation starts at
slack tide when the basin is completely full.}
\label{fig:bound}
\end{figure}










%		\subsection{Writing output}
%		\todo{Georg}

%		\subsection{Simulating passive tracers}
%		\todo{Georg}

%		\subsection{Temperature and salinity}
%		\todo{Georg}

%		\subsection{Linear and non-linear simulations}
%		\todo{Georg}

%		\subsection{Coriolis force}
%		\todo{Georg}

%		\subsection{Wind forcing}
%		\todo{Georg}

	\section{Advanced usage}

		\subsection{Variable time step}
		
Generally SHYFEM is run with a fixed time step given by the
parameter |idt|.
This choice is acceptable when the model runs in unconditionally
stable conditions (ie. linear simulation, no horizontal viscosity).

The introduction of the advective terms (|ilin=0|) or horizontal
viscosity (|ahpar| greater 0)
can introduce instabilities. To be sure that the model runs in stable 
conditions, it must be assured that the Courant Number is smaller than 1. 
Please note that only in the case of advection we should call
this number the Courant number. However, we will continue to use
the term Courant number for all stability related issues.

In the case of advection the Courant number is defined as
 \begin{equation}
 Cou=\frac{v\Delta t}{\Delta x}
 \end{equation}
where $v$ is the current speed, $\Delta t$ the time step and $\Delta x$
the element size. For finite elements, due to the triangular grid, this
expression is slightly more complicated. As can be seen, lowering the time step
will bring the Courant number below the limit of 1.

To keep the Courant Number under the limit it is necessary to adapt the 
time step at every computation. The variable timestep is computed
introducing in the |STR| file in the |$para| section the parameters
|itsplt|, |coumax| and |idtsyn|.

|coumax| gives the limit of the Courant number. This is normally 1, but since
no exact stability limit can be derived for the non-linear advective terms,
another value can be specified. If instabilities arise, a slightly lower
value than 1 (0.9) can be tried.

|itsplt| decides about the time step splitting.
If this value
               is 0, the time step will be kept constant at its initial
               value. A value of 1 devides the initial time step into
               (possibly) equal parts, but makes sure that at the end
               of the micro time steps one complete macro time
               step has been executed. The last mode |itsplt| = 2
               does not care about the macro time step, but always
               uses the biggest time step possible. In this case
               it is not assured that after some micro time steps
               a macro time step will be recovered. Please note
               that the initial macro time step |idt| will never be exceeded.

Finally, the parameter |idtsyn| is only used in case of |itsplt| = 2. 
This parameter makes sure that
               after a time of |idtsyn| the time step will be syncronized
               to this time. Therefore, setting |idtsyn| = 3600 means
               that there will be a time stamp every hour, even if the model
               has to take one very small time step in order to reach that
               time.

An example of how to set the variable time stepping scheme is shown
in \Fig\figref{vartime}. Here the Courant number is lowered to 0.9 and
the variable time step is syncronized every 3600 seconds (1 hour).

\begin{figure}[ht]
\begin{verbatim}
$para
        coumax = 0.9   itsplt = 2   idtsyn = 3600
$end
\end{verbatim}
\caption{Example of variable time step settings. The time step is syncronized
at every hour, and the Courant number is lowered to 0.9.}
\label{fig:vartime}
\end{figure}



		\subsection{3D computations}
		
The basic way to run the model is in 2D, computing for each element of the
grid one value for the whole water column.  All the variables are computed
in the center of the layer, halfway down the total depth.  Deeper basins
or highly variable bathymetry can require for the correct reproduction
of the velocities, temperature and salinity the need for 3D computation.

The 3D computation is performed on the basis of z layers. In this
representation each layer horizontally has constant depth over the whole
basin, but vertically the layer thickness may vary between different
layers. However, the first layer (surface layer) is of varying thickness
because of the water level variation, and the last layer of an element
might be only partially present due to the bathymetry.

Layers are counted from the the surface layer (layer 1) down to the
maximum layer, depending again on the local depth. Therefore, elements
(and nodes) normally have a different total number of layers from one to
each other. This is opposed to sigma layers where the number of total
layers is constant all over the basin, but the thickness of each layer
varies between different elements.

In order to use layers for 3D computations a new section |$layers|
has to be introduced into the |STR| file, where the sequence of depth
values of the bottom of the layers has to be declared.  Please, make sure
that in the file |Rules.make|, the number of allowed levels |nlvdim| is
greater or equal than the ones actually used in the |STR| file. Layer
depths must be declared in increasing order. An example of a |$layer|
section is given in figure \figref{layers}. Please note that the maximum
depth of the basin in the example must not exceed 20 m.

\begin{figure}[ht]
\begin{verbatim}
$layers
	2 4 6 8 10 13 16 20
$end
\end{verbatim}
\caption{Example of section {\tt \$layers}. The maximum depth of 
the basin is 20 meters. The first 5 layers have constant thickness 
of 2 m, while the last three vary between 3 and 4 m.}
\label{fig:layers}
\end{figure}

A specific treatment for the bottom layer has to be carried out.  In fact,
if the model runs on basins with variable bathymetry, for each element
there will be a different total number of layers. The bathymetric value
normally does not coincide with one of the layer depths, and therefore
the last layer must be treated seperately.

To declare how to treat the last layer two parameters have to be
inserted in the |$para| section. The first is |hlvmin|, the minimum
depth, expressed as a percentage with respect to the full layer depth,
ranging between 0 and 1, This is the fraction that the last layer
must have in order to be maintained as a seperate layer.  The second
parameter is |ilytyp| and it defines the kind of adjustment done on the
last layer. If it is set to 0 no adjustment is done, if it is set to 1
the depth of the last layer is adjusted to the one declared in the |STR|
file (full layer change).  If it is 2 the adjustment to the previous layer
is done only if the fraction of the last layer is smaller than |hlvmin|
(change of depth).  If it is 3 (default) the bathymetric depth is kept
and added to the last but one layer. Therefore with a value of 0 or 3
the total depth will never be changed, whereas with the other levels
the total depth might be adjusted.

As an example, take the layer definition of \Fig\figref{layers}. Let |hlvmin|
be set to 0.5, and let an element have a depth of 6.5 m. The total number
of layers is 4, where the first 3 have each a thickness of 2 m and the
last layer of this element (layer 4) is 0.5 m. However, the nominal
thickness of layer 4 is 2 m and therefore its relative thickness is 0.25
which is smaller than |hlvmin|. With |ilytyp|=0 no adjustment will be
done and the total number of layers in this element will be 4 and the
last layer will have a thickness of 0.5 m.  With |ilytyp|=1 the total
number of layers will be changed to 3 (all of them with 2 m thickness)
and the total depth will be adjusted to 6 m. The same will happen with
|ilytyp|=2, because the relative thickness in layer 4 is smaller than
|hlvmin|.  Finally, with |ilytyp|=3 the total number of layers will be
changed to 3 but the remaining depth of 0.5 m will be added to layer 3
that will become 2.5 m.

In the case the element has a depth of 7.5 m, the relative thickness is
now 0.75 and greater than |hlvmin|.  In this case, with |ilytyp|=0, 2
and 3 no adjustment will be done and the total number of layers in this
element will be 4 and the last layer will have a thickness of 1.5 m.
With |ilytyp|=1 the total number of layers will be kept as 4 but the
total depth will be adjusted to 8 m. This will make all layers equal to
2 m thickness.

The introduction of layers requires also to define the values of
vertical eddy viscosity and eddy diffusivity.  In any case a value of
these two parameters has to be set if the 3D run is performed. This
could be done by setting a constant value of the parameters |vistur|
(vertical viscosity) and |diftur| (vertical diffusivity). In this case
possible values are between 1\ten{-2} and 1\ten{-5}, depending on the
stability of the water column. Higher values (1\ten{-2}) indicate higher
stability and a stronger barotropic behavior.

The other possibility is to compute the vertical eddy coeficients through
a turbulence closure scheme. This usage will be described in the scetion
on turbulence.



		\subsection{Baroclinic terms}
		
The baroclinic pressure gradient term permits to compute
the variation of velocity due to the horizontal
gradients of temperature and/or salinity. These gradients
act on the horizontal variation of density. 

If the variations of temperature and salinity and the
baroclinic pressure gradient term has to be computed, 
the parameter |ibarcl| (in section |$para|)
must be set different from 0.

Setting |ibarcl| to a value different from 0 will simulate the
transport and diffusion of temperature and salinity in the basin. A value of 1
will compute the full baroclinic pressure terms. A value of 2
will do diagnostic simulations. This means that baroclinic pressure terms
are still included in the hydrodynamic equations, 
but temperature and salinity will not be computed
but will be read from a file. Finally for |ibarcl|=3 temperature and salinity
will be computed but no baroclinic pressure term will be used. In this case
the hydrodynamic equations and the equations for temperature and
salinity are decoupled and there is no feed back from the density field
to the currents.

In any case, if temperature and salinity are computed,
first they must be initialized either with
constant values or with variable 3D matrices.
In the first case
the reference values have to be imposed in
|temref| and |salref|. An example of this type of simulation is given
in \Fig\figref{baroc}.

If the temperature and salinity are given as 3D matrices files,
they must be provided in the |$name| section, giving the file 
names in |tempin| and |saltin|. In case of diagnostic simulations the
matrices of temperature and salinity have to be provided in the
files named |tempd| and |saltd| and data must be available for
the whole period of simulation.

\begin{figure}[ht]
\begin{verbatim}
$para
        ibarcl = 1   temref = 18.    salref = 35.
$end
\end{verbatim}
\caption{Example of baroclinic simulation. The initial values for temperature
and salinity are set to 18 C and 35.}
\label{fig:baroc}
\end{figure}



%		\subsection{Restart}
%		\todo{Marco}

		\subsection{Turbulence}
		
In the Reynolds equations turbulent eddy diffusivities and viscosities are
introduced into the equations that must be parameterized and given some
value. Moreover
SHYFEM assumes the hydrostatic approximation. Therefore, there is the need 
to parameterize the non-hydrostatic effects. These are considered
sub-scale processes which are mainly of convective nature.

Vertical eddy viscosities and diffusivities have to be defined 
if there is the intent to model the turbulence effects.
These vertical eddy viscosities and diffusivities can be set to 
constant values, defining |vistur| and |diftur| in the |$para| section.
There is also the opportunity to compute, at each timestep, 
variable values of them, using the turbulence closure module.

The parameter that has to be set in order too choose the 
turbulence scheme is |iturb| in the |$para| section.. 
If |iturb|=0 the vertical eddy viscosity and eddy diffusivity are 
set constant (default 0) and must be defined in |vistur| and |diftur|.

If |iturb|=1 the turbulence closure scheme applied is the $k-\epsilon$ model. 
If |iturb|=2 the GOTM turbulence closure module is used. In this case 
the file |gotmturb.nml| must be provided that sets all necessary parameters. 
This file must be declared in the section |$name| for the item |gotmpa|.

A default |gotmturb.nml| file is provided and it allows the computation of 
the vertical eddy viscosity and eddy diffusivity by means of the 
GOTM  $k-\epsilon$ model.
More information on the GOTM turbulence closure module can be found in
the GOTM Manual \footnote{http://www.gotm.net/index.php?go=documentation}.

If the turbulence module should be used, a value of |iturb|=2 is recommended.
An example of the settings for the turbulence closure scheme is given in
\Fig\figref{turbulence}.

\begin{figure}[ht]
\begin{verbatim}
$para
        iturb = 2
$end
$name
        gotmpa = 'gotmturb.nml'
$end
\end{verbatim}
\caption{Example of turbulence settings. The GOTM module for the
turbulence closure is used. The parameters are contained in file
gotmturb.nml.}
\label{fig:turbulence}
\end{figure}




		\subsection{Sediment transport}
		The sediment transport module calculates sediment transport for currents 
only or combined waves and currents over either cohesive and non-cohesive
sediments. The core of this module is derived by the SEDTRANS05 
sediment transport model, which is here coupled with SHYFEM. The sediment 
transport module computes the erosion and deposition rates at every mesh 
node and determines the sediment volume that is injected into the water 
column. After this step the sediments are advected with the transport and 
diffusion module described above. The module update every time step the 
characteristcs of the bottom in terms of grainsize composition and 
sediment density.

The sediment transport module is activated by setting |isedi| = 1 
and |sedgrs| in the section |sedtr|. 
For more details about the parameters see the Appendix C.

For more information about the sediment transport module please refer 
to Neumeier et al. \cite{urs:sedtrans05} and Ferrarin et al. 
\cite{ferrarin:morpho08}.

\subsubsection{Sediment transport formulation}
For the non-cohesive sediment there are two transport mechanisms, the 
bedload and the suspended transport, while for the cohesive sediment 
is assumed to exist only the suspended transport. Then we use the 
sediment continuity equation for the non-cohesive bedload transport 
and the transport and diffusion equation to describe the transport
of the suspended sediment, both cohesive and non-cohesive.
For the bedload component a direct advection scheme is used. 
Five methods are proposed to predict sediment transport for non-cohesive
sediments. The methods of Brown \cite{brown:engi}, Yalin \cite{yalin:bedload} 
and Van Rijn \cite{vanrijn93:prin} predict the bedload transport. The 
methods of Engelund and Hansen \cite{eh:momo} and Bagnold 
\cite{bagnolds:ma-sed} predict the total load transport.

Different approaches have been used to compute the sediment flux 
between water column and bottom for cohesive and non-cohesive sediment.
For cohesive sediments the model computes erosion and deposition rates,
while for non-cohesive the net sediment flux between bottom and water 
column s computed as the difference between the equilibrium
concentration and the existing concentration in the lower level.
Here the source (erosion occurs, flux from the bottom to the water 
column) term has been taken as explicit, whereas the sink term as 
semi-implicit (deposition occurs, flux from the water column to 
the bottom). This approach permits to avoid negative concentration 
due to deposition higher then the sediment mass present in the water 
column.

The vertical mixing coefficient has been calculated  using analytical
expressions given by \cite{vanrijn93:prin} for both the case of current
related and wave related turbulent mixing.

\subsubsection{Bed representation}
The bed module is designed to have spatially different characteristics, such 
as grainsize composition, sediment density and critical stress for erosion.
The bed is subdivided in several layers and levels. Each layer is considered
homogeneous, well mixed and characterized by its own grainsize distribution
(fraction of each class of sediment considered). On the level are defined
the dry bulk density $\rho_{dry}$ and the critical stress for erosion
$\tau_{ce}$; it is assumed that these variables vary linearly between two
levels. These characteristics could vary spatially in the domain.

At each location the uppermost layer has to be always greater or equal to
the surficial active, or mixed, layer that is available for suspension.
Sediment below the active layer is unavailable resuspension until the 
active layer moves downward either because erosion has occurred, or it 
has thickened due to increase shear stress. As active layer is considered 
the bottom roughness height considered as the sum of the grain roughness, 
the bedload roughness and the bedform (ripple) roughness.

Multiple sand grain size classes are considered to behave independently. At
each location the average grain size (based on the sediment fractions) is
used to compute the bed roughness and critical shear velocities.
Modification of the bed elevation and to the grain size distribution are
updated at each time step based on the net erosion and deposition.
For each size class, the volume of sediment removed from the bed during any
time step is limited by the amount available in the active layer. In this
way the model takes into account time-dependent and spatial sediment
distribution and bed armouring.

\subsubsection{Sand-mud mixture}
The morphological behaviour of estuaries and lagoons often depend on
non-cohesive as well as cohesive sediments. Prediction the distribution
of sediments in these environments, characterized by zonation of sand,
mud and mixed deposits, is of crucial importance for sustainable
management and development of such systems.

Based on laboratory and field experiments several researchers identified a
transition from non-cohesive to cohesive behaviour at increasing mud content
in a sand bed. A sand bed with small amount of mud already shows increased
resistance against erosion. Above a critical mud content (\% $<$ 0.063 mm), 
the bed behaved cohesively. The critical mud content depend on the history 
of the bed and on the geochemical properties of the sand and mud and could 
varies from 3-20 \%. Such a wide range clearly demonstrate that the 
parameters governing the erosion behaviour of sand-mud mixture, are
 not fully understood yet. For this reason is crucial to estimate 
the critical mud or clay content from laboratory and field experiments.

Below the critical mud content the sediments particles are eroded as
non-cohesive sediments. It is assumed that the sediments in suspension
always behave independently, either if flocculation processes of the
thinner particles could trap sand grains into the floc.

Moreover sand increases the binding between the clay particles and results
in a more compact and dense matrix which is more resistant to erosion. For
non-consolidated cohesive bed adding sand to mud increase the erosion
resistance because of the increased density and consolidation rate.
The improved critical stress for erosion due to sand particles is taken
into consideration increasing the dry density with the sand fraction.

\subsubsection{Sediment model output}
The sediment transport model writes the following output:
\begin{itemize}
\item erosion/deposition [m], 2D variable 80 in file SED
\item average grainsize of first bottom layer [m], 2D variable 81 in file SED 
\item bed shear stress [Pa] of first bottom layer, 2D variable 82 in file SED
\item updated node depth [m], 2D variable 83 in file SED
\item total bedload transport [kg/ms], 2D variable 84 in file SED
\item total suspended concentration [kg/m3], 3D variable 85 in file SCO
\end{itemize}

The time step and start time for writing to file SED and SCO
are defined by the parameters |idtcon| and |itmcon| in the |para|
section. These parameter are the same used for writing tracer
concentration, salinity and water temperature. If |idtcon| is not
defined, then the sediment model does not write any results.




		\subsection{Wind waves}
		\input{S_wave}

%		\subsection{Meteo forcing}
%		\todo{Debora}

%		\subsection{Lagrangian module}
%		\todo{Andrea}

%		\subsection{Large grids and projections}
%		\todo{Debora}

%%%%%%%%%%%%%%%%%%%%%%%%%%%%%%%%%%%%%%%%%%%%%%%%%%%%%%%%%%%%%%%%%%%%%%%%
%%%%%%%%%%%%%%%%%%%%%%%%%%%%%%%%%%%%%%%%%%%%%%%%%%%%%%%%%%%%%%%%%%%%%%%%
%%%%%%%%%%%%%%%%%%%%%%%%%%%%%%%%%%%%%%%%%%%%%%%%%%%%%%%%%%%%%%%%%%%%%%%%

%\chapter{Other modules}

%	\section{Residence times}
%	\todo{Andrea}

%	\section{Ecological module (EUTRO)}
%	\todo{Michol (c'e' gia' una descrizione di Donata)}

%	%\documentclass{report}
%\begin{document}
                                                                              
                                                                                                    
                                                                                                    
                                                                                                    
                                                                                                    
                                                                                                    
                                                                                                    
                                                                                                    
                                                                                                    
                                                                                                    
\newcommand{\tab}{\hspace{5mm}}

\newcommand{\Tone}{\ref{MassBalance}}
\newcommand{\Ttwoa}{\ref{FuncDesc}}
\newcommand{\Ttwob}{\ref{Paras}}
\newcommand{\Ttwoc}{\ref{Vars}}

\newcommand{\STone}{\ref{SMassBalance}}
\newcommand{\STtwoa}{\ref{SFuncDesc}}
\newcommand{\STtwob}{\ref{SParas}}
\newcommand{\STtwoc}{\ref{SVars}}




by Donata Melaku Canu, Georg Umgiesser, Cosimo Solidoro

\vspace{1cm}

The coupling between EUTRO and FEM constitute a structure which 
is meant to be a generic water quality for full eutrophication 
dynamics.
The Water Quality model is described fully in
Umgiesser et al. (2003).



\section{General Description}



The water quality model has been derived from the EUTRO module 
of WASP (released by the U.S. Environmental Protection Agency 
(EPA) (Ambrose et al., 1993) and modified. It simulates the evolution 
of nine state variables in the water column and sediment bed, 
including dissolved oxygen (DO), carbonaceous biochemical oxygen 
demand (CBOD), phytoplankton carbon and chlorophyll a (PHY), 
ammonia (NH3), nitrate (NOX), organic nitrogen (ON), organic 
phosphorus (OP), orthophosphate (OPO4) and zooplankton (ZOO). 
The interacting nine state variables can be considered as four 
interacting systems: the carbon cycle, the phosphorous cycle, 
the nitrogen cycle and the dissolved oxygen balance (Fig. ??). 
Different levels of complexity can be selected by switching the 
eight variables on and off, in order to address the specific 
topics.

The evolution of phytoplankton concentration (Reaction 1, 
Table \Tone)
is described by the anabolic and the catabolic terms, plus 
a grazing term related to zooplankton concentration (Reaction 
10, 11 and 12, Table \Ttwoa), which however is treated as a constant
in the original version. 
The anabolic term (Reaction 10, Table \Ttwoa) is related to light 
intensity, temperature and concentration of nutrients in water, 
while the catabolic term (Reaction 11, Table \Ttwoa) depends on temperature.

Phytoplankton growth is described by combining a maximum growth 
rate under optimal conditions, and a number of dimensionless 
factors, each ranging from 0 to 1, and each one referring to 
a specific environmental factor (nutrient, light availability), 
which reduces the phytoplanktonic growth insofar as environmental 
conditions are at sub-optimal levels. Phytoplankton stochiometry 
is fixed at the user-specified ratio, so that no luxury uptake 
mechanisms are considered, and the uptake of nutrients is directly 
linked to the phytoplankton growth, and described by the same 
one-step kinetic law. More specifically, the influence of inorganic 
phosphorous and nitrogen availability on phytoplankton growth/nutrients 
uptake is simulated by means of Michealis-Menten-Monod kinetics 
(Reactions 42 and 43, Table \Ttwoa). Phytoplankton uptakes nitrogen 
both in the forms of ammonia and nitrate, but ammonia is assimilated 
preferentially, as indicated in the ammonia preference relation 
(Reaction 38, Table \Ttwoa). The influence of temperature is given 
by an exponential relation (Reaction 13, Table \Ttwoa), while the 
functional forms for the limitation due to sub-optimal light 
condition can be chosen between three alternative options, namely 
the formulation proposed by Di Toro et al. (1971) and the one 
proposed by Smith (1980) (Di Toro and Smith subroutines, Reaction 
44, Table \Ttwoa) and the Steele formulation (Steele, 1962) that 
can use hourly light input values. The 
choice between different available functional forms (Ditoro, 
Smith, and Steele) is made by setting the index |LGHTSW| equal 
to 1, 2 or 3. The new version is therefore able to simulate diurnal 
variations depending on light intensity, such as night anoxia 
due to phytoplankton respiration during nighttime.

Finally, the two frequently used models for combining maximum 
growth and limiting factors, the multiplicative and the minimum 
(or Liebig's) model, are both implemented, and the user can choose 
which one to adopt (Reaction 41, Table \Ttwoa).

Nitrogen and phosphorous are then returned to the organic compartment 
(ON, OP) via phytoplankton and zooplankton respiration and death. 
After mineralization, the organic form is again converted into
the dissolved inorganic form available for phytoplankton growth. 

The DO mass balance is influenced by almost all of the processes 
going on in the system. The reaeration process acts to restore 
the thermodynamic equilibrium level, the saturation value, while 
respirations activities and mineralization of particulated and 
dissolved organic matter consume DO and, of course, photosynthetic 
activity produces it. Other terms included in the DO mass balance 
are the ones referring to redox reactions such as nitrification 
and denitrification. The reaeration rate is computed from the 
model in agreement with either the flow-induced rate or the wind-induced 
rate, whichever is larger. The wind-induced reaeration rate is 
determined as a function of wind speed, water and air temperature, 
in agreement with O'Connor (1983), while the flow-induced reaeration 
is based on the Covar method (Covar, 1976), i.e., it is calculated 
as a function of current velocity, depth and temperature.

The dynamic of a generic herbivorous zooplankton compartment 
(ZOO), meant to be representative of the pool of all the herbivorous 
zooplankton species, is followed and accordingly the subroutines 
relative to phytoplankton, organic matter, nutrients, and dissolved 
oxygen, which were influenced by such a modification. 

The grazing has been described by means of a type II functional 
relationship, as it is usually done for aquatic ecosystems. However, 
the possibility to select a type III relationship, as well as 
to maintain the original parameterisation of constant zooplankton, 
has been included.  

The zooplankton assimilates the ingested phytoplankton with an 
efficiency EFF, and the fraction not assimilated, ecologically 
representative of faecal pellets and sloppy feeding, is transferred 
to the organic matter compartments (dotted lines Fig. ??). Finally, 
zooplankton mortality is described by a first order kinetics. 
The code has been written by adopting the standard WASP nomenclature 
system, and the choice between the different available functional 
forms is performed by setting the index |IGRAZ|. A choice of 0 
(the default value) corresponds to the original EUTRO version, 
giving the user the ability to chose easily between the extended 
version or revert to the original one.


%\documentclass{report}
%\usepackage{a4}
%\usepackage{shortvrb}
%\begin{document}



\newcommand{\Otwo}{O${}_{2}$}
\newcommand{\Degree}{${}^{o}$}
\newcommand{\power}[1]{${}^{#1}$}

\newcommand{\opn}{ {} }


\newcommand{\VSPGB}{\vspace{0.2cm}}
\newcommand{\HSP}{\hspace*{1.5cm}}
\newcommand{\HHSP}{\hspace*{0.5cm}}
\newcommand{\GBox}[2]{\parbox{#1 cm}{\VSPGB#2\VSPGB}}
\newcommand{\GDBox}[1]{\GBox{5}{#1}}
\newcommand{\GEBox}[1]{\GBox{7}{#1}}
\newcommand{\GFBox}[1]{\GBox{7}{#1}}



\begin{table}\centering
\begin{tabular}{lll}
\hline


& & \\
$\frac{\partial S}{\partial t} =Q(S)$
& &
General Reactor Equation
\\
& & \\

$Q(PHY) = GPP - DPP - GRZ$
& 1 & 
Phytoplankton PHY [mg C/L]
\\

$Q(ZOO) = GZ - DZ$
& 2 &
Zooplankton ZOO [mg C/L]
\\

$Q(NH3) = N_{alg1} + ON1 - N_{alg2} - N1$
& 3 &
Ammonia NH3 [mg N/L]
\\

$Q(NOX) = N1 - NO_{alg} - NIT1$
& 4 &
Nitrate NOX [mg N/L]
\\

$Q(ON) = ON_{alg} - ON1$
& 5 &
Organic Nitrogen ON [mg N/L]
\\

$Q(OPO4) = OP_{alg1} + OP1 - OP_{alg2}$ 
& 6 &
\GBox{5}{
Inorganic Phosphorous OPO4 \\
\HHSP [mg P/L]
}
\\

$Q(OP) = OP_{alg3} - OP1$
& 7 &
Organic Phosphorous OP [mg P/L]
\\

$Q(CBOD) = C1 - OX - NIT2$
& 8 &
\GBox{5}{
Carbonaceous Biological Oxygen \\
\HHSP Demand CBOD [mg \Otwo/L]
}
\\

\GBox{6}{
$Q(DO) = DO1 + DO2 + DO3 $\\
\HSP $\opn - DO4 - N2 - OX - SOD$
}
& 9 &
Dissolved Oxygen DO [mg \Otwo/L]
\\


\hline
\end{tabular}
\caption{Mass balances}
\label{MassBalance}
\end{table}






\begin{table}\centering
\begin{tabular}{lll}
\hline


$GPP=GP1*PHY $
& 10 &
phytoplankton growth
\\

$DPP=DP1*PHY $
& 11 &
phytoplankton death 
\\

$GRZ=KGRZ*\frac{PHY}{PHY+KPZ}*ZOO$
& 12 &
grazing rate coefficient
\\

$GP1=L_{nut} *L_{light} *K1C*K1T^{(T-T_{0} )} $
& 13 &
\GDBox{
phytoplankton growth rate with nutrient and light limitation
}
\\

$DP1=RES+K1D$ 
& 14 &
\GDBox{
phytoplankton respiration and death rate
}
\\

$GZ=EFF*GRZ$ 
& 15 &
zooplankton growth rate
\\

$DZ=KDZ*ZOO$ 
& 16 &
zooplankton death rate
\\

$Z_{ineff} = (1-EFF)*GRZ $
& 17 &
grazing inefficiency on phytoplankton
\\

$Z_{sink} = Z_{ineff}+DZ $
& 18 &
sink of zooplankton
\\

$N_{alg1}= NC*DPP*(1-FON) $
& 19 &
source of ammonia from algal 
death
\\

$N_{alg2}=PN*NC*GPP $
& 20 &
sink of ammonia for algal growth
\\

$NO_{alg}= (1. - PN)*NC*GPP $
& 21 &
sink of nitrate for algal growth
\\

$ON_{alg}= NC*(DPP*FON+Z_{sink}) $
& 22 &
\GDBox{
source of organic nitrogen from phytoplankton and zooplankton death
}
\\

\GBox{5}{
$N1=KC_{nit} *KT_{nit}^{(T-T_{0})} *NH3$
\HSP $\opn *\frac{DO}{K_{nit}+DO}$
}
& 23 &
nitrification
\\

\GBox{5}{
$NIT1=KC_{denit} KT_{denit}^{(T-T_{0})}$
\HSP $\opn *NOX*\frac{K_{denit}}{K_{denit}+DO}$
}
& 24 &
denitrification
\\

$ON1=KNC_{\min } *KNT_{\min }^{(T-T_{0})} *ON$
& 25 &
mineralization of ON
\\

$OP1=KPC_{\min } *KPT_{\min } ^{(T-T_{0})} *OP$
& 26 &
mineralization of OP
\\

$OP_{alg1}=PC*DPP*(1. - FOP) $
& 27 &
\GDBox{
source of inorganic phosphorous from algal death
}
\\

$OP_{alg2}=PC*GPP $
& 28 &
\GDBox{
sink of inorganic phosphorous for algal growth
}
\\

$OP_{alg3}=PC*(DPP*FOP+Z_{sink}) $
& 29 &
\GDBox{
source of organic phosphorous from phytoplankton and zooplankton death
}
\\

\GBox{5}{
$OX=KDC*KDT^{(T-T_{0} )} $
\HSP $\opn *CBOD*\frac{DO}{KBOD+DO} $
}
& 30 &
oxidation of CBOD
\\

\hline
\end{tabular}
\caption{Functional Expression Description}
\label{FuncDesc}
\end{table}

\begin{table}\centering
\begin{tabular}{lll}
\hline

$C1=OC*(K1D*PHY+Z_{sink}) $
& 31 &
\GDBox{
source of CBOD from phytoplankton and zooplankton death
}
\\

$NIT2=\left( \frac{5}{4} *\frac{32}{14} *NIT1\right) $
& 32 &
sink of CBOD due to denitrification
\\

$DO1=KA*(O_{sat} - DO) $
& 33 &
reareation term
\\

$DO2=PN*GP1*PHY*OC $
& 34 &
\GDBox{
dissolved oxygen produced by phytoplankton using NH3
}
\\

\GBox{5}{
$DO3=(1-PN)*GP1*PHY$
\HSP $\opn * 32*\left( \frac{1}{12} +1.5*\frac{NC}{14} \right) $
}
& 35 &
growth of phytoplankton using NOX
\\

$DO4 = OC*RES*PHY$ 
& 36 &
respiration term
\\

$N2=\left( \frac{64}{14} *N1\right) $
& 37 &
oxygen consumption due to nitrification
\\

\GBox{6}{
$PN=\frac{NH3*NOX}{(KN+NH3)*(KN+NOX)}$ 
\HSP $\opn +\frac{NH3*KN}{(NH3+NOX)*(KN+NOX)} $
}
& 38 &
ammonia preference
\\

$RES=K1RC*K1RT^{(T-T_{0})} $
& 39 &
algal respiration
\\

$SOD=\frac{SOD1}{H} *SODT^{(T-T_{0})} $
& 40 &
sediment oxygen demand
\\

$L_{nut}= min(X1,X2) \, , \,  mult(X1,X2) $
& 41 &
\GDBox{
minimum or multiplicative nutrient limitation for phytoplankton growth
}
\\

$X1=\frac{NH3+NOX}{KN+NH3+NOX} $
& 42 &
\GDBox{
nitrogen limitation for phytoplankton growth
}
\\

$X2=\frac{OPO4}{\frac{KP}{FOPO4} +OPO4} $
& 43 &
\GDBox{
phosphorous limitation for phytoplankton growth
}
\\

$L_{light} =\frac{I_{0} }{I_{s} } *e^{-(KE*H)} *e^{(1-\frac{I_{0} }{I_{s}
} *e^{(-KE*H)} )} $
& 44 &
\GDBox{
light limitation for phytoplankton growth
}
\\

$KA=F(Wind,Vel,T,T_{air},H) $
& 45 &
re-areation coefficient 
\\


\hline
\end{tabular}
\addtocounter{table}{-1}
\caption{(continued) Functional Expression Description}
\label{FuncDesc1}
\end{table}













\begin{table}\centering
\begin{tabular}{ll}
\hline


$K1D=0.12$ day\power{-1}  
&
phytoplankton death rate constant
\\

$KGRZ=1.2$ day\power{-1}  
&
grazing rate constant
\\

$KPZ=0.5$ mg C/L 
&
\GFBox{
half saturation constant for phytoplankton in grazing
}
\\

$KDZ=0.168$ day\power{-1} 
&
zooplankton death rate
\\

$K1C=2.88$ day\power{-1} 
&
phytoplankton growth rate constant
\\

$K1T=1.068$ 
&
\GFBox{
phytoplankton growth rate temperature constant
}
\\

$KN=0.05$ mg N/L 
&
\GFBox{
nitrogen half saturation constant for phytoplankton growth
}
\\

$KP=0.01$ mg P/L 
&
\GFBox{
phosphorous half saturation constant for phytoplankton growth
}
\\

$KC_{nit}=0.05$ day\power{-1} 
&
nitrification rate constant
\\

$KT_{nit}=1.08$ 
&
nitrification rate temperature constant
\\

$K_{nit}=2.0$ mg \Otwo/L 
&
half saturation constant for nitrification
\\

$KC_{denit}=0.09$ day\power{-1} 
&
denitrification rate constant
\\

$KT_{denit}=1.045$ 
&
denitrification rate temperature constant
\\

$K_{denit}=0.1$ mg \Otwo/L 
&
half saturation constant for denitrification
\\

$KNC_{min}=0.075$ day\power{-1} 
&
mineralization of dissolved ON rate constant
\\

$KNT_{min}=1.08$ 
&
\GFBox{
mineralization of dissolved ON rate temperature constant
}
\\

$KDC=0.18$ day\power{-1}
&
oxidation of CBOD rate constant
\\

$KDT= 1.047$ 
&
oxidation of CBOD rate temperature constant
\\

$NC=0.115$ mg N/mg C 
&
N/C ratio
\\

$PC=0.025$ mg P/mg 
&
C P/C ratio
\\

$OC=32/12$ mg \Otwo/mg C 
&
O/C ratio
\\

$EFF=0.5$ 
&
grazing efficiency
\\

$FON=0.5$ 
&
fraction of ON from algal death
\\

$FOP=0.5$ 
&
fraction of OP from algal death
\\

$FOPO4=0.9$ 
&
fraction of dissolved inorganic phosphorous
\\

$KPC_{min}=0.0004$ day\power{-1}  
&
mineralization of dissolved OP rate constant
\\

$KPT_{min}=1.08$ 
&
\GFBox{
mineralization of dissolved OP rate temperature constant
}
\\

$KBOD=0.5$ mg \Otwo/L
&
CBOD half saturation constant for oxidation
\\

$K1RC=0.096$ day\power{-1}  
&
algal respiration rate constant
\\

$K1RT=1.068$ 
&
algal respiration rate temperature constant
\\

$I_{s}=1200000$ lux/day
&
\GFBox{
optimal value of light intensity for phytoplankton growth
}
\\

$KE=1.0$ m\power{-1}
&
light extinction coefficient
\\

\GBox{5}{
$SOD1=2.0$ mg \Otwo/L \\
\HSP day\power{-1} m 
}
&
sediment oxygen demand rate constant
\\

$SODT=1.08$ 
&
sediment oxygen demand temperature constant
\\

$T_{0}=20$ \Degree C 
&
optimal temperature value
\\


\hline
\end{tabular}
\caption{Parameters}
\label{Paras}
\end{table}





\begin{table}\centering
\begin{tabular}{lll}
\hline


$T$ 
& [\Degree C] &
water temperature
\\

$T_{air}$ 
& [\Degree C] &
air temperature
\\

$O_{sat}$  
& [mg/L] &
DO concentration value at saturation
\\

$I_{0}$ 
& [lux/day] &
incident light intensity at the surface
\\

$H$ 
& [m] &
depth
\\

$Vol$ 
& [m\power{3}] &
volume
\\

$Vel$ 
& [m/sec] &
current speed
\\

$Wind$ 
& [m/sec] &
wind speed
\\


\hline
\end{tabular}
\caption{Variables}
\label{Vars}
\end{table}




%\end{document}



\section{The coupling}


Mathematical models usually describe the coupling between ecological 
and physical process by suitable implementation of an advection/diffusion 
equation for a generic tracer, reads 

\begin{equation} \label{AdvDif}
\frac{\partial \,\Theta _{i} }{\partial \,t} \,\,+U\cdot \nabla \,\Theta
_{i} -\,w^{s}_{i} \,\frac{\partial \,\Theta _{i} }{\partial \,z} \,=\,\,K_{h}
\,\nabla _{H}^{2} \Theta _{i} \,+\,\frac{\partial \,}{\partial \,z}
\,\left[ K_{v} \,\frac{\partial \,\Theta _{i} }{\partial \,z} \right]
\,+\,F\,\left( \Theta \,,\,T,\,I\,,\,\,..\right)
\end{equation}

where $U$ is the (average components of the) velocity, 
the $\Theta_{i}$ are the tracers which compose the entire 
vector of the biological state variable $\Theta$ and 
$F$ is a source term. $T$ and $I$ indicate, respectively, 
water temperature and Irradiance level, while $w^{s}_{i}$ represent 
the downward flux rates (sinking velocity) for the tracer 
$\Theta_{i}$, 
and $K_{h}$ and $K_{v}$ are the eddy coefficients for 
horizontal and vertical turbulent diffusion.

The term $F$ includes the contributions of the biological/biogeochemical 
activities, and the whole biological state vector $\Theta$ 
is explicitly considered in the last term of equation \ref{AdvDif}, without 
a spatial operator. As far as the biologically induced variations 
are regarded, the fate of each tracer in every location $x,y,z$ 
is tightly coupled to other tracers in the same location, but 
is not directly influenced by processes going on elsewhere. 

Therefore, in this approximation the global temporal variation 
of any tracer (state variable, conservative or not) can be split 
into the sum of two independent contributions: 

\begin{equation}
\frac{\partial \,\Theta _{i} }{\partial \,t} \,=\,\,\left. \frac{\partial
\,\Theta _{i} }{\partial \,t} \right\arrowvert_{phys} +\left. \frac{\partial
\,\Theta _{i} }{\partial \,t} \right\arrowvert_{biol}
\end{equation}

and it might be convenient, in writing a computer code, to devote 
independent modules to computation of each of them. Indeed, most 
of the modern water quality programs do have, at least conceptually, 
a modular structure. In this way the same code can be used for 
simulating different situations: by switching off the module 
referring to the reactor term the transport of a purely passive 
tracer is reproduced, while a 0D, close and uniformly stirred 
biological system is simulated if the module referring to the 
physical term is not included. Finally, the inclusion of both 
modules gives the evolution of tracers subjected to both physical 
and biogeochemical transformation, in a representation that, 
depending upon the parameterisation of the physical module, can 
be 1, 2 or 3 dimensional.

The whole water quality module is contained in a file
|weutro.f| and the call to EUTRO is made through a subroutine call
that is done from the main program through an appropriate 
interface. There is a clean division between the hydrodynamic 
motor, parameters used by the model and the resolution of the 
differential equations and the ecological model as evidenced 
by the overall structure of the modules. 

It is the responsibility of the main module to implement the 
time loop administration, the advective and diffusive transport 
of the state variables, both in the horizontal and vertical direction 
and the application of the boundary conditions. 

The typical use of the new EUTRO module is as follows: the main 
program first sets all parameters needed in EUTRO through the 
call to |EUTRO_INI|. These parameters are the kinetic constants 
of the reactions that are described in EUTRO and are considered 
constant for a site. They have to be set therefore only once 
at the beginning of the simulation. Once set, these parameters 
are available to the EUTRO module as global parameters.

For every box in the discretized domain (horizontal and vertical), 
and for every time step, the main program calls the subroutine 
|EUTRO0D|. Inside |EUTRO0D| the differential equations that describe 
the bio-chemical reactions are solved with a simple Euler scheme.

The values passed into |EUTRO0D| can be roughly divided into 4 
groups. The first group is made out of the aforementioned constants 
that represent the kinetic constants and other parameters that 
do not vary in time and space. The second group represents the 
state variables that are actually modified by |EUTRO0D| through 
the bio-chemical reactions. These variables are transported and 
diffused by the main routine and are just passed into |EUTRO0D| 
for the description of the processes. After the call no memory 
remains in |EUTRO0D| of these state variables. They must therefore 
be stored away by the main routine to be used in the next time 
step again. The third and fourth groups of values have to do 
with the forcing terms. They have been divided in order to account 
for the different nature of the forcing terms. The third group 
consists of the hydrodynamic forcing terms that are directly 
computed by the hydrodynamic model and parameters related to 
the box discretization. They consist of water temperature, salinity, 
current velocity, and depth and type of the box. Here the type 
identifies the position of the box (surface, water column, sediment), 
which is needed for some of the forcings to be applied. These 
variables are passed directly into EUTRO through a parameter 
list. The last group contains other forcing terms that are not 
directly related to the hydrodynamic model. These consist of 
the meteorological forcings (wind speed, air temperature, ice 
cover), light climate (surface light intensity, day length) and 
sediment fluxes. These parameters are set through a number of 
commodity functions that are called by the main routine. The 
reason why the last two parameter groups are handled differently 
from each other has also to do with the fact that the third group 
is highly variable in time and space. Variables like current 
velocity change with every time step and are normally different 
from element to element. The fourth group is very often only 
slowly variable in time (light, wind) and can very often be set 
constant in space. Therefore these values can be set at larger 
intervals, and do not have to be changed when looping over all 
the elements in the domain.

The overall flow of information during one time step is the following: 
First the hydrodynamic model resolves the momentum and continuity 
equation to update the current velocities and water levels. After 
that the physical (temperature and salinity) and bio-chemical 
scalars are advected and diffused. Once this advection step has 
been handled the new loadings and forcing terms are set-up and 
then |EUTRO0D| is called for the bio-chemical reactions. 


Note that the operator splitting technique, which decouples the 
advective and diffusive transport from the source term, allows 
for different time steps of the two processes for a more efficient 
use of the computer resources. 


Default values of the water quality parameters are already set 
in the code. Owner specific parameters for the water quality 
model should be written in the subroutine |param_user|.





\section{Light limitation}

\input steele.tex



\section{Initialization}



This section describes
the interpolation of data for the initialisation of the model.

To create a file with initial conditions the program
|laplap| can be used. The program is called as
|laplap < namefile.dat|.
This makes a laplacian interpolation of specified data contained in 
the |namefile.dat|.
This data file should have the first line empty and shold contain 
two colums containing, respectively, node number and data values for
the node.

It generates two files, |laplap.nos| and |laplap.dat|. The first 
one can be used to check if the procedure has been conducted 
well, creating a map with the plotting procedure (see Postprocessing 
section). The |.dat| file name should be given in the section
|$name| 
of the |.str| file to initialize the model.

You can create initialisation files for temperature, salinity, 
wind field and biological variables.
If you want to initialise the biological model with biological 
data you should create a single data file merging the 9 data 
files (one for each variable) using the |inputmerge.f| routine.




\newcommand{\listroutine}[1]{\paragraph{\texttt{#1}}}
\newcommand{\lt}{$<$}
\newcommand{\gt}{$>$}



\section{Post processing}

This section shows how to generate derivate variables.

The post processing routines elaborate the water quality outputs 
to generate derivate variables. They allow to generate variables 
such as averages, (both, in time and in space), sum differences, 
and water quality variables such us Vismara, TRIX and BOD5.

The routines and their usage are the following:

\listroutine{nosmaniav.f}
It generates a file containing, for each node of the spatial 
domani, average, minimum and maximum values of the specified 
variable of the whole simulation.

\listroutine{nosmaniqual.f}
It generates a water quality file from the elaboration of the 
state variable.
It computes, for each node, and at each time step a water quality 
index that can be chosen between two suggested indexes: Vismara 
QualityV and TRIX a well known quality index applied to the 
water quality definition at coastal seas and estuaries.

These indices can be computed using the definitions
in Table \ref{ClassWQI} and
the following equations:

\begin{center}
\begin{verb}
QualityV = class(O2sat) + class(BOD5) + class(NH3)
\\
auxt1 = (phyto/30.)*1000\\
auxt2 = (nh3+nox)*1000\\
auxt3 = (opo4+op)*1000\\
TRIX = log10(auxt1*o2satp*auxt2*auxt3+1.5)/1.2
\end{verb}
\end{center}


\begin{table}\centering
\begin{tabular}{lccccc}
\hline

Class(Var) & 1 & 2 & 3 & 4 & 5\\
O2sat & 90-110 & 70-90 or 110-120 & 50-70 or 120-130 & 30-50 & \lt30 or \gt130\\
BOD5 & \lt3 & 3-6 & 6-9 & 9-15 & \gt15\\
NH3 & \lt0.4 & 0.4-1 & 1-2 & 2-5 & \lt5\\

\hline
\end{tabular}
\caption{Classification of Water Quality Indices}
\label{ClassWQI}
\end{table}




\listroutine{nosmanintot.f}
generates a file of total inorganic Nitrogen as sum of NH3 and 
NOx

\listroutine{nosdif.f}
computer for the chosen state variable, the difference between 
the values at two times step. 


\listroutine{nosdiff.f}
computer the difference between the variable outputs of two simulations 



\listroutine{nosmanibod5.f}
computes the BOD5 values from the CBOD outputs as:

\begin{center}
\begin{verb}
bod5 = cbod*(1. - exp( -5. * par1 ))
\\
        + (64./14.) * nh3 * (1. - exp( -5. * par2 ))
\end{verb}
\end{center}

To run one of the postprocessing routine write the name of the 
routine and enter.




\section{The Sediment Module}

The sediment buffer action on the biogeochemical cycles could 
be very important, especially in the shallow water basins and 
during the storm surge events.

The routine |wsedim| (introduced in April 2004) aims to address 
the resuspention/sinking dynamics of nitrogen and phosphorous.
This routine can be switched on and off as needed by the user, 
setting the |bsedim| parameter |true| or |false| in the |bio3d| 
routine. It is called after the the eutrophication subroutine.

It allows to follow the dynamics of two additional variables, 
OPsed and ONsed that simulate the evolution of Nitrogen and Phosphorous 
detritus in sediment that are not subjected to advection-diffusion 
processes.

These two variables interact with the Nitrogen and Phosphorous 
cycle as decribed by the equations in Table \STone. When 
the |wsedim| subroutine is switched on, OP, ON, NH3 and OPO4 are 
updated at each time step in agreement with those equations.

The resuspension is a linear function of the water velocity calculated 
by the hydrodynamic model at each box, as written in Table \STtwoa. 
The amount of the sinking nutrients depends on specific prossess 
parameters, as given in Table \STtwoa, and on the depth of the underlying 
column.


%\documentclass{report}
%\usepackage{a4}
%\usepackage{shortvrb}
%\begin{document}
%
%\newcommand{\Degree}{${}^{o}$}
%\newcommand{\power}[1]{${}^{#1}$}
%
%\newcommand{\VSPGB}{\vspace{0.2cm}}
%\newcommand{\HSP}{\hspace*{1.5cm}}
%\newcommand{\HHSP}{\hspace*{0.5cm}}
%\newcommand{\GBox}[2]{\parbox{#1 cm}{\VSPGB #2 \VSPGB}}
%\newcommand{\GDBox}[1]{\GBox{5}{#1}}
%\newcommand{\GEBox}[1]{\GBox{7}{#1}}
%\newcommand{\GFBox}[1]{\GBox{7}{#1}}
\newcommand{\GHBox}[1]{\GBox{5.2}{#1}}
%
%\newcommand{\opn}[1]{\, #1 \,}
%\newcommand{\opn}{ {} }
%\newcommand{\opn}[1]{#1}



\begin{table}\centering
\begin{tabular}{lll}
\hline


& & \\
$\frac{\partial S}{\partial t} =Q(S)_{sed}$
& &
General Reactor Equation
\\
& & \\

$Q(NH3)_{sed}= NH3_{res}$
& 3 &
Ammonia NH3 [mg N/L]
\\

$Q(ON)_{sed} = (ON_{res} - ON_{sink})$
& 5 &
Organic Nitrogen ON [mg N/L]
\\

$Q(OPO4)_{sed}= OPO4_{res}$
& 6 &
\GBox{5}{
Inorganic Phosphorous \\
\HHSP OPO4 [mg P/L]
}
\\

$Q(OP)_{sed}= (OP_{res}-OP_{sink})$
& 7 &
Organic Phosphorous OP [mg P/L]
\\

\GBox{5}{
$Q(ON_{sed})= ON_{sink} - ON_{res} $
\HSP $\opn - NH3_{res}$
}
& 1sed &
\GBox{5}{
Sediment Organic Nitrogen \\
\HHSP ON${}_{sed}$ [mg N/L]
}
\\

\GBox{5}{
$Q(OP_{sed})= OP_{sink}- OP_{res}$
\HSP $\opn - OPO4_{res}$
}
& 2sed &
\GBox{5}{
Sediment Organic Phosphorous \\
\HHSP OP${}_{sed}$ [mg N/L]
}
\\


\hline
\end{tabular}
\caption{Sediment Mass balances}
\label{SMassBalance}
\end{table}




\begin{table}\centering
\begin{tabular}{lll}
\hline

\GBox{5}{
$NH3_{res}= Vol_{sed}*KNC_{sed}$
\HSP $\opn *KNT^{(T-T_0)}* ON_{sed}$
}
& 1 &
\GHBox{
Mineralization of organic nitrogen in sediment
}
\\

$ON_{res} = Vol_{sed}*KN_{res}*F_{vel}*ON_{sed}$
& 2 &
\GHBox{
Resuspention of organic nitrogen in sediment
}
\\

\GBox{5}{
$ON_{sink}= Vol*\frac{(1-exp(-dt/\tau_N))}{dt}$
\HSP $\opn *ON*FPON$
}
& 3 &
\GHBox{
Sink of organic nitrogen from the water column
}
\\

\GBox{5}{
$OPO4_{res}= Vol_{sed}*KPC_{sed}$
\HSP $\opn *KPT^{(T-T_0)}*OP_{sed}$
}
& 4 &
\GHBox{
Mineralization of organic phosphorous in sediment
}
\\

$OP_{res}= Vol_{sed} * KP_{res} * F_{vel} * OP_{sed}$
& 5 &
\GHBox{
Resuspention of organic phosphorous in sediment
}
\\

\GBox{5}{
$OP_{sink}= Vol*\frac{(1-exp(-dt/\tau_P))}{dt}$
\HSP $\opn *OP*FPOP$
}
& 6 &
\GHBox{
Sink of organic phosphorous from the water column
}
\\

$\tau_N = H/w_s$
& 7 &
\GHBox{
Time scale for sinking processes of organic N
}
\\

$\tau_P = H/w_s$
& 8 &
\GHBox{
Time scale for sinking processes of organic P
}
\\

\hline
\end{tabular}
\caption{Sediment functional expressions}
\label{SFuncDesc}
\end{table}







\begin{table}\centering
\begin{tabular}{lll}
\hline

$KNC_{sed} = 0.075$
&
Mineralization of sediment ON rate constant
\\

$KNT = 1.08$
&
Mineralization of sediment ON rate temperature constant
\\

$KPC_{sed} = 0.22$
&
Mineralization of sediment OP rate constant
\\

$KPT = 1.08$
&
Mineralization of sediment OP rate temperature constant
\\

$KN_{res} = 0.1$
&
Fraction of sediment depth resuspended/day
\\

$KP_{res} = 0.1$
&
Fraction of sediment depth resuspended/day
\\

$FPON = 0.5$
&
Fraction of particulate organic N
\\

$FPOP = 0.5$
&
Fraction of particulate organic P
\\

$F_{vel} = 1$
&
Velocity coefficient
\\

$w_s = 10$ m/day
&
Sinking velocity
\\

$T_0 = 20$ \Degree C
&
Optimal temperature value
\\

\hline
\end{tabular}
\caption{Sediment parameters}
\label{SParas}
\end{table}




\begin{table}\centering
\begin{tabular}{lll}
\hline

$Vol$
& [m\power{3}] &
volume
\\

$Vol_{sed}$
& [m\power{3}] &
sediment volume
\\

$H$
& [m] &
total depth of water column
\\

$dt$
& [sec] &
time step
\\

\hline
\end{tabular}
\caption{Sediment variables}
\label{SVars}
\end{table}







%\end{document}



%\end{document}

%	\subsection{Parameters for the Water Quality Module}
%	\input{S_biopar_h.tex}

%%%%%%%%%%%%%%%%%%%%%%%%%%%%%%%%%%%%%%%%%%%%%%%%%%%%%%%%%%%%%%%%%%%%%%%%
%%%%%%%%%%%%%%%%%%%%%%%%%%%%%%%%%%%%%%%%%%%%%%%%%%%%%%%%%%%%%%%%%%%%%%%%
%%%%%%%%%%%%%%%%%%%%%%%%%%%%%%%%%%%%%%%%%%%%%%%%%%%%%%%%%%%%%%%%%%%%%%%%

%\chapter{Postprocessing}
%
% Da qui in poi ci pensiamo piu' tardi...
%
%\section{Running the post processing routines}
%\section{Time series: gnuplot}
%\section{Plotting spatial data: plots}
%	\subsection{Basic usage}
%	\subsection{Advanced usage}
%\section{Exporting data}

%%%%%%%%%%%%%%%%%%%%%%%%%%%%%%%%%%%%%%%%%%%%%%%%%%%%%%%%%%%%%%%%%%%%%%%%
%%%%%%%%%%%%%%%%%%%%%%%%%%%%%%%%%%%%%%%%%%%%%%%%%%%%%%%%%%%%%%%%%%%%%%%%
%%%%%%%%%%%%%%%%%%%%%%%%%%%%%%%%%%%%%%%%%%%%%%%%%%%%%%%%%%%%%%%%%%%%%%%%

%\chapter{Final thoughts}

%%%%%%%%%%%%%%%%%%%%%%%%%%%%%%%%%%%%%%%%%%%%%%%%%%%%%%%%%%%%%%%%%%%%%%%%
%%%%%%%%%%%%%%%%%%%%%%%%%%%%%%%%%%%%%%%%%%%%%%%%%%%%%%%%%%%%%%%%%%%%%%%%
%%%%%%%%%%%%%%%%%%%%%%%%%%%%%%%%%%%%%%%%%%%%%%%%%%%%%%%%%%%%%%%%%%%%%%%%
%%%%%%%%%%%%%%%%%%%%%%%%%%%%%%%%%%%%%%%%%%%%%%%%%%%%%%%%%%%%%%%%%%%%%%%%
%%%%%%%%%%%%%%%%%%%%%%%%%%%%%%%%%%%%%%%%%%%%%%%%%%%%%%%%%%%%%%%%%%%%%%%%



\appendix


\chapter{Hydrodynamic equations and resolution techniques}

	
%%%%%%%%%%%%%%%%%%%%%%%%%%%%%%%%%%%%%%%%%%%%%%%%%%%%%%%%%%
%%%%%%% user commands %%%%%%%%%%%%%%%%%%%%%%%%%%%%%%%%%%%%
%%%%%%%%%%%%%%%%%%%%%%%%%%%%%%%%%%%%%%%%%%%%%%%%%%%%%%%%%%

\newcommand{\paren}[1]	{ \left( #1 \right) }
\newcommand{\mez}{\mbox{$\frac{1}{2}$}}
\newcommand{\dpp}{\mbox{$\partial$}}
\newcommand{\zz}{\zeta}
\newcommand{\un}{\mbox{$U^{n+1}$}}
\newcommand{\uo}{\mbox{$U^{n}$}}
\newcommand{\vn}{\mbox{$V^{n+1}$}}
\newcommand{\vo}{\mbox{$V^{n}$}}
\newcommand{\zn}{\mbox{$\zz^{n+1}$}}
\newcommand{\zo}{\mbox{$\zz^{n}$}}
\newcommand{\up}{\mbox{$U^{\prime}$}}
\newcommand{\vp}{\mbox{$V^{\prime}$}}
\newcommand{\zp}{\mbox{$\zz^{\prime}$}}
\newcommand{\dzxp}{\mez \frac{\dpp (\zn + \zo)}{\dpp x}}
\newcommand{\dzyp}{\mez \frac{\dpp (\zn + \zo)}{\dpp y}}
\newcommand{\dzxn}{\mbox{$\frac{\dpp \zn}{\dpp x}$}}
\newcommand{\dzyn}{\mbox{$\frac{\dpp \zn}{\dpp y}$}}
\newcommand{\dzxo}{\mbox{$\frac{\dpp \zo}{\dpp x}$}}
\newcommand{\dzyo}{\mbox{$\frac{\dpp \zo}{\dpp y}$}}

\newcommand{\tdif}[1] {\frac{\partial #1}{\partial t}}
\newcommand{\xdif}[1] {\frac{\partial #1}{\partial x}}
\newcommand{\ydif}[1] {\frac{\partial #1}{\partial y}}
\newcommand{\zdif}[1] {\frac{\partial #1}{\partial z}}
\newcommand{\dt} {\mbox{$\Delta t$}}
\newcommand{\dthalf} {\mbox{$\frac{\Delta t}{2}$}}
\newcommand{\dtt} {\mbox{$\frac{\Delta t}{2}$}}
\newcommand{\dx} {\mbox{$\Delta x$}}
\newcommand{\dy} {\mbox{$\Delta y$}}

\newcommand{\beq} {\begin{equation}}
\newcommand{\eeq} {\end{equation}}
\newcommand{\beqa} {\begin{eqnarray}}
\newcommand{\eeqa} {\end{eqnarray}}

\newcommand{\olds} {\mbox{$\scriptstyle (0)$}}
\newcommand{\news} {\mbox{$\scriptstyle (1)$}}
\newcommand{\meds} {\mbox{$\scriptscriptstyle (\frac{1}{2})$}}
\newcommand{\half} {\mbox{$\scriptstyle \frac{1}{2}$}}

\newcommand{\nsz} {\normalsize}
\newcommand{\uold} {\mbox{$U^{\olds}$}}
\newcommand{\vold} {\mbox{$V^{\olds}$}}
\newcommand{\unew} {\mbox{$U^{\news}$}}
\newcommand{\vnew} {\mbox{$V^{\news}$}}
\newcommand{\zold} {\zeta^{(0)}}
\newcommand{\znew} {\zeta^{(1)}}
\newcommand{\resr} {{\cal R}}
\newcommand{\drho} {\frac{1}{\rho_{0}}}
\newcommand{\fracs}[2] {\mbox{$\frac{#1}{#2}$}}
%\newcommand{\deltat} {\mbox{$\tilde{\delta}$}}
%\newcommand{\gammat} {\mbox{$\tilde{\gamma}$}}
\newcommand{\ffxx} {\tilde{f_x}}
\newcommand{\ffyy} {\tilde{f_y}}

\newcommand{\uv} {{\bf U}}
\newcommand{\uvold} {{\bf U^{(0)}}}
\newcommand{\uvnew} {{\bf U^{(1)}}}
\newcommand{\af} {\alpha_{f}}
\newcommand{\ac} {\alpha_{c}}
\newcommand{\am} {\alpha_{m}}
\newcommand{\duv} {\Delta {\bf U}}
\newcommand{\dzeta} {\Delta \zeta}
\newcommand{\iv} {{\bf I}}
\newcommand{\ivh} {\hat{\bf I}}
\newcommand{\fv} {{\bf F}}
\newcommand{\uvh} {\hat{\bf U}}


%%%%%%%%%%%%%%%%%%%%%%%%%%%%%%%%%%%%%%%%%%%%%%%%%%%%%%%%%%
%%%%%%% hyphenation %%%%%%%%%%%%%%%%%%%%%%%%%%%%%%%%%%%%%%
%%%%%%%%%%%%%%%%%%%%%%%%%%%%%%%%%%%%%%%%%%%%%%%%%%%%%%%%%%


\section{Equations and Boundary Conditions}

The equations used in the model are the well known vertically integrated
shallow water equations in their formulation with water levels and
transports.

\beq \label{ubar}
\tdif{U} + gH \xdif{\zeta} + RU + X = 0
\eeq
\beq
\tdif{V} + gH \ydif{\zeta} + RV + Y = 0
\eeq
\beq \label{zcon}
\tdif{\zeta} + \xdif{U} + \ydif{V} = 0
\eeq
where $\zeta$ is the water level, $u,v$ the velocities in $x$ and $y$
direction,
$U,V$ the vertical integrated velocities (total  or barotropic
transports)
\[
 U = \int_{-h}^{\zeta} u \: dz \; \hspace{1.cm}
 V = \int_{-h}^{\zeta} v \: dz \;
\]
$g$ the gravitational acceleration, $H=h+\zeta$ the total water
depth, $h$ the undisturbed water depth,
$t$ the time and $R$ the friction coefficient. The terms $X,Y$ contain
all other terms that may be added to the equations like the wind stress or
the nonlinear terms and that need not be treated implicitly in the
time discretization.
following treatment.

The friction coefficient has been expressed as
\begin{equation}
	R = \frac{g \sqrt{u^{2}+v^{2}}}{C^{2} H}
\end{equation}
with $C$ the Chezy coefficient. The Chezy term is itself not retained
constant but varies with the water depth as
\begin{equation}
	C = k_{s} H^{1/6}
\end{equation}
where $k_{s}$ is the Strickler coefficient.

In this version of the model the Coriolis term, the turbulent friction term
and the nonlinear advective terms have not been implemented.

At open boundaries the water levels are prescribed. At closed boundaries
the normal velocity component is set to zero whereas the tangential velocity
is a free parameter. This corresponds to a full slip condition.


%%%%%%%%%%%%%%%%%%%%%%%%%%%%%%%%%%%%%%%%%%%%%%%%%%%%%%%%%%%%%%%%%%%%%%%%%
%%%%%%%%%%%%%%%%%%%%%%%%%%%%%%%%%%%%%%%%%%%%%%%%%%%%%%%%%%%%%%%%%%%%%%%%%
%%%%%%%%%%%%%%%%%%%%%%%%%%%%%%%%%%%%%%%%%%%%%%%%%%%%%%%%%%%%%%%%%%%%%%%%%
%%%%%%%%%%%%%%%%%%%%%%%%%%%%%%%%%%%%%%%%%%%%%%%%%%%%%%%%%%%%%%%%%%%%%%%%%
%%%%%%%%%%%%%%%%%%%%%%%%%%%%%%%%%%%%%%%%%%%%%%%%%%%%%%%%%%%%%%%%%%%%%%%%%





\section{The Model}

The model uses the semi-implicit time discretization to accomplish
the time integration. In the space the finite element method has
been used, not in its standard formulation, but using staggered finite
elements. In the following a description of the method is given.



%%%%%%%%%%%%%%%%%%%%%%%%%%%%%%%%%%%%%%%%%%%%%%%%%%%%%%%%%%%%%%%%%%%%%%%%%
%%%%%%%%%%%%%%%%%%%%%%%%%%%%%%%%%%%%%%%%%%%%%%%%%%%%%%%%%%%%%%%%%%%%%%%%%
%%%%%%%%%%%%%%%%%%%%%%%%%%%%%%%%%%%%%%%%%%%%%%%%%%%%%%%%%%%%%%%%%%%%%%%%%
%%%%%%%%%%%%%%%%%%%%%%%%%%%%%%%%%%%%%%%%%%%%%%%%%%%%%%%%%%%%%%%%%%%%%%%%%
%%%%%%%%%%%%%%%%%%%%%%%%%%%%%%%%%%%%%%%%%%%%%%%%%%%%%%%%%%%%%%%%%%%%%%%%%



\subsection{Discretization in Time - The Semi-Implicit Method}

Looking for an efficient time integration method
a semi-implicit scheme has been chosen.
The semi-implicit scheme combines the advantages of
the explicit and the implicit scheme. It is unconditionally stable for any
time step $\dt$ chosen and allows the two momentum equations to be
solved explicitly without solving a linear system. 

The only equation
that has to be solved implicitly is the continuity equation. Compared
to a fully implicit solution of the shallow water equations the dimensions
of the matrix are reduced to one third. Since the solution of a linear
system is roughly proportional to the cube of the dimension of the system
the saving in computing time is approximately a factor of 30.

It has to be pointed out that it is important not to be limited with the time
step by the CFL criterion for the speed of the external gravity waves
\[
        \dt < \frac{\dx}{\sqrt{gH}}
\]
where $\dx$ is the minimum distance between the nodes in an element.
With the discretization described below in most parts of the lagoon
we have $\dx \approx$ 500m and $H \approx$ 1m, so $\dt \approx 200$ sec.
But the limitation of the time step is determined by the worst case.
For example, for $\dx = 100$ m and $H = 40$ m
the time step criterion would be $\dt < 5$ sec, a
prohibitive small value.

The equations (1)-(3) are discretized as follows
\begin{equation}
\label{zn}
\frac{\zn-\zo}{\dt}
                        + \mez \frac{\dpp (\un + \uo)}{\dpp x}
                        + \mez \frac{\dpp (\vn + \vo)}{\dpp y} = 0
\end{equation}
\begin{equation}
\frac{\un-\uo}{\dt} + gH \dzxp + R \un + X = 0
\end{equation}
\begin{equation}
\frac{\vn-\vo}{\dt} + gH \dzyp + R \vn + Y = 0
\end{equation}

With this time discretization the friction term has been formulated
fully implicit, $X,Y$ fully explicit and all the other terms
have been centered in time. The reason for the implicit treatment
of the friction term is to avoid a sign inversion in the term when
the friction parameter gets too high. An example of this behavior is
given in Backhaus \cite{Backhaus83}.

If the two momentum equations are solved for the unknowns $\un$ and $\vn$
we have
\begin{equation}
\label{un}
\un = \frac{1}{1+\dt R} \paren{ \uo - \dt gH \dzxp - \dt X }
\end{equation}
\begin{equation}
\label{vn}
\vn = \frac{1}{1+\dt R} \paren{ \vo - \dt gH \dzyp - \dt Y }
\end{equation}

If $\zn$ were known, the solution for
$\un$ and $\vn$ could directly be given. To find $\zn$ we insert
(\ref{un}) and (\ref{vn}) in (\ref{zn}). After some transformations
(\ref{zn}) reads
\begin{eqnarray} \label{zsys}
        \zn
    & - &
        (\dt/2)^{2} \frac{g}{1+\dt R}         \nonumber
	\paren{ \frac{\dpp}{\dpp x}(H \dzxn) + \frac{\dpp}{\dpp y}(H \dzyn) } \\
    & = &
        \zo + (\dt/2)^{2} \frac{g}{1+\dt R}
	\paren{ \frac{\dpp}{\dpp x}(H \dzxo) + \frac{\dpp}{\dpp y}(H \dzyo) } \\
    & - & (\dt/2) \paren{ \frac{2+\dt R}{1+\dt R} }
        \paren{                               \nonumber
          \frac{\dpp \uo}{\dpp x}
        + \frac{\dpp \vo}{\dpp y}
        } \\
    & + & \frac{\dt^{2}}{2(1+\dt R)}            \nonumber
                \paren{ \frac{\dpp X}{\dpp x} + \frac{\dpp Y}{\dpp y} }
\end{eqnarray}

The terms on the left hand side contain the unknown $\zn$, the right hand
contains only known values of the old time level. If the spatial derivatives
are now expressed by the finite element method a linear system with the unknown
$\zn$ is obtained and can be solved by standard methods. Once the solution
for $\zn$ is obtained it can be substituted into (\ref{un}) and (\ref{vn})
and these two equations can be solved explicitly. In this way all unknowns
of the new time step have been found.

Note that the variable $H$ also contains the water level through
$H=h+\zz$. In order to avoid the equations to become nonlinear $\zz$
is evaluated at the old time level so $H=h+\zo$ and $H$ is a known quantity.


%%%%%%%%%%%%%%%%%%%%%%%%%%%%%%%%%%%%%%%%%%%%%%%%%%%%%%%%%%%%%%%%%%%%%%%%%
%%%%%%%%%%%%%%%%%%%%%%%%%%%%%%%%%%%%%%%%%%%%%%%%%%%%%%%%%%%%%%%%%%%%%%%%%
%%%%%%%%%%%%%%%%%%%%%%%%%%%%%%%%%%%%%%%%%%%%%%%%%%%%%%%%%%%%%%%%%%%%%%%%%
%%%%%%%%%%%%%%%%%%%%%%%%%%%%%%%%%%%%%%%%%%%%%%%%%%%%%%%%%%%%%%%%%%%%%%%%%
%%%%%%%%%%%%%%%%%%%%%%%%%%%%%%%%%%%%%%%%%%%%%%%%%%%%%%%%%%%%%%%%%%%%%%%%%




%%%%%%%%%%%%%%%%%%%%%%%%%%%%%%%%%%%%%%%%%%%%%%%%%%%%%%%%%%%%%%%%%%%%%%%%%
%%%%%%%%%%%%%%%%%%%%%%%%%%%%%%%%%%%%%%%%%%%%%%%%%%%%%%%%%%%%%%%%%%%%%%%%%
%%%%%%%%%%%%%%%%%%%%%%%%%%%%%%%%%%%%%%%%%%%%%%%%%%%%%%%%%%%%%%%%%%%%%%%%%
%%%%%%%%%%%%%%%%%%%%%%%%%%%%%%%%%%%%%%%%%%%%%%%%%%%%%%%%%%%%%%%%%%%%%%%%%
%%%%%%%%%%%%%%%%%%%%%%%%%%%%%%%%%%%%%%%%%%%%%%%%%%%%%%%%%%%%%%%%%%%%%%%%%



\subsection{Discretization in Space - The Finite Element Method}


While the time discretization has been explained above, the discretization
in space has still to be carried out. This is done 
using staggered finite elements. 
With the semi-implicit method described above
it is shown below that using linear triangular elements
for all unknowns 
will not be mass conserving. Furthermore the resulting model
will have propagation properties that introduce high numeric damping
in the solution of the equations.

For these reasons a quite new approach has been adopted here. The water
levels and the velocities (transports) are described by using form
functions of different order, being the standard linear form functions
for the water levels but stepwise constant form functions for the
transports. This will result in a grid that resembles more a staggered
grid in finite difference discretizations.

\subsubsection{Formalism}

Let $u$ be an approximate solution of a linear differential
equation $L$. We expand $u$ with the help of basis functions $\phi_{m}$
as
\begin{equation}
\label{exp}
	u=\phi_{m} u_{m} \mbox{\hspace{1cm}} m=1,K
\end{equation}
where $u_{m}$ is the coefficient of the function $\phi_{m}$ and $K$
is the order of the approximation.
In case of linear finite
elements it will just be the number of nodes of the grid used to
discretize the domain.

To find the values $u_{m}$ we try to minimize the residual
that arises when $u$ is introduced into $L$ multiplying the equation $L$
by some weighting functions $\Psi_{n}$ and
integrating over the whole domain leading to
\begin{equation}
\label{int}
\int_{\Omega} \psi_{n} L(u) \: d\Omega \; = 
\int_{\Omega} \psi_{n} L(\phi_{m} u_{m}) \: d\Omega \;
= u_{m} \int_{\Omega} \psi_{n} L(\phi_{m}) \: d\Omega \;
\end{equation}

If the integral is identified with the elements of a matrix $a_{nm}$
we can write (\ref{int}) also as a linear system
\begin{equation} \label{sys}
	a_{nm}u_{m} = 0 \mbox{\hspace{1cm}} n=1,K \hspace{0.5cm} m=1,K
\end{equation}

Once the basis and weighting functions have been specified the system
may be set up and (\ref{sys}) may be solved for the unknowns $u_{m}$.






\subsubsection{Staggered Finite Elements}

For decades finite elements have been used in fluid mechanics in
a standardized manner.
The form functions $\phi_{m}$ were chosen as continuous piecewise linear
functions allowing a subdivision of the whole area of interest into small
triangular elements specifying the coefficients $u_{m}$ at the vertices
(called nodes)
of the triangles. The functions $\phi_{m}$ are 1 at node
$m$ and 0 at all other nodes and thus different from 0 only in the
triangles containing the node $m$.
An example is given in the upper left part of Fig. 1a
where the form function for node $i$ is shown. The full circle indicates
the node where the function $\phi_{i}$ take the value
1 and the hollow circles where they are 0.


\begin{figure}
\vspace{5.cm}
\caption{a) form functions in domain \hspace{1.cm} b) domain of
influence of node $i$}
%\end{figure}

\begin{picture}(300,10)

\put(10,30){
\begin{picture}(1,1)
\put(80,90){\line(0,1){30}}
\put(80,90){\line(3,1){30}}
\put(80,90){\line(2,-3){20}}
\put(80,90){\line(-2,-3){20}}
\put(80,90){\line(-3,1){30}}
\put(60,60){\line(1,0){40}}
\put(60,60){\line(1,-2){20}}
\put(60,60){\line(-2,-3){20}}
\put(60,60){\line(-4,1){40}}
\put(60,60){\line(-1,4){10}}
\put(50,100){\line(-1,-1){30}}
\put(50,100){\line(-3,1){30}}
\put(50,100){\line(-1,4){10}}
\put(50,100){\line(3,2){30}}
\put(80,120){\line(-2,1){40}}
\put(80,120){\line(1,3){10}}
\put(80,120){\line(3,-2){30}}
\put(110,100){\line(-2,5){20}}
\put(110,100){\line(1,1){30}}
\put(110,100){\line(1,-1){30}}
\put(110,100){\line(-1,-4){10}}
\put(100,60){\line(4,1){40}}
\put(100,60){\line(3,-4){30}}
\put(100,60){\line(-1,-2){20}}
\put(130,20){\line(1,5){10}}
\put(130,20){\line(-1,0){50}}
\put(140,130){\line(0,-1){60}}
\put(140,130){\line(-5,2){50}}
\put(40,140){\line(5,1){50}}
\put(40,140){\line(-2,-3){20}}
\put(20,110){\line(0,-4){40}}
\put(40,30){\line(-1,2){20}}
\put(40,30){\line(4,-1){40}}
\thicklines
\put(50,120){\line(1,-1){30}}
\put(50,120){\line(1,0){30}}
\put(50,120){\line(-1,2){10}}
\put(50,120){\line(-3,-1){30}}
\put(50,120){\line(-3,-5){30}}
\put(50,120){\line(1,-6){10}}
\put(50,120){\line(0,-1){20}}
\put(100,80){\line(-1,-2){20}}
\put(100,80){\line(3,-4){30}}
\put(100,80){\line(0,-2){20}}
\put(80,40){\line(0,-2){20}}
\put(80,40){\line(1,0){50}}
\put(130,40){\line(0,-2){20}}
\thinlines
\put(100,30){$n$}
\put(45,88){$i$}
\put(5,90){$\phi_{i}$}
\put(100,10){$\psi_{n}$}
%\put(80,90){\circle*{5}}
%\put(60,60){\circle*{5}}
\put(50,100){\circle*{5}}
%\put(80,120){\circle*{5}}
%\put(110,100){\circle*{5}}
%\put(100,60){\circle*{5}}
\put(100,60){\circle*{5}}
\put(80,20){\circle*{5}}
\put(130,20){\circle*{5}}
%\put(50,100){\circle{7}}
\put(80,90){\circle{7}}
\put(80,120){\circle{7}}
\put(60,60){\circle{7}}
\put(20,70){\circle{7}}
\put(20,110){\circle{7}}
\put(40,140){\circle{7}}
\end{picture}}
%\end{center}



%\put(250,30){
\put(200,30){
\begin{picture}(1,1)
\put(80,90){\line(0,1){30}}
\put(80,90){\line(3,1){30}}
\put(80,90){\line(2,-3){20}}
\put(80,90){\line(-2,-3){20}}
\put(80,90){\line(-3,1){30}}
\put(60,60){\line(1,0){40}}
\put(60,60){\line(1,-2){20}}
\put(60,60){\line(-2,-3){20}}
\put(60,60){\line(-4,1){40}}
\put(60,60){\line(-1,4){10}}
\put(50,100){\line(-1,-1){30}}
\put(50,100){\line(-3,1){30}}
\put(50,100){\line(-1,4){10}}
\put(50,100){\line(3,2){30}}
\put(80,120){\line(-2,1){40}}
\put(80,120){\line(1,3){10}}
\put(80,120){\line(3,-2){30}}
\put(110,100){\line(-2,5){20}}
\put(110,100){\line(1,1){30}}
\put(110,100){\line(1,-1){30}}
\put(110,100){\line(-1,-4){10}}
\put(100,60){\line(4,1){40}}
\put(100,60){\line(3,-4){30}}
\put(100,60){\line(-1,-2){20}}
\put(130,20){\line(1,5){10}}
\put(130,20){\line(-1,0){50}}
\put(140,130){\line(0,-1){60}}
\put(140,130){\line(-5,2){50}}
\put(40,140){\line(5,1){50}}
\put(40,140){\line(-2,-3){20}}
\put(20,110){\line(0,-4){40}}
\put(40,30){\line(-1,2){20}}
\put(40,30){\line(4,-1){40}}
\put(85,95){$i$}
\put(53,107){$j$}
\put(80,90){\circle*{5}}
\put(60,60){\circle*{5}}
\put(50,100){\circle*{5}}
\put(80,120){\circle*{5}}
\put(110,100){\circle*{5}}
\put(100,60){\circle*{5}}
\put(50,100){\circle{7}}
\put(80,90){\circle{7}}
\put(80,120){\circle{7}}
\put(60,60){\circle{7}}
\put(20,70){\circle{7}}
\put(20,110){\circle{7}}
\put(40,140){\circle{7}}
\end{picture}}

\end{picture}


\end{figure}


The contributions $a_{nm}$ to the system matrix
are therefore different from 0 only in
elements containing node $m$ and the evaluation of the matrix elements
can be performed on an element basis where all coefficients and unknowns
are linear functions of $x$ and $y$.

This approach is straightforward but not very satisfying with the
semi-implicit time stepping scheme for reasons explained below.
Therefore
an other way has been followed in the present formulation. The fluid domain
is still divided in triangles and the water levels are still defined
at the nodes of the grid
and represented by piecewise linear interpolating functions
in the internal of each element, i.e.
\[
        \zeta = \zeta_{m} \phi_{m} \hspace{1cm} m=1,K
\]
However, the transports are now
expanded, over each triangle, with piecewise constant
(non continuous) form functions $\psi_{n}$ over the whole domain. We therefore
write
\[
        U = U_{n} \psi_{n} \hspace{1cm} n=1,J
\]
where $n$ is now running over all
triangles and $J$ is the total number of triangles.
An example of $\psi_{n}$ is given in the lower right part of Fig. 1a.
Note that the form function is constant 1 over the whole element,
but outside the element identically 0. Thus it is discontinuous
at the element borders.

Since we may
identify the center of gravity of the triangle with the point where
the transports $U_{n}$ are defined (contrary to the water levels
$\zeta_{m}$ which are defined on the vertices of the triangles), the
resulting grid may be seen as a staggered grid where the unknowns
are defined on different locations. This kind of grid is usually used
with the finite difference method. With the form functions used here
the grid of the finite element model resembles
very much an Arakawa B-grid that defines the water levels on the center
and the velocities on the four vertices of a square.

Staggered finite elements have been first introduced into
fluid mechanics by Schoenstadt \cite{Schoenstadt80}. 
He showed that the un-staggered
finite element formulation of the shallow water equations has very
poor geostrophic adjustment properties. Williams 
\cite{Williams81a, Williams81b}
proposed a similar algorithm, the one
actually used in this paper, introducing constant form functions for the
velocities. He showed the excellent propagation and geostrophic
adjustment properties of this scheme.


\subsubsection{The Practical Realization}

The integration of the partial differential equation is now performed by
using the subdivision of the domain in elements (triangles). The
water levels $\zeta$ are expanded in piecewise linear functions
$\phi_{m}, \; m=1,K$ and
the transports are expanded in piecewise constant functions
$\psi_{n}, \; n=1,J$ where $K$ and $J$ are the total number of nodes
and elements respectively.

As weighting functions we use $\psi_{n}$ for the momentum equations
and $\phi_{m}$ for the continuity equation. In this way there will
be $K$ equations for the unknowns $\zeta$ (one for each node) and
$J$ equations for the transports (one for each element).

In all cases the consistent mass matrix has been substituted with
the their lumped equivalent. This was mainly done
to avoid solving a linear system in the case of the momentum equations.
But it was of use also in the solution of the continuity equation
because the amount of mass relative to 
one node does not depend on the surrounding
nodes. This was important especially for the flood and dry mechanism
in order to conserve mass.


\subsubsection{Finite Element Equations}

If equations (\ref{un},\ref{vn},\ref{zsys}) are multiplied with their
weighting functions and integrated over an element we can write down
the finite element equations. But the solution of the water levels does
actually not use the continuity equation in the form (\ref{zsys}), but
a slightly different formulation. Starting from equation (\ref{zn}),
multiplied by the weighting function $\Phi_{M}$ and integrated over one
element yields


\[
          \int_{\Omega} \Phi_{N} (\zn-\zo) \: d\Omega \;
+ (\dthalf) \int_{\Omega} 
	  \left( 
	   \Phi_{N} \frac{\dpp (\un + \uo)}{\dpp x} 
+          \Phi_{N} \frac{\dpp (\vn + \vo)}{\dpp y} 
          \right)
	  \: d\Omega \;
= 0
\]
If we integrate by parts the last two integrals we obtain
\[
          \int_{\Omega} \Phi_{N} (\zn-\zo) \: d\Omega \;
- (\dthalf) \int_{\Omega} 
	  \left( 
	    \frac{\dpp \Phi_{N}}{\dpp x} (\un+\uo) 
+           \frac{\dpp \Phi_{N}}{\dpp y} (\vn+\vo)
	  \right)
	  \: d\Omega \;
= 0
\]
plus two line integrals, not shown, over the boundary of each element
that specify the normal flux over the three element
sides. In the interior of the domain,
once all contributions of all elements have
been summed, these terms cancel at every node,
leaving only the contribution of the
line integral on the boundary of the domain. There, however, the
boundary condition to impose is exactly no normal flux over
material boundaries. Thus, the contribution of these line integrals
is zero.

If now the expressions for $\un,\vn$ are introduced, we obtain a system
with again only the water levels as unknowns
\beqa
\int_{\Omega} \Phi_{N} \zn \: d\Omega \;
 & + & (\dt/2)^{2} \alpha g 
\int_{\Omega} H ( \xdif{\Phi_{N}} \xdif{\zn}  \nonumber
 + \ydif{\Phi_{N}} \ydif{\zn} ) \: d\Omega \; \\
 & = &
\int_{\Omega} \Phi_{N} \zo \: d\Omega \;	\nonumber
+ (\dt/2)^{2} \alpha g 
\int_{\Omega} H ( \xdif{\Phi_{N}} \xdif{\zo} 
 + \ydif{\Phi_{N}} \ydif{\zo} ) \: d\Omega \;  \\
 & + &
 (\dt/2)(1+\alpha) \int_{\Omega}  
  ( \xdif{\Phi_{N}} \uo + \ydif{\Phi_{N}} \vo ) \: d\Omega \; \\
 & - & (\dt^{2}/2) \alpha \nonumber
\int_{\Omega} ( \xdif{\Phi_{N}} X + \ydif{\Phi_{N}} Y ) \: d\Omega \; 
\eeqa
Here we have introduced the symbol $\alpha$ as a shortcut for
\[
\alpha = \frac{1}{1+\dt R}
\]
The variables and unknowns may now be expanded with their basis
functions and the complete system may be set up.


%%%%%%%%%%%%%%%%%%%%%%%%%%%%%%%%%%%%%%%%%%%%%%%%%%%%%%%%%%%%%%%%%%%%%%%%%
%%%%%%%%%%%%%%%%%%%%%%%%%%%%%%%%%%%%%%%%%%%%%%%%%%%%%%%%%%%%%%%%%%%%%%%%%
%%%%%%%%%%%%%%%%%%%%%%%%%%%%%%%%%%%%%%%%%%%%%%%%%%%%%%%%%%%%%%%%%%%%%%%%%
%%%%%%%%%%%%%%%%%%%%%%%%%%%%%%%%%%%%%%%%%%%%%%%%%%%%%%%%%%%%%%%%%%%%%%%%%
%%%%%%%%%%%%%%%%%%%%%%%%%%%%%%%%%%%%%%%%%%%%%%%%%%%%%%%%%%%%%%%%%%%%%%%%%

\subsection{Mass Conservation}

It should be pointed out that only through the use of this staggered grid
the semi-implicit time discretization may be implemented in a feasible
manner. If the Galerkin method is applied
 in a naive way to the resulting equation
(\ref{zsys}) (introducing the linear form functions for transports
and water levels and setting up the system matrix),
the model is not mass conserving.
This may be seen in the following way (see Fig. 1b for reference).
In the computation of the water level at
node $i$, only $\zeta$ and transport values
belonging to triangles that contain node $i$ enter the computation
(full circles in Fig. 1b).
But when, in a second step, the barotropic transports
of node $j$ are computed, water levels of nodes that lie further apart
from the original node $i$ are used
(hollow circles in Fig. 1b).
These water levels have not been included in
the computation of $\zeta_i$, the water level at node $i$.
So the computed transports are actually different
from the transports inserted formally in the continuity equation.
The continuity equation is therefore not satisfied.

These contributions of nodes lying further apart could in principle
be accounted for. In this case
not only the triangles
$\Omega_{i}$ around node $i$ but also all the triangles that have
nodes in common with the triangles $\Omega_{i}$ would give
contributions to node $i$, namely all nodes and elements shown
in Fig. 1b.
The result would be
an increase of the bandwidth of the matrix for the $\zeta$ computation
disadvantageous in terms of memory and time requirements.

Using instead the approach of the staggered finite elements, actually
only the water levels of elements around node $i$ are needed for
the computation of the transports in the triangles $\Omega_i$.
In this case the model satisfies the
continuity equation and is perfectly mass conserving.



\subsection{Inter-tidal Flats}

Part of a basin may consist of areas that are
flooded during high tides and emerge as islands at ebb tide. These
inter-tidal flats are quite difficult to handle numerically because
the elements that represent these areas are neither
islands nor water elements. The boundary line defining their
contours is wandering during the evolution
of time and a mathematical model must reproduce this features.

For reasons of computer time savings a simplified algorithm has been chosen
to represent the inter-tidal flats. When the water level in at least
one of the three nodes of an element falls below a minimum value (5 cm)
the element is considered an island and is taken out of the system.
It will be reintroduced only when in all three
nodes the water level is again higher then the minimum value.
Because in dry nodes no water level is computed anymore, an estimate
of the water level has to be given with some sort of extrapolation mechanism
using the water nodes nearby.

This algorithm has the advantage that it is very easy to
implement and very fast. The dynamical features close to the
inter-tidal flats are of course not well reproduced but the
behavior of the method for the rest of the lagoon
gave satisfactory results.

In any case, since the method stores the water levels of the
last time step, before the element is switched off, introducing the
element in a later moment with the same water levels conserves the
mass balance. This method showed a much better performance
than the one where the new elements were introduced with the water
levels taken from the extrapolation of the surrounding nodes.

%%%%%%%%%%%%%%%%%%%%%%%%%%%%%%%%%%%%%%%%%%%%%%%%%%%%%%%%%%%%%%%%%%%%%%%%%
%%%%%%%%%%%%%%%%%%%%%%%%%%%%%%%%%%%%%%%%%%%%%%%%%%%%%%%%%%%%%%%%%%%%%%%%%
%%%%%%%%%%%%%%%%%%%%%%%%%%%%%%%%%%%%%%%%%%%%%%%%%%%%%%%%%%%%%%%%%%%%%%%%%
%%%%%%%%%%%%%%%%%%%%%%%%%%%%%%%%%%%%%%%%%%%%%%%%%%%%%%%%%%%%%%%%%%%%%%%%%
%%%%%%%%%%%%%%%%%%%%%%%%%%%%%%%%%%%%%%%%%%%%%%%%%%%%%%%%%%%%%%%%%%%%%%%%%



\chapter{File formats}

%	\section{STR file}

	\section{GRD file}
	
\begin{code}

format of input file for grid utility

=============================================================

legend :

n	item number (node, element, line)
t	type
d	depth
x,y	coordinates
ntot	number of following nodes
n1,n2	node numbers

=============================================================

format of lines in input file :

comment:

0 [anything]

node:

1	n	t	x	y	[d]

element:

2	n	t	ntot	n1 n2 n3 ... 	[d]

line:

3	n	t	ntot	n1 n2 ...	[d]

=============================================================

comment :

lines may be split at any point, except befor optional argument
d must not be on seperate line
if line is split, the continuation line(s) must start with a blank
blank lines can be inserted as needed
if d is not specified -999. will be used (flag)
use t=0 if you do not know what to use
n must be unique for every item type 
item numbers need not be consecutive
the sequence of items is not important, nodes can be mixed with elements/lines
the minimum number of nodes for element items is 3
the minimum number of nodes for line items is 2
element items should have all nodes unique
line items with the same first and last node are considered closed

=============================================================

example 1 :

0 example of one line

1 11 0 10. 10.
1 12 0 20. 20.

3 7 0 2 11 12

#----------------

example 2 :

0 example of one element with continuation line

1 11 0 10. 10.
1 12 0 20. 20.
1 15 0 10. 20.

2 7 0 3 
   11 12 15

=============================================================

\end{code}



%	\section{Time series}

%	\section{Output file formats}





\chapter{Parameter list}


\section{Parameter list for the SHYFEM model}


\subsection{Section {\tt \$title}}

This section must be always the first section in the parameter input file.
It contains only three lines. An example is given in 
figure \ref{fig:titleexample}.

\begin{figure}[ht]
\begin{verbatim}
$title
        free one line description of simulation
        name_of_simulation
        name_of_basin
$end
\end{verbatim}
\caption{Example of section {\tt \$title}}
\label{fig:titleexample}
\end{figure}

The first line of this section is a free one line description of
the simulation that is to be carried out. The next line contains
the name of the simulation.
All created files will use this name in the main part of the file name
with different extensions. Therefore the hydrodynamic output file
(extension |out|) will be named |name_of_simulation.out|.
The last line gives the name of the basin file to be used. This
is the pre-processed file of the basin with extension |bas|.
In our example the basin file |name_of_basin.bas| is used.

The directory where this files are read from or written to depends
on the settings in section {\tt \$name}. Using the default
the program will read from and write to the current directory.

\subsection{Section {\tt \$para}}

This section defines the general behavior of the simulation,
gives various constants of parameters and determines what
output files are written. In the following the meaning of
all possible parameters is given.

Note that the only compulsory parameters in this section are 
the ones that chose the duration of the simulation and the
integration time step. All other parameters are optional.

\input{S_para_h.tex}

\subsection{Section {\tt \$waves}}

This section defines the parameters for the wind wave module.

\input{P_wave.tex}

\subsection{Section {\tt \$sedtr}}

This section defines the parameters for the sediment 
transport module.

\input{P_sediment.tex}

\subsection{Section {\tt \$name}}

In this sections names of directories or input files can be
given. All directories default to the current directory,
whereas all file names are empty, i.e., no input files are
given.

\input{S_name.tex}
\input{S_name_h.tex}

\subsection{Section {\tt \$bound}}

\input{S_bound.tex}


\subsection{Section {\tt \$wind}}

\input{S_wind.tex}



\subsection{Section {\tt \$extra}}

In this section the node numbers of so called ``extra'' points are given. 
These are points where water level and velocities are written to create
a time series that can be elaborated later. The output for these ``extra''
points consumes little memory and can be therefore written with a
much higher frequency (typically the same as the integration time step)
than the complete hydrodynamic output. The output is written
to file EXT.

The node numbers are specified in a free format on one ore more lines.
An example can be seen in figure \ref{fig:example}. No keywords
are expected in this section.


\subsection{Section {\tt \$flux}}

In this section transects are specified through which the discharge
of water is computed by the program and written to file FLX.
The transects are defined by their nodes through which they run.
All nodes in one transect must be adjacent, i.e., they must form a
continuous line in the FEM network.

The nodes of the transects are specified in free format. Between
two transects one or more 0's must be inserted. An example is given in
figure \ref{fig:fluxexample}.

\begin{figure}[ht]
\begin{verbatim}
$flux
	1001 1002 1004 0
	35 37 46 0 0 56 57 58 0
	407
	301
	435 0 89 87
$end
\end{verbatim}
\caption{Example of section {\tt \$flux}}
\label{fig:fluxexample}
\end{figure}

The example shows the definition of 5 transects. As can be seen, the 
nodes of the transects can be given on one line alone (first transect),
two transects on one line (transect 2 and 3), spread over more lines
(transect 4) and a last transect.



\section{Parameter list for the post processing routines}

The format of the parameter input file is the same as the one for
the main routine. Please see this section for more information
on the format of the parameter input file.

Some sections of the parameter input file are identical to the 
sections used in the main routine. For easier reference we will
repeat the possible parameters of these section here.






\subsection{Section {\tt \$title}}

This section must be always the first section in the parameter input file.
It contains only three lines. An example has been given already in 
figure \ref{fig:titleexample}.

The only difference with respect to the {\tt \$para} section of the main
routine is the first line. Here any description of the output can be used.
It is just a way to label the parameter file.  The other two line with
the name of simulation and the basin are used to open the files needed
for plotting.


\subsection{Section {\tt \$para}}

\input{S_para_a.tex}


\subsection{Section {\tt \$color}}

\input{S_color.tex}


\subsection{Section {\tt \$arrow}}

\input{S_arrow.tex}


\subsection{Section {\tt \$legend}}

\input{S_legend.tex}


\subsection{Section {\tt \$legvar}}

\input{S_legvar.tex}


\subsection{Section {\tt \$name}}

\input{S_name.tex}







\bibliography{abbrev,lag,sedi}
\addcontentsline{toc}{chapter}{Bibliography}



\end{document}




