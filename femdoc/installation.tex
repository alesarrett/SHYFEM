
\newcommand{\sysfiles}{.bashrc .bash\_profile .profile}

Before compiling it is advisable to install some files for a simpler
usage of the model. As long as you only want to tun a simulation, this
step is not strictly necessary. But if you will run some scripts of the
distribution, these scripts will not work properly if you do not install
the model.

In order to install the model, you should run

\begin{code}
    make install
\end{code}

This command will do the following:

\begin{itemize}

\item It hardcodes the installation directory in all scripts of the
model so only programs of the installed version will be executed.

\item It inserts a symbolic link |syhfem| from the home directory to
the root of the SHYFEM installation.

\item It inserts a small snippet of code into the initialization files
\ttt{\sysfiles} that are in your home directory. This will adjust your
path to point to the SHYFEM directory and gives you access to some
administrative commands.

\end{itemize}

After this command you will find the original files that have been changed
in your home directory saved with a trailing number (e.g., |.profile.35624|
or similar).  If you encounter problems, just substitute back these files.

In order that your new settings will take effect you will have to log
out and log in again. Alternatively, you will have to execute one of
the files \ttt{\sysfiles} that have actually changed. If |.bashrc| has been
changed, then run |. ~/.bashrc|.

If you do not want to run the installation routine, you should at least
manually insert a symbolic link to the root of the SHYFEM model.
Otherwise some of the commands and shell scripts will not work
properly. This can be done with the command

\begin{codem}
    cd
    ln -s \shydir shyfem
\end{codem}

from your home directory. If there is already such a link existing you
will first have to delete it (|rm shyfem|).

You should also think about the possibility to add the fembin path to your
default paths to have the main utility commands always available. To do
this open your |.bashrc| file in the home directory and add the following
lines at the end of the file:

\begin{verbatim}
SHYFEMDIR=$HOME/fem
PATH=$PATH:$SHYFEMDIR/fembin
export SHYFEMDIR PATH
\end{verbatim}

However, the same effect can be achieved more easily by using the
abovementioned command |make install|.

If you ever want to uninstall the model, you can do it with the command
|make uninstall|. This will delete the symbolic link, cancel the hard
links in the model scripts and restore the systemfiles \ttt{\sysfiles}
to their original content.

Please note that you will still have to delete manually the model
directory. This can be done with the command \ttt{rm -rf \shydir}). In
this way, however, changes to the code you have made will be lost.

