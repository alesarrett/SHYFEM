
% $Id$

\documentclass{report}

\usepackage{a4}
\usepackage{shortvrb}
\usepackage{pslatex}
\usepackage{alltt}		% as verbatim, but interpret \ { }

\usepackage{version}
%\newenvironment{comment}{}{}

\newcommand{\shy}{{\tt SHYFEM}}
\newcommand{\psp}{{\tt SHYFEM}}
%\newcommand{\VERSION}{4.31}

\input{P_version.tex}

\newcommand{\descrpsep}{\vspace{0.2cm}}
\newcommand{\descrpitem}[1]{\descrpsep\parbox[t]{2cm}{#1}}
\newcommand{\descrptext}[1]{\parbox[t]{10cm}{#1}\descrpsep}
\newcommand{\densityunit}{kg\,m${}^{-3}$}
\newcommand{\accelunit}{m\,s${}^{-2}$}
\newcommand{\maccelunit}{m${}^{4}$\,s${}^{-2}$}
\newcommand{\dischargeunit}{m${}^3$\,s${}^{-1}$}

\newcommand{\ten}[1]{$\cdot 10^{#1}$}
\newcommand{\degrees}{${}^o$}

\parindent 0cm

\MakeShortVerb{\|}


%%%%%%%%%%%%%%%%%%%%%%%%%%%%%%%%%%%%%%%%%%%%%%%%%%%%%%%%% front matter

\title{%
	\shy{} 
	\\Finite Element Model for Coastal Seas
	\\~
	\\User Manual
	}

\author{%
	Georg Umgiesser
	\\ISDGM-CNR, Castello 1364/A
	\\30122 Venezia, Italy
	\vspace{0.5cm}
	\\georg.umgiesser@ismar.cnr.it
	\vspace{1cm}
	\\Version \VERSION
	}

%\address{ISDGM-CNR}

%%%%%%%%%%%%%%%%%%%%%%%%%%%%%%%%%%%%%%%%%%%%%%%%%%%%%%%%% document

\begin{document}

\bibliographystyle{plain}

\pagenumbering{roman}
\pagestyle{plain}

\maketitle

%\begin{abstract}
%A simple graphic plotting package is presented that can be used for
%the creation of PostScript graphics. The routines are callable from
%Fortran and C.
%
%Only the basic plotting commands have been implemented. The library
%allows you to plot lines and points, fill arbitrary shapes
%with arbitrary color and write text with an arbitrary point size.
%It also allows for producing more pages in one plot. Color can be
%used as gray scale or through the RGB and HSB color spaces.
%The coordinate system may be set to best adjust to the drawing.
%Clipping graphics in a given rectangle is implemented.
%\end{abstract}

\thispagestyle{empty}

\newpage

\tableofcontents

\newpage





\chapter*{Disclaimer}
\addcontentsline{toc}{chapter}{Disclaimer}


\begin{quotation}
  									 
   Copyright (c) 1992-1998 by Georg Umgiesser				 
  									 
   Permission to use, copy, modify, and distribute this software	 
   and its documentation for any purpose and without fee is hereby	 
   granted, provided that the above copyright notice appear in all	 
   copies and that both that copyright notice and this permission	 
   notice appear in supporting documentation.				 
  									 
   This file is provided AS IS with no warranties of any kind.		 
   The author shall have no liability with respect to the		 
   infringement of copyrights, trade secrets or any patents by		 
   this file or any part thereof.  In no event will the author		 
   be liable for any lost revenue or profits or other special,		 
   indirect and consequential damages.					 
  									 
   Comments and additions should be sent to the author:			 
  									 
	\begin{verbatim}
  			Georg Umgiesser					 
  			ISDGM/CNR					 
  			S. Polo 1364					 
  			30125 Venezia					 
  			Italy						 
  									 
  			Tel.   : ++39-41-5216875			 
  			Fax    : ++39-41-2602340			 
  			E-Mail : georg@lagoon.isdgm.ve.cnr.it		 
	\end{verbatim}
\end{quotation}



\begin{comment}

\chapter*{Availability}
\addcontentsline{toc}{chapter}{Availability}


The library \shy{} is available free of charge bye anonymous ftp.
Connect to
|ftp.isdgm.ve.cnr.it| and look in the directory |/pub/|\shy{}.
Please read the README file of the distribution. Only source code
is available.

The model should compile out of the box for nearly all Unix-like
systems. Please follow the instructions in the README file.

Please send bug reports to the author (|georg@lagoon.isdgm.ve.cnr.it|).

\end{comment}


\newpage

\pagenumbering{arabic}










\chapter{Introduction}


The finite element program \shy{} is a program package that can be used
to resolve the hydrodynamic equations in lagoons, coastal seas,
estuaries and lakes. The program uses finite elements for the
resolution of the hydrodynamic equations. These finite elements,
together with an effective semi-implicit time resolution algorithm,
makes this program especially suitable for application to a complicated
geometry and bathymetry.

This version of the program \shy{} resolves the depth integrated
shallow water equations. It is therefore recommended for the
application of very shallow basins or well mixed estuaries. Storm surge
phenomena can be investigated also.  This two-dimensional version of
the program is not suited for the application to baroclinic driven
flows or large scale flows where the the Coriolis acceleration is
important.

Finite elements are superior to finite differences when dealing with
complex bathymetric situations and geometries. Finite differences are
limited to a regular outlay of their grids. This will be a problem if
only parts of a basin need high resolution.  The finite element method
has an advantage in this case allowing more flexibility with its
subdivision of the system in triangles varying in form and size.

This model is especially adapted to run in very shallow basins. It is
possible to simulate shallow water flats, i.e., tidal marshes that in a
tidal cycle may be covered with water during high tide and then fall
dry during ebb tide. This phenomenon is handled by the model in a mass
conserving way.

Finite element methods have been introduced into hydrodynamics since
1973 and have been extensively applied to shallow water equations by
numerous authors \cite{Grotkop73, Taylor75, Herrling77, Herrling78, Holz82}.

The model presented here \cite{Umgies86, Umgies93} uses the mathematical
formulation of the semi-implicit algorithm that decouples the solution
of the water levels and velocity components from each other leading to
smaller systems to solve. Models of this type have been presented from
1971 on by many authors \cite{Kwizak71, Duwe82, Backhaus83}.



\chapter{Equations and resolution techniques}


%%%%%%%%%%%%%%%%%%%%%%%%%%%%%%%%%%%%%%%%%%%%%%%%%%%%%%%%%%
%%%%%%% user commands %%%%%%%%%%%%%%%%%%%%%%%%%%%%%%%%%%%%
%%%%%%%%%%%%%%%%%%%%%%%%%%%%%%%%%%%%%%%%%%%%%%%%%%%%%%%%%%

\newcommand{\paren}[1]	{ \left( #1 \right) }
\newcommand{\mez}{\mbox{$\frac{1}{2}$}}
\newcommand{\dpp}{\mbox{$\partial$}}
\newcommand{\zz}{\zeta}
\newcommand{\un}{\mbox{$U^{n+1}$}}
\newcommand{\uo}{\mbox{$U^{n}$}}
\newcommand{\vn}{\mbox{$V^{n+1}$}}
\newcommand{\vo}{\mbox{$V^{n}$}}
\newcommand{\zn}{\mbox{$\zz^{n+1}$}}
\newcommand{\zo}{\mbox{$\zz^{n}$}}
\newcommand{\up}{\mbox{$U^{\prime}$}}
\newcommand{\vp}{\mbox{$V^{\prime}$}}
\newcommand{\zp}{\mbox{$\zz^{\prime}$}}
\newcommand{\dzxp}{\mez \frac{\dpp (\zn + \zo)}{\dpp x}}
\newcommand{\dzyp}{\mez \frac{\dpp (\zn + \zo)}{\dpp y}}
\newcommand{\dzxn}{\mbox{$\frac{\dpp \zn}{\dpp x}$}}
\newcommand{\dzyn}{\mbox{$\frac{\dpp \zn}{\dpp y}$}}
\newcommand{\dzxo}{\mbox{$\frac{\dpp \zo}{\dpp x}$}}
\newcommand{\dzyo}{\mbox{$\frac{\dpp \zo}{\dpp y}$}}

\newcommand{\tdif}[1] {\frac{\partial #1}{\partial t}}
\newcommand{\xdif}[1] {\frac{\partial #1}{\partial x}}
\newcommand{\ydif}[1] {\frac{\partial #1}{\partial y}}
\newcommand{\zdif}[1] {\frac{\partial #1}{\partial z}}
\newcommand{\dt} {\mbox{$\Delta t$}}
\newcommand{\dthalf} {\mbox{$\frac{\Delta t}{2}$}}
\newcommand{\dtt} {\mbox{$\frac{\Delta t}{2}$}}
\newcommand{\dx} {\mbox{$\Delta x$}}
\newcommand{\dy} {\mbox{$\Delta y$}}

\newcommand{\beq} {\begin{equation}}
\newcommand{\eeq} {\end{equation}}
\newcommand{\beqa} {\begin{eqnarray}}
\newcommand{\eeqa} {\end{eqnarray}}

\newcommand{\olds} {\mbox{$\scriptstyle (0)$}}
\newcommand{\news} {\mbox{$\scriptstyle (1)$}}
\newcommand{\meds} {\mbox{$\scriptscriptstyle (\frac{1}{2})$}}
\newcommand{\half} {\mbox{$\scriptstyle \frac{1}{2}$}}

\newcommand{\nsz} {\normalsize}
\newcommand{\uold} {\mbox{$U^{\olds}$}}
\newcommand{\vold} {\mbox{$V^{\olds}$}}
\newcommand{\unew} {\mbox{$U^{\news}$}}
\newcommand{\vnew} {\mbox{$V^{\news}$}}
\newcommand{\zold} {\zeta^{(0)}}
\newcommand{\znew} {\zeta^{(1)}}
\newcommand{\resr} {{\cal R}}
\newcommand{\drho} {\frac{1}{\rho_{0}}}
\newcommand{\fracs}[2] {\mbox{$\frac{#1}{#2}$}}
%\newcommand{\deltat} {\mbox{$\tilde{\delta}$}}
%\newcommand{\gammat} {\mbox{$\tilde{\gamma}$}}
\newcommand{\ffxx} {\tilde{f_x}}
\newcommand{\ffyy} {\tilde{f_y}}

\newcommand{\uv} {{\bf U}}
\newcommand{\uvold} {{\bf U^{(0)}}}
\newcommand{\uvnew} {{\bf U^{(1)}}}
\newcommand{\af} {\alpha_{f}}
\newcommand{\ac} {\alpha_{c}}
\newcommand{\am} {\alpha_{m}}
\newcommand{\duv} {\Delta {\bf U}}
\newcommand{\dzeta} {\Delta \zeta}
\newcommand{\iv} {{\bf I}}
\newcommand{\ivh} {\hat{\bf I}}
\newcommand{\fv} {{\bf F}}
\newcommand{\uvh} {\hat{\bf U}}


%%%%%%%%%%%%%%%%%%%%%%%%%%%%%%%%%%%%%%%%%%%%%%%%%%%%%%%%%%
%%%%%%% hyphenation %%%%%%%%%%%%%%%%%%%%%%%%%%%%%%%%%%%%%%
%%%%%%%%%%%%%%%%%%%%%%%%%%%%%%%%%%%%%%%%%%%%%%%%%%%%%%%%%%


\section{Equations and Boundary Conditions}

The equations used in the model are the well known vertically integrated
shallow water equations in their formulation with water levels and
transports.

\beq \label{ubar}
\tdif{U} + gH \xdif{\zeta} + RU + X = 0
\eeq
\beq
\tdif{V} + gH \ydif{\zeta} + RV + Y = 0
\eeq
\beq \label{zcon}
\tdif{\zeta} + \xdif{U} + \ydif{V} = 0
\eeq
where $\zeta$ is the water level, $u,v$ the velocities in $x$ and $y$
direction,
$U,V$ the vertical integrated velocities (total  or barotropic
transports)
\[
 U = \int_{-h}^{\zeta} u \: dz \; \hspace{1.cm}
 V = \int_{-h}^{\zeta} v \: dz \;
\]
$g$ the gravitational acceleration, $H=h+\zeta$ the total water
depth, $h$ the undisturbed water depth,
$t$ the time and $R$ the friction coefficient. The terms $X,Y$ contain
all other terms that may be added to the equations like the wind stress or
the nonlinear terms and that need not be treated implicitly in the
time discretization.
following treatment.

The friction coefficient has been expressed as
\begin{equation}
	R = \frac{g \sqrt{u^{2}+v^{2}}}{C^{2} H}
\end{equation}
with $C$ the Chezy coefficient. The Chezy term is itself not retained
constant but varies with the water depth as
\begin{equation}
	C = k_{s} H^{1/6}
\end{equation}
where $k_{s}$ is the Strickler coefficient.

In this version of the model the Coriolis term, the turbulent friction term
and the nonlinear advective terms have not been implemented.

At open boundaries the water levels are prescribed. At closed boundaries
the normal velocity component is set to zero whereas the tangential velocity
is a free parameter. This corresponds to a full slip condition.


%%%%%%%%%%%%%%%%%%%%%%%%%%%%%%%%%%%%%%%%%%%%%%%%%%%%%%%%%%%%%%%%%%%%%%%%%
%%%%%%%%%%%%%%%%%%%%%%%%%%%%%%%%%%%%%%%%%%%%%%%%%%%%%%%%%%%%%%%%%%%%%%%%%
%%%%%%%%%%%%%%%%%%%%%%%%%%%%%%%%%%%%%%%%%%%%%%%%%%%%%%%%%%%%%%%%%%%%%%%%%
%%%%%%%%%%%%%%%%%%%%%%%%%%%%%%%%%%%%%%%%%%%%%%%%%%%%%%%%%%%%%%%%%%%%%%%%%
%%%%%%%%%%%%%%%%%%%%%%%%%%%%%%%%%%%%%%%%%%%%%%%%%%%%%%%%%%%%%%%%%%%%%%%%%





\section{The Model}

The model uses the semi-implicit time discretization to accomplish
the time integration. In the space the finite element method has
been used, not in its standard formulation, but using staggered finite
elements. In the following a description of the method is given.



%%%%%%%%%%%%%%%%%%%%%%%%%%%%%%%%%%%%%%%%%%%%%%%%%%%%%%%%%%%%%%%%%%%%%%%%%
%%%%%%%%%%%%%%%%%%%%%%%%%%%%%%%%%%%%%%%%%%%%%%%%%%%%%%%%%%%%%%%%%%%%%%%%%
%%%%%%%%%%%%%%%%%%%%%%%%%%%%%%%%%%%%%%%%%%%%%%%%%%%%%%%%%%%%%%%%%%%%%%%%%
%%%%%%%%%%%%%%%%%%%%%%%%%%%%%%%%%%%%%%%%%%%%%%%%%%%%%%%%%%%%%%%%%%%%%%%%%
%%%%%%%%%%%%%%%%%%%%%%%%%%%%%%%%%%%%%%%%%%%%%%%%%%%%%%%%%%%%%%%%%%%%%%%%%



\subsection{Discretization in Time - The Semi-Implicit Method}

Looking for an efficient time integration method
a semi-implicit scheme has been chosen.
The semi-implicit scheme combines the advantages of
the explicit and the implicit scheme. It is unconditionally stable for any
time step $\dt$ chosen and allows the two momentum equations to be
solved explicitly without solving a linear system. 

The only equation
that has to be solved implicitly is the continuity equation. Compared
to a fully implicit solution of the shallow water equations the dimensions
of the matrix are reduced to one third. Since the solution of a linear
system is roughly proportional to the cube of the dimension of the system
the saving in computing time is approximately a factor of 30.

It has to be pointed out that it is important not to be limited with the time
step by the CFL criterion for the speed of the external gravity waves
\[
        \dt < \frac{\dx}{\sqrt{gH}}
\]
where $\dx$ is the minimum distance between the nodes in an element.
With the discretization described below in most parts of the lagoon
we have $\dx \approx$ 500m and $H \approx$ 1m, so $\dt \approx 200$ sec.
But the limitation of the time step is determined by the worst case.
For example, for $\dx = 100$ m and $H = 40$ m
the time step criterion would be $\dt < 5$ sec, a
prohibitive small value.

The equations (1)-(3) are discretized as follows
\begin{equation}
\label{zn}
\frac{\zn-\zo}{\dt}
                        + \mez \frac{\dpp (\un + \uo)}{\dpp x}
                        + \mez \frac{\dpp (\vn + \vo)}{\dpp y} = 0
\end{equation}
\begin{equation}
\frac{\un-\uo}{\dt} + gH \dzxp + R \un + X = 0
\end{equation}
\begin{equation}
\frac{\vn-\vo}{\dt} + gH \dzyp + R \vn + Y = 0
\end{equation}

With this time discretization the friction term has been formulated
fully implicit, $X,Y$ fully explicit and all the other terms
have been centered in time. The reason for the implicit treatment
of the friction term is to avoid a sign inversion in the term when
the friction parameter gets too high. An example of this behavior is
given in Backhaus \cite{Backhaus83}.

If the two momentum equations are solved for the unknowns $\un$ and $\vn$
we have
\begin{equation}
\label{un}
\un = \frac{1}{1+\dt R} \paren{ \uo - \dt gH \dzxp - \dt X }
\end{equation}
\begin{equation}
\label{vn}
\vn = \frac{1}{1+\dt R} \paren{ \vo - \dt gH \dzyp - \dt Y }
\end{equation}

If $\zn$ were known, the solution for
$\un$ and $\vn$ could directly be given. To find $\zn$ we insert
(\ref{un}) and (\ref{vn}) in (\ref{zn}). After some transformations
(\ref{zn}) reads
\begin{eqnarray} \label{zsys}
        \zn
    & - &
        (\dt/2)^{2} \frac{g}{1+\dt R}         \nonumber
	\paren{ \frac{\dpp}{\dpp x}(H \dzxn) + \frac{\dpp}{\dpp y}(H \dzyn) } \\
    & = &
        \zo + (\dt/2)^{2} \frac{g}{1+\dt R}
	\paren{ \frac{\dpp}{\dpp x}(H \dzxo) + \frac{\dpp}{\dpp y}(H \dzyo) } \\
    & - & (\dt/2) \paren{ \frac{2+\dt R}{1+\dt R} }
        \paren{                               \nonumber
          \frac{\dpp \uo}{\dpp x}
        + \frac{\dpp \vo}{\dpp y}
        } \\
    & + & \frac{\dt^{2}}{2(1+\dt R)}            \nonumber
                \paren{ \frac{\dpp X}{\dpp x} + \frac{\dpp Y}{\dpp y} }
\end{eqnarray}

The terms on the left hand side contain the unknown $\zn$, the right hand
contains only known values of the old time level. If the spatial derivatives
are now expressed by the finite element method a linear system with the unknown
$\zn$ is obtained and can be solved by standard methods. Once the solution
for $\zn$ is obtained it can be substituted into (\ref{un}) and (\ref{vn})
and these two equations can be solved explicitly. In this way all unknowns
of the new time step have been found.

Note that the variable $H$ also contains the water level through
$H=h+\zz$. In order to avoid the equations to become nonlinear $\zz$
is evaluated at the old time level so $H=h+\zo$ and $H$ is a known quantity.


%%%%%%%%%%%%%%%%%%%%%%%%%%%%%%%%%%%%%%%%%%%%%%%%%%%%%%%%%%%%%%%%%%%%%%%%%
%%%%%%%%%%%%%%%%%%%%%%%%%%%%%%%%%%%%%%%%%%%%%%%%%%%%%%%%%%%%%%%%%%%%%%%%%
%%%%%%%%%%%%%%%%%%%%%%%%%%%%%%%%%%%%%%%%%%%%%%%%%%%%%%%%%%%%%%%%%%%%%%%%%
%%%%%%%%%%%%%%%%%%%%%%%%%%%%%%%%%%%%%%%%%%%%%%%%%%%%%%%%%%%%%%%%%%%%%%%%%
%%%%%%%%%%%%%%%%%%%%%%%%%%%%%%%%%%%%%%%%%%%%%%%%%%%%%%%%%%%%%%%%%%%%%%%%%




%%%%%%%%%%%%%%%%%%%%%%%%%%%%%%%%%%%%%%%%%%%%%%%%%%%%%%%%%%%%%%%%%%%%%%%%%
%%%%%%%%%%%%%%%%%%%%%%%%%%%%%%%%%%%%%%%%%%%%%%%%%%%%%%%%%%%%%%%%%%%%%%%%%
%%%%%%%%%%%%%%%%%%%%%%%%%%%%%%%%%%%%%%%%%%%%%%%%%%%%%%%%%%%%%%%%%%%%%%%%%
%%%%%%%%%%%%%%%%%%%%%%%%%%%%%%%%%%%%%%%%%%%%%%%%%%%%%%%%%%%%%%%%%%%%%%%%%
%%%%%%%%%%%%%%%%%%%%%%%%%%%%%%%%%%%%%%%%%%%%%%%%%%%%%%%%%%%%%%%%%%%%%%%%%



\subsection{Discretization in Space - The Finite Element Method}


While the time discretization has been explained above, the discretization
in space has still to be carried out. This is done 
using staggered finite elements. 
With the semi-implicit method described above
it is shown below that using linear triangular elements
for all unknowns 
will not be mass conserving. Furthermore the resulting model
will have propagation properties that introduce high numeric damping
in the solution of the equations.

For these reasons a quite new approach has been adopted here. The water
levels and the velocities (transports) are described by using form
functions of different order, being the standard linear form functions
for the water levels but stepwise constant form functions for the
transports. This will result in a grid that resembles more a staggered
grid in finite difference discretizations.

\subsubsection{Formalism}

Let $u$ be an approximate solution of a linear differential
equation $L$. We expand $u$ with the help of basis functions $\phi_{m}$
as
\begin{equation}
\label{exp}
	u=\phi_{m} u_{m} \mbox{\hspace{1cm}} m=1,K
\end{equation}
where $u_{m}$ is the coefficient of the function $\phi_{m}$ and $K$
is the order of the approximation.
In case of linear finite
elements it will just be the number of nodes of the grid used to
discretize the domain.

To find the values $u_{m}$ we try to minimize the residual
that arises when $u$ is introduced into $L$ multiplying the equation $L$
by some weighting functions $\Psi_{n}$ and
integrating over the whole domain leading to
\begin{equation}
\label{int}
\int_{\Omega} \psi_{n} L(u) \: d\Omega \; = 
\int_{\Omega} \psi_{n} L(\phi_{m} u_{m}) \: d\Omega \;
= u_{m} \int_{\Omega} \psi_{n} L(\phi_{m}) \: d\Omega \;
\end{equation}

If the integral is identified with the elements of a matrix $a_{nm}$
we can write (\ref{int}) also as a linear system
\begin{equation} \label{sys}
	a_{nm}u_{m} = 0 \mbox{\hspace{1cm}} n=1,K \hspace{0.5cm} m=1,K
\end{equation}

Once the basis and weighting functions have been specified the system
may be set up and (\ref{sys}) may be solved for the unknowns $u_{m}$.






\subsubsection{Staggered Finite Elements}

For decades finite elements have been used in fluid mechanics in
a standardized manner.
The form functions $\phi_{m}$ were chosen as continuous piecewise linear
functions allowing a subdivision of the whole area of interest into small
triangular elements specifying the coefficients $u_{m}$ at the vertices
(called nodes)
of the triangles. The functions $\phi_{m}$ are 1 at node
$m$ and 0 at all other nodes and thus different from 0 only in the
triangles containing the node $m$.
An example is given in the upper left part of Fig. 1a
where the form function for node $i$ is shown. The full circle indicates
the node where the function $\phi_{i}$ take the value
1 and the hollow circles where they are 0.


\begin{figure}
\vspace{5.cm}
\caption{a) form functions in domain \hspace{1.cm} b) domain of
influence of node $i$}
%\end{figure}

\begin{picture}(300,10)

\put(10,30){
\begin{picture}(1,1)
\put(80,90){\line(0,1){30}}
\put(80,90){\line(3,1){30}}
\put(80,90){\line(2,-3){20}}
\put(80,90){\line(-2,-3){20}}
\put(80,90){\line(-3,1){30}}
\put(60,60){\line(1,0){40}}
\put(60,60){\line(1,-2){20}}
\put(60,60){\line(-2,-3){20}}
\put(60,60){\line(-4,1){40}}
\put(60,60){\line(-1,4){10}}
\put(50,100){\line(-1,-1){30}}
\put(50,100){\line(-3,1){30}}
\put(50,100){\line(-1,4){10}}
\put(50,100){\line(3,2){30}}
\put(80,120){\line(-2,1){40}}
\put(80,120){\line(1,3){10}}
\put(80,120){\line(3,-2){30}}
\put(110,100){\line(-2,5){20}}
\put(110,100){\line(1,1){30}}
\put(110,100){\line(1,-1){30}}
\put(110,100){\line(-1,-4){10}}
\put(100,60){\line(4,1){40}}
\put(100,60){\line(3,-4){30}}
\put(100,60){\line(-1,-2){20}}
\put(130,20){\line(1,5){10}}
\put(130,20){\line(-1,0){50}}
\put(140,130){\line(0,-1){60}}
\put(140,130){\line(-5,2){50}}
\put(40,140){\line(5,1){50}}
\put(40,140){\line(-2,-3){20}}
\put(20,110){\line(0,-4){40}}
\put(40,30){\line(-1,2){20}}
\put(40,30){\line(4,-1){40}}
\thicklines
\put(50,120){\line(1,-1){30}}
\put(50,120){\line(1,0){30}}
\put(50,120){\line(-1,2){10}}
\put(50,120){\line(-3,-1){30}}
\put(50,120){\line(-3,-5){30}}
\put(50,120){\line(1,-6){10}}
\put(50,120){\line(0,-1){20}}
\put(100,80){\line(-1,-2){20}}
\put(100,80){\line(3,-4){30}}
\put(100,80){\line(0,-2){20}}
\put(80,40){\line(0,-2){20}}
\put(80,40){\line(1,0){50}}
\put(130,40){\line(0,-2){20}}
\thinlines
\put(100,30){$n$}
\put(45,88){$i$}
\put(5,90){$\phi_{i}$}
\put(100,10){$\psi_{n}$}
%\put(80,90){\circle*{5}}
%\put(60,60){\circle*{5}}
\put(50,100){\circle*{5}}
%\put(80,120){\circle*{5}}
%\put(110,100){\circle*{5}}
%\put(100,60){\circle*{5}}
\put(100,60){\circle*{5}}
\put(80,20){\circle*{5}}
\put(130,20){\circle*{5}}
%\put(50,100){\circle{7}}
\put(80,90){\circle{7}}
\put(80,120){\circle{7}}
\put(60,60){\circle{7}}
\put(20,70){\circle{7}}
\put(20,110){\circle{7}}
\put(40,140){\circle{7}}
\end{picture}}
%\end{center}



%\put(250,30){
\put(200,30){
\begin{picture}(1,1)
\put(80,90){\line(0,1){30}}
\put(80,90){\line(3,1){30}}
\put(80,90){\line(2,-3){20}}
\put(80,90){\line(-2,-3){20}}
\put(80,90){\line(-3,1){30}}
\put(60,60){\line(1,0){40}}
\put(60,60){\line(1,-2){20}}
\put(60,60){\line(-2,-3){20}}
\put(60,60){\line(-4,1){40}}
\put(60,60){\line(-1,4){10}}
\put(50,100){\line(-1,-1){30}}
\put(50,100){\line(-3,1){30}}
\put(50,100){\line(-1,4){10}}
\put(50,100){\line(3,2){30}}
\put(80,120){\line(-2,1){40}}
\put(80,120){\line(1,3){10}}
\put(80,120){\line(3,-2){30}}
\put(110,100){\line(-2,5){20}}
\put(110,100){\line(1,1){30}}
\put(110,100){\line(1,-1){30}}
\put(110,100){\line(-1,-4){10}}
\put(100,60){\line(4,1){40}}
\put(100,60){\line(3,-4){30}}
\put(100,60){\line(-1,-2){20}}
\put(130,20){\line(1,5){10}}
\put(130,20){\line(-1,0){50}}
\put(140,130){\line(0,-1){60}}
\put(140,130){\line(-5,2){50}}
\put(40,140){\line(5,1){50}}
\put(40,140){\line(-2,-3){20}}
\put(20,110){\line(0,-4){40}}
\put(40,30){\line(-1,2){20}}
\put(40,30){\line(4,-1){40}}
\put(85,95){$i$}
\put(53,107){$j$}
\put(80,90){\circle*{5}}
\put(60,60){\circle*{5}}
\put(50,100){\circle*{5}}
\put(80,120){\circle*{5}}
\put(110,100){\circle*{5}}
\put(100,60){\circle*{5}}
\put(50,100){\circle{7}}
\put(80,90){\circle{7}}
\put(80,120){\circle{7}}
\put(60,60){\circle{7}}
\put(20,70){\circle{7}}
\put(20,110){\circle{7}}
\put(40,140){\circle{7}}
\end{picture}}

\end{picture}


\end{figure}


The contributions $a_{nm}$ to the system matrix
are therefore different from 0 only in
elements containing node $m$ and the evaluation of the matrix elements
can be performed on an element basis where all coefficients and unknowns
are linear functions of $x$ and $y$.

This approach is straightforward but not very satisfying with the
semi-implicit time stepping scheme for reasons explained below.
Therefore
an other way has been followed in the present formulation. The fluid domain
is still divided in triangles and the water levels are still defined
at the nodes of the grid
and represented by piecewise linear interpolating functions
in the internal of each element, i.e.
\[
        \zeta = \zeta_{m} \phi_{m} \hspace{1cm} m=1,K
\]
However, the transports are now
expanded, over each triangle, with piecewise constant
(non continuous) form functions $\psi_{n}$ over the whole domain. We therefore
write
\[
        U = U_{n} \psi_{n} \hspace{1cm} n=1,J
\]
where $n$ is now running over all
triangles and $J$ is the total number of triangles.
An example of $\psi_{n}$ is given in the lower right part of Fig. 1a.
Note that the form function is constant 1 over the whole element,
but outside the element identically 0. Thus it is discontinuous
at the element borders.

Since we may
identify the center of gravity of the triangle with the point where
the transports $U_{n}$ are defined (contrary to the water levels
$\zeta_{m}$ which are defined on the vertices of the triangles), the
resulting grid may be seen as a staggered grid where the unknowns
are defined on different locations. This kind of grid is usually used
with the finite difference method. With the form functions used here
the grid of the finite element model resembles
very much an Arakawa B-grid that defines the water levels on the center
and the velocities on the four vertices of a square.

Staggered finite elements have been first introduced into
fluid mechanics by Schoenstadt \cite{Schoenstadt80}. 
He showed that the un-staggered
finite element formulation of the shallow water equations has very
poor geostrophic adjustment properties. Williams 
\cite{Williams81a, Williams81b}
proposed a similar algorithm, the one
actually used in this paper, introducing constant form functions for the
velocities. He showed the excellent propagation and geostrophic
adjustment properties of this scheme.


\subsubsection{The Practical Realization}

The integration of the partial differential equation is now performed by
using the subdivision of the domain in elements (triangles). The
water levels $\zeta$ are expanded in piecewise linear functions
$\phi_{m}, \; m=1,K$ and
the transports are expanded in piecewise constant functions
$\psi_{n}, \; n=1,J$ where $K$ and $J$ are the total number of nodes
and elements respectively.

As weighting functions we use $\psi_{n}$ for the momentum equations
and $\phi_{m}$ for the continuity equation. In this way there will
be $K$ equations for the unknowns $\zeta$ (one for each node) and
$J$ equations for the transports (one for each element).

In all cases the consistent mass matrix has been substituted with
the their lumped equivalent. This was mainly done
to avoid solving a linear system in the case of the momentum equations.
But it was of use also in the solution of the continuity equation
because the amount of mass relative to 
one node does not depend on the surrounding
nodes. This was important especially for the flood and dry mechanism
in order to conserve mass.


\subsubsection{Finite Element Equations}

If equations (\ref{un},\ref{vn},\ref{zsys}) are multiplied with their
weighting functions and integrated over an element we can write down
the finite element equations. But the solution of the water levels does
actually not use the continuity equation in the form (\ref{zsys}), but
a slightly different formulation. Starting from equation (\ref{zn}),
multiplied by the weighting function $\Phi_{M}$ and integrated over one
element yields


\[
          \int_{\Omega} \Phi_{N} (\zn-\zo) \: d\Omega \;
+ (\dthalf) \int_{\Omega} 
	  \left( 
	   \Phi_{N} \frac{\dpp (\un + \uo)}{\dpp x} 
+          \Phi_{N} \frac{\dpp (\vn + \vo)}{\dpp y} 
          \right)
	  \: d\Omega \;
= 0
\]
If we integrate by parts the last two integrals we obtain
\[
          \int_{\Omega} \Phi_{N} (\zn-\zo) \: d\Omega \;
- (\dthalf) \int_{\Omega} 
	  \left( 
	    \frac{\dpp \Phi_{N}}{\dpp x} (\un+\uo) 
+           \frac{\dpp \Phi_{N}}{\dpp y} (\vn+\vo)
	  \right)
	  \: d\Omega \;
= 0
\]
plus two line integrals, not shown, over the boundary of each element
that specify the normal flux over the three element
sides. In the interior of the domain,
once all contributions of all elements have
been summed, these terms cancel at every node,
leaving only the contribution of the
line integral on the boundary of the domain. There, however, the
boundary condition to impose is exactly no normal flux over
material boundaries. Thus, the contribution of these line integrals
is zero.

If now the expressions for $\un,\vn$ are introduced, we obtain a system
with again only the water levels as unknowns
\beqa
\int_{\Omega} \Phi_{N} \zn \: d\Omega \;
 & + & (\dt/2)^{2} \alpha g 
\int_{\Omega} H ( \xdif{\Phi_{N}} \xdif{\zn}  \nonumber
 + \ydif{\Phi_{N}} \ydif{\zn} ) \: d\Omega \; \\
 & = &
\int_{\Omega} \Phi_{N} \zo \: d\Omega \;	\nonumber
+ (\dt/2)^{2} \alpha g 
\int_{\Omega} H ( \xdif{\Phi_{N}} \xdif{\zo} 
 + \ydif{\Phi_{N}} \ydif{\zo} ) \: d\Omega \;  \\
 & + &
 (\dt/2)(1+\alpha) \int_{\Omega}  
  ( \xdif{\Phi_{N}} \uo + \ydif{\Phi_{N}} \vo ) \: d\Omega \; \\
 & - & (\dt^{2}/2) \alpha \nonumber
\int_{\Omega} ( \xdif{\Phi_{N}} X + \ydif{\Phi_{N}} Y ) \: d\Omega \; 
\eeqa
Here we have introduced the symbol $\alpha$ as a shortcut for
\[
\alpha = \frac{1}{1+\dt R}
\]
The variables and unknowns may now be expanded with their basis
functions and the complete system may be set up.


%%%%%%%%%%%%%%%%%%%%%%%%%%%%%%%%%%%%%%%%%%%%%%%%%%%%%%%%%%%%%%%%%%%%%%%%%
%%%%%%%%%%%%%%%%%%%%%%%%%%%%%%%%%%%%%%%%%%%%%%%%%%%%%%%%%%%%%%%%%%%%%%%%%
%%%%%%%%%%%%%%%%%%%%%%%%%%%%%%%%%%%%%%%%%%%%%%%%%%%%%%%%%%%%%%%%%%%%%%%%%
%%%%%%%%%%%%%%%%%%%%%%%%%%%%%%%%%%%%%%%%%%%%%%%%%%%%%%%%%%%%%%%%%%%%%%%%%
%%%%%%%%%%%%%%%%%%%%%%%%%%%%%%%%%%%%%%%%%%%%%%%%%%%%%%%%%%%%%%%%%%%%%%%%%

\subsection{Mass Conservation}

It should be pointed out that only through the use of this staggered grid
the semi-implicit time discretization may be implemented in a feasible
manner. If the Galerkin method is applied
 in a naive way to the resulting equation
(\ref{zsys}) (introducing the linear form functions for transports
and water levels and setting up the system matrix),
the model is not mass conserving.
This may be seen in the following way (see Fig. 1b for reference).
In the computation of the water level at
node $i$, only $\zeta$ and transport values
belonging to triangles that contain node $i$ enter the computation
(full circles in Fig. 1b).
But when, in a second step, the barotropic transports
of node $j$ are computed, water levels of nodes that lie further apart
from the original node $i$ are used
(hollow circles in Fig. 1b).
These water levels have not been included in
the computation of $\zeta_i$, the water level at node $i$.
So the computed transports are actually different
from the transports inserted formally in the continuity equation.
The continuity equation is therefore not satisfied.

These contributions of nodes lying further apart could in principle
be accounted for. In this case
not only the triangles
$\Omega_{i}$ around node $i$ but also all the triangles that have
nodes in common with the triangles $\Omega_{i}$ would give
contributions to node $i$, namely all nodes and elements shown
in Fig. 1b.
The result would be
an increase of the bandwidth of the matrix for the $\zeta$ computation
disadvantageous in terms of memory and time requirements.

Using instead the approach of the staggered finite elements, actually
only the water levels of elements around node $i$ are needed for
the computation of the transports in the triangles $\Omega_i$.
In this case the model satisfies the
continuity equation and is perfectly mass conserving.



\subsection{Inter-tidal Flats}

Part of a basin may consist of areas that are
flooded during high tides and emerge as islands at ebb tide. These
inter-tidal flats are quite difficult to handle numerically because
the elements that represent these areas are neither
islands nor water elements. The boundary line defining their
contours is wandering during the evolution
of time and a mathematical model must reproduce this features.

For reasons of computer time savings a simplified algorithm has been chosen
to represent the inter-tidal flats. When the water level in at least
one of the three nodes of an element falls below a minimum value (5 cm)
the element is considered an island and is taken out of the system.
It will be reintroduced only when in all three
nodes the water level is again higher then the minimum value.
Because in dry nodes no water level is computed anymore, an estimate
of the water level has to be given with some sort of extrapolation mechanism
using the water nodes nearby.

This algorithm has the advantage that it is very easy to
implement and very fast. The dynamical features close to the
inter-tidal flats are of course not well reproduced but the
behavior of the method for the rest of the lagoon
gave satisfactory results.

In any case, since the method stores the water levels of the
last time step, before the element is switched off, introducing the
element in a later moment with the same water levels conserves the
mass balance. This method showed a much better performance
than the one where the new elements were introduced with the water
levels taken from the extrapolation of the surrounding nodes.

%%%%%%%%%%%%%%%%%%%%%%%%%%%%%%%%%%%%%%%%%%%%%%%%%%%%%%%%%%%%%%%%%%%%%%%%%
%%%%%%%%%%%%%%%%%%%%%%%%%%%%%%%%%%%%%%%%%%%%%%%%%%%%%%%%%%%%%%%%%%%%%%%%%
%%%%%%%%%%%%%%%%%%%%%%%%%%%%%%%%%%%%%%%%%%%%%%%%%%%%%%%%%%%%%%%%%%%%%%%%%
%%%%%%%%%%%%%%%%%%%%%%%%%%%%%%%%%%%%%%%%%%%%%%%%%%%%%%%%%%%%%%%%%%%%%%%%%
%%%%%%%%%%%%%%%%%%%%%%%%%%%%%%%%%%%%%%%%%%%%%%%%%%%%%%%%%%%%%%%%%%%%%%%%%



\chapter{Pre-Processing}


The pre-processing routine |vp| is used to generate an
optimized version of the file that describes the basin
where the main program is to be run. In the following a
short introduction in using this program is given.

\section{The pre-processing routine {\tt vp}}

The main routine |hp| reads the basin file generated by
the pre-processing routine |vp| and uses it as the description
of the domain where the hydrodynamic equations have to be
solved.

The program |vp| is started by typing |vp| on the command line.
From this point on the program is interactive, asking you about
the basin file name and other options. Please follow the online
instructions.

The routine |vp| reads a file of type GRD. This type of file
can be generated and manipulated by the program |grid| which
is not described here. In short, the file GRD consists of
nodes and elements that describe the geometrical layout
of the basin. Moreover, the elements have a type and a depth.

The depth is needed by the main program |hp| to run the model.
The type of the element is used by |hp| to determine
the friction parameter on the bottom, since this parameter
may be assigned differently, depending on the various situations
of the bottom roughness.

This file GRD is read by |vp| and transformed into an
unformatted file BAS. It is this file that is then read
by the main routine |hp|. Therefore, if the name of the
basin is |lagoon|, then the file GRD is called |lagoon.grd|
and the output of the pre-processing routine |vp| is
called |lagoon.bas|.

The program |vp| normally uses the depths assigned to the
elements in the file GRD to determine the depth of the
finite elements to use in the program |hp|. In the case
that these depth values are not complete, and that all nodes
have depths assigned in the GRD file, the nodal values of the
depths are used and interpolated onto the elements. However,
if also these nodal depth values are incomplete or are missing
altogether, the program terminates with an error.

\section{Optimization of the bandwidth}

The main task of routine |vp| is the optimization of the 
internal numbering of the nodes and elements.
Re-numbering the elements is just a mere convenience. When
assembling the system matrix the contribution of
one element after the other has to be added to the system matrix.
If the elements are numbered in terms of lowest node numbers,
then the access of the nodal pointers is more regular in 
computer memory and paging is more likely to be inhibited.

However, re-numbering the nodes is absolutely necessary.
The system matrix to be solved is of band-matrix type.
I.e., non-zero entries are all concentrated along the
main diagonal in a more or less narrow band. The larger this
band is, the larger the amount of cpu time spent to
solve the system. The time to solve a band matrix
is of order $n \cdot m^2$, where $n$ is the size of the
matrix and $m$ is the bandwidth. Note that $m$ is normally
much smaller than $n$.

If the nodes are left with the original numbering, it is very likely
that the bandwidth is very high, unless the nodes in the
file GRD are by chance already optimized. Since the bandwidth $m$
is entering the above formula quadratically, the amount
of time spent solving the matrix will be prohibitive.
E.g., halving the bandwidth will speed up computations by
a factor of 4.

The bandwidth is equal to the maximum difference of node numbers
in one element. It is therefore important to re-number the
nodes in order to minimize this number. However, there exist
only heuristic algorithms for the minimization of this number.

The two main algorithms used in the routine |vp| are
the Cuthill McGee algorithm and the algorithm of Rosen. The first
one, starting from one node, tries to number all neighbors in
a greedy way, optimizing only this step. From the points
numbered in this step, the next neighbors are numbered.

This procedure is tried from more than one node, possibly
from all boundary nodes. The numbering resulting from this
algorithm is normally very good and needs only slight
enhancements to be optimum.

Once all nodes are numbered, the Rosen algorithm tries to
exchange these node numbers, where the highest difference
can be found. This normally gives only a slight improvement
of the bandwidth. It has been seen, however, that, if the
node numbers coming out from the Cuthill McGee algorithm
are reversed, before feeding them into the Rosen algorithm, 
the results tend to be slightly better. This step is also
performed by the program.

All these steps are performed by the program without
intervention by the operator, if the automatic optimization
of bandwidth is chosen in the program |vp|. The choices
are to not perform the bandwidth optimization at all
(GRD file has already optimized node numbering), perform
it automatically or perform it manually. It is suggested
to always perform automatic optimization of the bandwidth.
This choice will lead to a nearly optimum numbering of the
nodes and will be by all means good results.

If, however, you decide to do a manual optimization, please
follow the online instructions in the program.

\section{Internal and external node numbering}

As explained above, the elements and nodes of the basin are re-numbered 
in order to optimize the bandwidth of the system matrix and so
the execution speed of the program. 

However, this re-numbering of the node and elements is transparent
to the user. The program keeps pointers from the original numbering
(external numbers) to the optimized numbering (internal numbers).
The user has to deal only with external numbers, even if the 
program uses internally the other number system.

Moreover, the internal numbers are generated consecutively.
Therefore, if there are a total of 4000 nodes in the system, the internal
nodes run from 1 to 4000. The external node numbers,
on the other side, can be anything the user likes. They just must be
unique. This allows for insertion and deletion of nodes without
having to re-number over and over again the basin.

The nodes that have to be specified in the input parameter file
use again external numbers. In this way, changing the structure of
the basin does not at all change the node and element numbers in the
input parameter file. Except in the case, where modifications
actually touch nodes and elements that are specified in the 
parameter file.



\chapter{The Model}

In the following an overview is given on running the model
\shy{}. The model needs a parameter input file that is read 
on standard input. Moreover, it needs some external files that
are specified in this parameter input file. The model produces
several external files with the results of the simulation. Again,
the name of this files can be influenced by the parameter input file


\section{The Parameter Input File}

The model reads one input file that determines the behavior of the
simulation. All possible parameters can and must be set in this file.
If other data files are to be read, here is the place where to specify
them.

The model reads this parameter file from standard input. Thus, if
the model binary is called |hp| and the parameter file |param.str|, 
then the following line starts the simulation
\begin{verbatim}
	hp < param.str
\end{verbatim}
and runs the model.

\subsection{The General Structure of the Parameter Input File}

The input parameter file is the file that guides program
performance. It contains all necessary information for the main routine
to execute the model. Nearly all parameters that can
be given have a default value which is used when the parameter
is not listed in the file. Only some time parameters are compulsory
and must be present in the file.

The format of the file looks very like a namelist format, but is
not dependent on the compiler used. Values of parameters are given
in the form :  
|name = value|  or  |name = 'text'|.  If |name|
is an array the following format is used : 
\begin{verbatim}
          name = value1 , value2, ... valueN
\end{verbatim}
The list can continue on the following lines. Blanks before and after
the equal sign are ignored. More then one parameter can be present
on one line. As separator blank, tab and comma can be used.

Parameters, arrays and data must be given in between certain sections.
A section starts with the character {\tt \$} followed by a keyword and
ends with {\tt \$end}. The {\tt \$keyword} and {\tt \$end} must not
contain any blank characters and must be the first non blank characters
in the line. Other characters following the keyword on the same line
separated by a valid separator are ignored.

Several sections of data may be present in the input parameter file.
Further ahead all sections are presented and the possible
parameters that can be specified are explained. The sequence in
which the sections appear is of no importance. However, the first 
section must always be section |\$title|, the section that
determines the name of simulation and the basin file to use and
gives a one line description of the simulation.

Lines outside of the sections are ignored. This gives
the possibility to comment the parameter input file.

Figure \ref{fig:str_example} shows an example of a typical input
parameter file and the use of the sections and definition of
parameters.

\begin{figure}
\begin{alltt}
\input{example.str}
\end{alltt}
\caption{Example of a parameter input file ({\tt STR} file)}
\label{fig:str_example}
\end{figure}

\subsection{Typical Usage of the model}


This section explains typical usage of the model. It will show how the
model can be run doing basic 2D simulations, compute T/S, do 3D simulations,
set up the turbulence module etc. This section is conceived as
a simple HOWTO document. For the exact meaning and usage of the single
parameters, please see the section on input parameters.

\subsubsection{2D Hydrodynamic Simulation}

To run a simulation, two things are needed. The first is the description
of the basin and the numerical grid, which must be prepared beforehand
and then must be compiled in a form that the model can use. This is typically
done by the routine |vp| that, starting from a file |.grd| creates a file
|.bas|. This will be called the basin file from now on.

The second thing that is needed is a description of the simulation and
the forcings that have to be applied. This is done through a 
input parameter description file. Here we call it a |STR| file, because
historically these files always ended with an extension of |.str|. However,
any extension can be used.

\begin{figure}
\begin{alltt}
\input{basic.str}
\end{alltt}
\caption{Example of a basic parameter input file ({\tt STR} file)}
\label{fig:str_basic}
\end{figure}

A basic version of an |STR| file can be found in \ref{fig:str_basic}. In
fact, it is so basic, it really does not do anything. Here only the
compulsory parameters have been inserted. These are:

\begin{itemize}

\item An introductory section |$title| where on three lines the following information is given:

\begin{enumerate}
\item A description of the run. This can be any text that fits on one line.
\item The name of the simulation. This name is used for all files that 
the simulation produces. These files differ from each other only by 
their extension.
\item The name of the basin. This is the basin file without the extension
|.ext|.
\end{enumerate}

\item A section |$para| that contains all necessary parameters for the
simulation to be run. The only compulsory parameters are the ones that
specify the start of the simulation |itanf|, its end |itend| and its 
time step |idt|.

\end{itemize}

In order to be more helpful, some more information must be added to the
|STR| file. As an example let's have a look on \ref{fig:str_example}. Here
we have added two parameters that deal with the type of friction
to be used. |ireib| specifies the bottom friction formulation, here
through a simple quadratic bulk formula. (For the exact meaning of the
parameters, please refer to the section lateron where all parameters
are listed.) The parameter |czdef| specifies the value to use for the
bottom drag coefficient.

The lats parameter in the |$para| section is |dragco| which is the
drag coefficient to use for the wind file specified later. If n

ggugguggu

do with 


\subsection{The Single Sections of the Parameter Input File}

\subsubsection{Section {\tt \$title}}

This section must be always the first section in the parameter input file.
It contains only three lines. An example is given in 
figure \ref{fig:titleexample}.

\begin{figure}[ht]
\begin{verbatim}
$title
        free one line description of simulation
        name_of_simulation
        name_of_basin
$end
\end{verbatim}
\caption{Example of section {\tt \$title}}
\label{fig:titleexample}
\end{figure}

The first line of this section is a free one line description of
the simulation that is to be carried out. The next line contains
the name of the simulation (in this case |name_of_simulation|).
All created files will use this name in the main part of the file name
with different extensions. Therefore the hydrodynamic output file
(extension |out|) will be named |name_of_simulation.out|.
The last line gives the name of the basin file to be used. This
is the pre-processed file of the basin with extension |bas|.
In our example the basin file |name_of_basin.bas| is used.

The directory where this files are read from or written to depends
on the settings in section {\tt \$name}. Using the default
the program will read from and write to the current directory.

\subsubsection{Section {\tt \$para}}

This section defines the general behavior of the simulation,
gives various constants of parameters and determines what
output files are written. In the following the meaning of
all possible parameters is given.

Note that the only compulsory parameters in this section are 
the ones that chose the duration of the simulation and the
integration time step. All other parameters are optional.

\input{S_para_h.tex}

\subsubsection{Section {\tt \$name}}

In this sections names of directories or input files can be
given. All directories default to the current directory,
whereas all file names are empty, i.e., no input files are
given.

\input{S_name.tex}
\input{S_name_h.tex}

\subsubsection{Section {\tt \$bound}}

\input{S_bound.tex}


\subsubsection{Section {\tt \$wind}}

\input{S_wind.tex}



\subsubsection{Section {\tt \$extra}}

In this section the node numbers of so called ``extra'' points are given. 
These are points where water level and velocities are written to create
a time series that can be elaborated later. The output for these ``extra''
points consumes little memory and can be therefore written with a
much higher frequency (typically the same as the integration time step)
than the complete hydrodynamical output. The output is written
to file EXT.

The node numbers are specified in a free format on one ore more lines.
An example can be seen in figure \ref{fig:str_example}. No keywords
are expected in this section.


\subsubsection{Section {\tt \$flux}}

In this section transects are specified through which the discharge
of water is computed by the program and written to file FLX.
The transects are defined by their nodes through which they run.
All nodes in one transect must be adjacent, i.e., they must form a
continuous line in the FEM network.

The nodes of the transects are specified in free format. Between
two transects one or more 0's must be inserted. An example is given in
figure \ref{fig:fluxexample}.

\begin{figure}[ht]
\begin{verbatim}
$flux
	1001 1002 1004 0
	35 37 46 0 0 56 57 58 0
	407
	301
	435 0 89 87
$end
\end{verbatim}
\caption{Example of section {\tt \$flux}}
\label{fig:fluxexample}
\end{figure}

The example shows the definition of 5 transects. As can be seen, the 
nodes of the transects can be given on one line alone (first transect),
two transects on one line (transect 2 and 3), spread over more lines
(transect 4) and a last transect.


\chapter{Post-Processing}


There are several routines that do a post-processing of the results of the 
main routine. The most important are described in this chapter.
Note that in the model framework no program is supplied to do
time series plots. However, there are utility routines that will extract
data from the output files. These data will be written in a way
that it can be imported into a spreadsheet or any other plotting
program that does the nice plotting.


\section{Plotting of maps with {\tt plotmap}}

\subsection{The parameter input file for {\tt plotmap}}

The format of the parameter input file is the same as the one for
the main routine. Please see this section for more information
on the format of the parameter input file.

Some sections of the parameter input file are identical to the 
sections used in the main routine. For easier reference we will
repeat the possible parameters of these section here.


\subsubsection{Section {\tt \$title}}

This section must be always the first section in the parameter input file.
It contains only three lines. An example has been given already in 
figure \ref{fig:titleexample}.

The only difference respect to the {\tt \$para} section of the main routine
is the first line. Here any description of the output can be used.
It is just a way to label the parameter file.
The other two line with the name of simulation and the basin are used
to open the files needed for plotting.


\subsubsection{Section {\tt \$para}}

\input{S_para_a.tex}


\subsubsection{Section {\tt \$color}}

\input{S_color.tex}


\subsubsection{Section {\tt \$arrow}}

\input{S_arrow.tex}


\subsubsection{Section {\tt \$legend}}

\input{S_legend.tex}


\subsubsection{Section {\tt \$legvar}}

\input{S_legvar.tex}


\subsubsection{Section {\tt \$name}}

\input{S_name.tex}


\chapter{The Water Quality Module}

%\documentclass{report}
%\begin{document}
                                                                              
                                                                                                    
                                                                                                    
                                                                                                    
                                                                                                    
                                                                                                    
                                                                                                    
                                                                                                    
                                                                                                    
                                                                                                    
\newcommand{\tab}{\hspace{5mm}}

\newcommand{\Tone}{\ref{MassBalance}}
\newcommand{\Ttwoa}{\ref{FuncDesc}}
\newcommand{\Ttwob}{\ref{Paras}}
\newcommand{\Ttwoc}{\ref{Vars}}

\newcommand{\STone}{\ref{SMassBalance}}
\newcommand{\STtwoa}{\ref{SFuncDesc}}
\newcommand{\STtwob}{\ref{SParas}}
\newcommand{\STtwoc}{\ref{SVars}}




by Donata Melaku Canu, Georg Umgiesser, Cosimo Solidoro

\vspace{1cm}

The coupling between EUTRO and FEM constitute a structure which 
is meant to be a generic water quality for full eutrophication 
dynamics.
The Water Quality model is described fully in
Umgiesser et al. (2003).



\section{General Description}



The water quality model has been derived from the EUTRO module 
of WASP (released by the U.S. Environmental Protection Agency 
(EPA) (Ambrose et al., 1993) and modified. It simulates the evolution 
of nine state variables in the water column and sediment bed, 
including dissolved oxygen (DO), carbonaceous biochemical oxygen 
demand (CBOD), phytoplankton carbon and chlorophyll a (PHY), 
ammonia (NH3), nitrate (NOX), organic nitrogen (ON), organic 
phosphorus (OP), orthophosphate (OPO4) and zooplankton (ZOO). 
The interacting nine state variables can be considered as four 
interacting systems: the carbon cycle, the phosphorous cycle, 
the nitrogen cycle and the dissolved oxygen balance (Fig. ??). 
Different levels of complexity can be selected by switching the 
eight variables on and off, in order to address the specific 
topics.

The evolution of phytoplankton concentration (Reaction 1, 
Table \Tone)
is described by the anabolic and the catabolic terms, plus 
a grazing term related to zooplankton concentration (Reaction 
10, 11 and 12, Table \Ttwoa), which however is treated as a constant
in the original version. 
The anabolic term (Reaction 10, Table \Ttwoa) is related to light 
intensity, temperature and concentration of nutrients in water, 
while the catabolic term (Reaction 11, Table \Ttwoa) depends on temperature.

Phytoplankton growth is described by combining a maximum growth 
rate under optimal conditions, and a number of dimensionless 
factors, each ranging from 0 to 1, and each one referring to 
a specific environmental factor (nutrient, light availability), 
which reduces the phytoplanktonic growth insofar as environmental 
conditions are at sub-optimal levels. Phytoplankton stochiometry 
is fixed at the user-specified ratio, so that no luxury uptake 
mechanisms are considered, and the uptake of nutrients is directly 
linked to the phytoplankton growth, and described by the same 
one-step kinetic law. More specifically, the influence of inorganic 
phosphorous and nitrogen availability on phytoplankton growth/nutrients 
uptake is simulated by means of Michealis-Menten-Monod kinetics 
(Reactions 42 and 43, Table \Ttwoa). Phytoplankton uptakes nitrogen 
both in the forms of ammonia and nitrate, but ammonia is assimilated 
preferentially, as indicated in the ammonia preference relation 
(Reaction 38, Table \Ttwoa). The influence of temperature is given 
by an exponential relation (Reaction 13, Table \Ttwoa), while the 
functional forms for the limitation due to sub-optimal light 
condition can be chosen between three alternative options, namely 
the formulation proposed by Di Toro et al. (1971) and the one 
proposed by Smith (1980) (Di Toro and Smith subroutines, Reaction 
44, Table \Ttwoa) and the Steele formulation (Steele, 1962) that 
can use hourly light input values. The 
choice between different available functional forms (Ditoro, 
Smith, and Steele) is made by setting the index |LGHTSW| equal 
to 1, 2 or 3. The new version is therefore able to simulate diurnal 
variations depending on light intensity, such as night anoxia 
due to phytoplankton respiration during nighttime.

Finally, the two frequently used models for combining maximum 
growth and limiting factors, the multiplicative and the minimum 
(or Liebig's) model, are both implemented, and the user can choose 
which one to adopt (Reaction 41, Table \Ttwoa).

Nitrogen and phosphorous are then returned to the organic compartment 
(ON, OP) via phytoplankton and zooplankton respiration and death. 
After mineralization, the organic form is again converted into
the dissolved inorganic form available for phytoplankton growth. 

The DO mass balance is influenced by almost all of the processes 
going on in the system. The reaeration process acts to restore 
the thermodynamic equilibrium level, the saturation value, while 
respirations activities and mineralization of particulated and 
dissolved organic matter consume DO and, of course, photosynthetic 
activity produces it. Other terms included in the DO mass balance 
are the ones referring to redox reactions such as nitrification 
and denitrification. The reaeration rate is computed from the 
model in agreement with either the flow-induced rate or the wind-induced 
rate, whichever is larger. The wind-induced reaeration rate is 
determined as a function of wind speed, water and air temperature, 
in agreement with O'Connor (1983), while the flow-induced reaeration 
is based on the Covar method (Covar, 1976), i.e., it is calculated 
as a function of current velocity, depth and temperature.

The dynamic of a generic herbivorous zooplankton compartment 
(ZOO), meant to be representative of the pool of all the herbivorous 
zooplankton species, is followed and accordingly the subroutines 
relative to phytoplankton, organic matter, nutrients, and dissolved 
oxygen, which were influenced by such a modification. 

The grazing has been described by means of a type II functional 
relationship, as it is usually done for aquatic ecosystems. However, 
the possibility to select a type III relationship, as well as 
to maintain the original parameterisation of constant zooplankton, 
has been included.  

The zooplankton assimilates the ingested phytoplankton with an 
efficiency EFF, and the fraction not assimilated, ecologically 
representative of faecal pellets and sloppy feeding, is transferred 
to the organic matter compartments (dotted lines Fig. ??). Finally, 
zooplankton mortality is described by a first order kinetics. 
The code has been written by adopting the standard WASP nomenclature 
system, and the choice between the different available functional 
forms is performed by setting the index |IGRAZ|. A choice of 0 
(the default value) corresponds to the original EUTRO version, 
giving the user the ability to chose easily between the extended 
version or revert to the original one.


%\documentclass{report}
%\usepackage{a4}
%\usepackage{shortvrb}
%\begin{document}



\newcommand{\Otwo}{O${}_{2}$}
\newcommand{\Degree}{${}^{o}$}
\newcommand{\power}[1]{${}^{#1}$}

\newcommand{\opn}{ {} }


\newcommand{\VSPGB}{\vspace{0.2cm}}
\newcommand{\HSP}{\hspace*{1.5cm}}
\newcommand{\HHSP}{\hspace*{0.5cm}}
\newcommand{\GBox}[2]{\parbox{#1 cm}{\VSPGB#2\VSPGB}}
\newcommand{\GDBox}[1]{\GBox{5}{#1}}
\newcommand{\GEBox}[1]{\GBox{7}{#1}}
\newcommand{\GFBox}[1]{\GBox{7}{#1}}



\begin{table}\centering
\begin{tabular}{lll}
\hline


& & \\
$\frac{\partial S}{\partial t} =Q(S)$
& &
General Reactor Equation
\\
& & \\

$Q(PHY) = GPP - DPP - GRZ$
& 1 & 
Phytoplankton PHY [mg C/L]
\\

$Q(ZOO) = GZ - DZ$
& 2 &
Zooplankton ZOO [mg C/L]
\\

$Q(NH3) = N_{alg1} + ON1 - N_{alg2} - N1$
& 3 &
Ammonia NH3 [mg N/L]
\\

$Q(NOX) = N1 - NO_{alg} - NIT1$
& 4 &
Nitrate NOX [mg N/L]
\\

$Q(ON) = ON_{alg} - ON1$
& 5 &
Organic Nitrogen ON [mg N/L]
\\

$Q(OPO4) = OP_{alg1} + OP1 - OP_{alg2}$ 
& 6 &
\GBox{5}{
Inorganic Phosphorous OPO4 \\
\HHSP [mg P/L]
}
\\

$Q(OP) = OP_{alg3} - OP1$
& 7 &
Organic Phosphorous OP [mg P/L]
\\

$Q(CBOD) = C1 - OX - NIT2$
& 8 &
\GBox{5}{
Carbonaceous Biological Oxygen \\
\HHSP Demand CBOD [mg \Otwo/L]
}
\\

\GBox{6}{
$Q(DO) = DO1 + DO2 + DO3 $\\
\HSP $\opn - DO4 - N2 - OX - SOD$
}
& 9 &
Dissolved Oxygen DO [mg \Otwo/L]
\\


\hline
\end{tabular}
\caption{Mass balances}
\label{MassBalance}
\end{table}






\begin{table}\centering
\begin{tabular}{lll}
\hline


$GPP=GP1*PHY $
& 10 &
phytoplankton growth
\\

$DPP=DP1*PHY $
& 11 &
phytoplankton death 
\\

$GRZ=KGRZ*\frac{PHY}{PHY+KPZ}*ZOO$
& 12 &
grazing rate coefficient
\\

$GP1=L_{nut} *L_{light} *K1C*K1T^{(T-T_{0} )} $
& 13 &
\GDBox{
phytoplankton growth rate with nutrient and light limitation
}
\\

$DP1=RES+K1D$ 
& 14 &
\GDBox{
phytoplankton respiration and death rate
}
\\

$GZ=EFF*GRZ$ 
& 15 &
zooplankton growth rate
\\

$DZ=KDZ*ZOO$ 
& 16 &
zooplankton death rate
\\

$Z_{ineff} = (1-EFF)*GRZ $
& 17 &
grazing inefficiency on phytoplankton
\\

$Z_{sink} = Z_{ineff}+DZ $
& 18 &
sink of zooplankton
\\

$N_{alg1}= NC*DPP*(1-FON) $
& 19 &
source of ammonia from algal 
death
\\

$N_{alg2}=PN*NC*GPP $
& 20 &
sink of ammonia for algal growth
\\

$NO_{alg}= (1. - PN)*NC*GPP $
& 21 &
sink of nitrate for algal growth
\\

$ON_{alg}= NC*(DPP*FON+Z_{sink}) $
& 22 &
\GDBox{
source of organic nitrogen from phytoplankton and zooplankton death
}
\\

\GBox{5}{
$N1=KC_{nit} *KT_{nit}^{(T-T_{0})} *NH3$
\HSP $\opn *\frac{DO}{K_{nit}+DO}$
}
& 23 &
nitrification
\\

\GBox{5}{
$NIT1=KC_{denit} KT_{denit}^{(T-T_{0})}$
\HSP $\opn *NOX*\frac{K_{denit}}{K_{denit}+DO}$
}
& 24 &
denitrification
\\

$ON1=KNC_{\min } *KNT_{\min }^{(T-T_{0})} *ON$
& 25 &
mineralization of ON
\\

$OP1=KPC_{\min } *KPT_{\min } ^{(T-T_{0})} *OP$
& 26 &
mineralization of OP
\\

$OP_{alg1}=PC*DPP*(1. - FOP) $
& 27 &
\GDBox{
source of inorganic phosphorous from algal death
}
\\

$OP_{alg2}=PC*GPP $
& 28 &
\GDBox{
sink of inorganic phosphorous for algal growth
}
\\

$OP_{alg3}=PC*(DPP*FOP+Z_{sink}) $
& 29 &
\GDBox{
source of organic phosphorous from phytoplankton and zooplankton death
}
\\

\GBox{5}{
$OX=KDC*KDT^{(T-T_{0} )} $
\HSP $\opn *CBOD*\frac{DO}{KBOD+DO} $
}
& 30 &
oxidation of CBOD
\\

\hline
\end{tabular}
\caption{Functional Expression Description}
\label{FuncDesc}
\end{table}

\begin{table}\centering
\begin{tabular}{lll}
\hline

$C1=OC*(K1D*PHY+Z_{sink}) $
& 31 &
\GDBox{
source of CBOD from phytoplankton and zooplankton death
}
\\

$NIT2=\left( \frac{5}{4} *\frac{32}{14} *NIT1\right) $
& 32 &
sink of CBOD due to denitrification
\\

$DO1=KA*(O_{sat} - DO) $
& 33 &
reareation term
\\

$DO2=PN*GP1*PHY*OC $
& 34 &
\GDBox{
dissolved oxygen produced by phytoplankton using NH3
}
\\

\GBox{5}{
$DO3=(1-PN)*GP1*PHY$
\HSP $\opn * 32*\left( \frac{1}{12} +1.5*\frac{NC}{14} \right) $
}
& 35 &
growth of phytoplankton using NOX
\\

$DO4 = OC*RES*PHY$ 
& 36 &
respiration term
\\

$N2=\left( \frac{64}{14} *N1\right) $
& 37 &
oxygen consumption due to nitrification
\\

\GBox{6}{
$PN=\frac{NH3*NOX}{(KN+NH3)*(KN+NOX)}$ 
\HSP $\opn +\frac{NH3*KN}{(NH3+NOX)*(KN+NOX)} $
}
& 38 &
ammonia preference
\\

$RES=K1RC*K1RT^{(T-T_{0})} $
& 39 &
algal respiration
\\

$SOD=\frac{SOD1}{H} *SODT^{(T-T_{0})} $
& 40 &
sediment oxygen demand
\\

$L_{nut}= min(X1,X2) \, , \,  mult(X1,X2) $
& 41 &
\GDBox{
minimum or multiplicative nutrient limitation for phytoplankton growth
}
\\

$X1=\frac{NH3+NOX}{KN+NH3+NOX} $
& 42 &
\GDBox{
nitrogen limitation for phytoplankton growth
}
\\

$X2=\frac{OPO4}{\frac{KP}{FOPO4} +OPO4} $
& 43 &
\GDBox{
phosphorous limitation for phytoplankton growth
}
\\

$L_{light} =\frac{I_{0} }{I_{s} } *e^{-(KE*H)} *e^{(1-\frac{I_{0} }{I_{s}
} *e^{(-KE*H)} )} $
& 44 &
\GDBox{
light limitation for phytoplankton growth
}
\\

$KA=F(Wind,Vel,T,T_{air},H) $
& 45 &
re-areation coefficient 
\\


\hline
\end{tabular}
\addtocounter{table}{-1}
\caption{(continued) Functional Expression Description}
\label{FuncDesc1}
\end{table}













\begin{table}\centering
\begin{tabular}{ll}
\hline


$K1D=0.12$ day\power{-1}  
&
phytoplankton death rate constant
\\

$KGRZ=1.2$ day\power{-1}  
&
grazing rate constant
\\

$KPZ=0.5$ mg C/L 
&
\GFBox{
half saturation constant for phytoplankton in grazing
}
\\

$KDZ=0.168$ day\power{-1} 
&
zooplankton death rate
\\

$K1C=2.88$ day\power{-1} 
&
phytoplankton growth rate constant
\\

$K1T=1.068$ 
&
\GFBox{
phytoplankton growth rate temperature constant
}
\\

$KN=0.05$ mg N/L 
&
\GFBox{
nitrogen half saturation constant for phytoplankton growth
}
\\

$KP=0.01$ mg P/L 
&
\GFBox{
phosphorous half saturation constant for phytoplankton growth
}
\\

$KC_{nit}=0.05$ day\power{-1} 
&
nitrification rate constant
\\

$KT_{nit}=1.08$ 
&
nitrification rate temperature constant
\\

$K_{nit}=2.0$ mg \Otwo/L 
&
half saturation constant for nitrification
\\

$KC_{denit}=0.09$ day\power{-1} 
&
denitrification rate constant
\\

$KT_{denit}=1.045$ 
&
denitrification rate temperature constant
\\

$K_{denit}=0.1$ mg \Otwo/L 
&
half saturation constant for denitrification
\\

$KNC_{min}=0.075$ day\power{-1} 
&
mineralization of dissolved ON rate constant
\\

$KNT_{min}=1.08$ 
&
\GFBox{
mineralization of dissolved ON rate temperature constant
}
\\

$KDC=0.18$ day\power{-1}
&
oxidation of CBOD rate constant
\\

$KDT= 1.047$ 
&
oxidation of CBOD rate temperature constant
\\

$NC=0.115$ mg N/mg C 
&
N/C ratio
\\

$PC=0.025$ mg P/mg 
&
C P/C ratio
\\

$OC=32/12$ mg \Otwo/mg C 
&
O/C ratio
\\

$EFF=0.5$ 
&
grazing efficiency
\\

$FON=0.5$ 
&
fraction of ON from algal death
\\

$FOP=0.5$ 
&
fraction of OP from algal death
\\

$FOPO4=0.9$ 
&
fraction of dissolved inorganic phosphorous
\\

$KPC_{min}=0.0004$ day\power{-1}  
&
mineralization of dissolved OP rate constant
\\

$KPT_{min}=1.08$ 
&
\GFBox{
mineralization of dissolved OP rate temperature constant
}
\\

$KBOD=0.5$ mg \Otwo/L
&
CBOD half saturation constant for oxidation
\\

$K1RC=0.096$ day\power{-1}  
&
algal respiration rate constant
\\

$K1RT=1.068$ 
&
algal respiration rate temperature constant
\\

$I_{s}=1200000$ lux/day
&
\GFBox{
optimal value of light intensity for phytoplankton growth
}
\\

$KE=1.0$ m\power{-1}
&
light extinction coefficient
\\

\GBox{5}{
$SOD1=2.0$ mg \Otwo/L \\
\HSP day\power{-1} m 
}
&
sediment oxygen demand rate constant
\\

$SODT=1.08$ 
&
sediment oxygen demand temperature constant
\\

$T_{0}=20$ \Degree C 
&
optimal temperature value
\\


\hline
\end{tabular}
\caption{Parameters}
\label{Paras}
\end{table}





\begin{table}\centering
\begin{tabular}{lll}
\hline


$T$ 
& [\Degree C] &
water temperature
\\

$T_{air}$ 
& [\Degree C] &
air temperature
\\

$O_{sat}$  
& [mg/L] &
DO concentration value at saturation
\\

$I_{0}$ 
& [lux/day] &
incident light intensity at the surface
\\

$H$ 
& [m] &
depth
\\

$Vol$ 
& [m\power{3}] &
volume
\\

$Vel$ 
& [m/sec] &
current speed
\\

$Wind$ 
& [m/sec] &
wind speed
\\


\hline
\end{tabular}
\caption{Variables}
\label{Vars}
\end{table}




%\end{document}



\section{The coupling}


Mathematical models usually describe the coupling between ecological 
and physical process by suitable implementation of an advection/diffusion 
equation for a generic tracer, reads 

\begin{equation} \label{AdvDif}
\frac{\partial \,\Theta _{i} }{\partial \,t} \,\,+U\cdot \nabla \,\Theta
_{i} -\,w^{s}_{i} \,\frac{\partial \,\Theta _{i} }{\partial \,z} \,=\,\,K_{h}
\,\nabla _{H}^{2} \Theta _{i} \,+\,\frac{\partial \,}{\partial \,z}
\,\left[ K_{v} \,\frac{\partial \,\Theta _{i} }{\partial \,z} \right]
\,+\,F\,\left( \Theta \,,\,T,\,I\,,\,\,..\right)
\end{equation}

where $U$ is the (average components of the) velocity, 
the $\Theta_{i}$ are the tracers which compose the entire 
vector of the biological state variable $\Theta$ and 
$F$ is a source term. $T$ and $I$ indicate, respectively, 
water temperature and Irradiance level, while $w^{s}_{i}$ represent 
the downward flux rates (sinking velocity) for the tracer 
$\Theta_{i}$, 
and $K_{h}$ and $K_{v}$ are the eddy coefficients for 
horizontal and vertical turbulent diffusion.

The term $F$ includes the contributions of the biological/biogeochemical 
activities, and the whole biological state vector $\Theta$ 
is explicitly considered in the last term of equation \ref{AdvDif}, without 
a spatial operator. As far as the biologically induced variations 
are regarded, the fate of each tracer in every location $x,y,z$ 
is tightly coupled to other tracers in the same location, but 
is not directly influenced by processes going on elsewhere. 

Therefore, in this approximation the global temporal variation 
of any tracer (state variable, conservative or not) can be split 
into the sum of two independent contributions: 

\begin{equation}
\frac{\partial \,\Theta _{i} }{\partial \,t} \,=\,\,\left. \frac{\partial
\,\Theta _{i} }{\partial \,t} \right\arrowvert_{phys} +\left. \frac{\partial
\,\Theta _{i} }{\partial \,t} \right\arrowvert_{biol}
\end{equation}

and it might be convenient, in writing a computer code, to devote 
independent modules to computation of each of them. Indeed, most 
of the modern water quality programs do have, at least conceptually, 
a modular structure. In this way the same code can be used for 
simulating different situations: by switching off the module 
referring to the reactor term the transport of a purely passive 
tracer is reproduced, while a 0D, close and uniformly stirred 
biological system is simulated if the module referring to the 
physical term is not included. Finally, the inclusion of both 
modules gives the evolution of tracers subjected to both physical 
and biogeochemical transformation, in a representation that, 
depending upon the parameterisation of the physical module, can 
be 1, 2 or 3 dimensional.

The whole water quality module is contained in a file
|weutro.f| and the call to EUTRO is made through a subroutine call
that is done from the main program through an appropriate 
interface. There is a clean division between the hydrodynamic 
motor, parameters used by the model and the resolution of the 
differential equations and the ecological model as evidenced 
by the overall structure of the modules. 

It is the responsibility of the main module to implement the 
time loop administration, the advective and diffusive transport 
of the state variables, both in the horizontal and vertical direction 
and the application of the boundary conditions. 

The typical use of the new EUTRO module is as follows: the main 
program first sets all parameters needed in EUTRO through the 
call to |EUTRO_INI|. These parameters are the kinetic constants 
of the reactions that are described in EUTRO and are considered 
constant for a site. They have to be set therefore only once 
at the beginning of the simulation. Once set, these parameters 
are available to the EUTRO module as global parameters.

For every box in the discretized domain (horizontal and vertical), 
and for every time step, the main program calls the subroutine 
|EUTRO0D|. Inside |EUTRO0D| the differential equations that describe 
the bio-chemical reactions are solved with a simple Euler scheme.

The values passed into |EUTRO0D| can be roughly divided into 4 
groups. The first group is made out of the aforementioned constants 
that represent the kinetic constants and other parameters that 
do not vary in time and space. The second group represents the 
state variables that are actually modified by |EUTRO0D| through 
the bio-chemical reactions. These variables are transported and 
diffused by the main routine and are just passed into |EUTRO0D| 
for the description of the processes. After the call no memory 
remains in |EUTRO0D| of these state variables. They must therefore 
be stored away by the main routine to be used in the next time 
step again. The third and fourth groups of values have to do 
with the forcing terms. They have been divided in order to account 
for the different nature of the forcing terms. The third group 
consists of the hydrodynamic forcing terms that are directly 
computed by the hydrodynamic model and parameters related to 
the box discretization. They consist of water temperature, salinity, 
current velocity, and depth and type of the box. Here the type 
identifies the position of the box (surface, water column, sediment), 
which is needed for some of the forcings to be applied. These 
variables are passed directly into EUTRO through a parameter 
list. The last group contains other forcing terms that are not 
directly related to the hydrodynamic model. These consist of 
the meteorological forcings (wind speed, air temperature, ice 
cover), light climate (surface light intensity, day length) and 
sediment fluxes. These parameters are set through a number of 
commodity functions that are called by the main routine. The 
reason why the last two parameter groups are handled differently 
from each other has also to do with the fact that the third group 
is highly variable in time and space. Variables like current 
velocity change with every time step and are normally different 
from element to element. The fourth group is very often only 
slowly variable in time (light, wind) and can very often be set 
constant in space. Therefore these values can be set at larger 
intervals, and do not have to be changed when looping over all 
the elements in the domain.

The overall flow of information during one time step is the following: 
First the hydrodynamic model resolves the momentum and continuity 
equation to update the current velocities and water levels. After 
that the physical (temperature and salinity) and bio-chemical 
scalars are advected and diffused. Once this advection step has 
been handled the new loadings and forcing terms are set-up and 
then |EUTRO0D| is called for the bio-chemical reactions. 


Note that the operator splitting technique, which decouples the 
advective and diffusive transport from the source term, allows 
for different time steps of the two processes for a more efficient 
use of the computer resources. 


Default values of the water quality parameters are already set 
in the code. Owner specific parameters for the water quality 
model should be written in the subroutine |param_user|.





\section{Light limitation}

\input steele.tex



\section{Initialization}



This section describes
the interpolation of data for the initialisation of the model.

To create a file with initial conditions the program
|laplap| can be used. The program is called as
|laplap < namefile.dat|.
This makes a laplacian interpolation of specified data contained in 
the |namefile.dat|.
This data file should have the first line empty and shold contain 
two colums containing, respectively, node number and data values for
the node.

It generates two files, |laplap.nos| and |laplap.dat|. The first 
one can be used to check if the procedure has been conducted 
well, creating a map with the plotting procedure (see Postprocessing 
section). The |.dat| file name should be given in the section
|$name| 
of the |.str| file to initialize the model.

You can create initialisation files for temperature, salinity, 
wind field and biological variables.
If you want to initialise the biological model with biological 
data you should create a single data file merging the 9 data 
files (one for each variable) using the |inputmerge.f| routine.




\newcommand{\listroutine}[1]{\paragraph{\texttt{#1}}}
\newcommand{\lt}{$<$}
\newcommand{\gt}{$>$}



\section{Post processing}

This section shows how to generate derivate variables.

The post processing routines elaborate the water quality outputs 
to generate derivate variables. They allow to generate variables 
such as averages, (both, in time and in space), sum differences, 
and water quality variables such us Vismara, TRIX and BOD5.

The routines and their usage are the following:

\listroutine{nosmaniav.f}
It generates a file containing, for each node of the spatial 
domani, average, minimum and maximum values of the specified 
variable of the whole simulation.

\listroutine{nosmaniqual.f}
It generates a water quality file from the elaboration of the 
state variable.
It computes, for each node, and at each time step a water quality 
index that can be chosen between two suggested indexes: Vismara 
QualityV and TRIX a well known quality index applied to the 
water quality definition at coastal seas and estuaries.

These indices can be computed using the definitions
in Table \ref{ClassWQI} and
the following equations:

\begin{center}
\begin{verb}
QualityV = class(O2sat) + class(BOD5) + class(NH3)
\\
auxt1 = (phyto/30.)*1000\\
auxt2 = (nh3+nox)*1000\\
auxt3 = (opo4+op)*1000\\
TRIX = log10(auxt1*o2satp*auxt2*auxt3+1.5)/1.2
\end{verb}
\end{center}


\begin{table}\centering
\begin{tabular}{lccccc}
\hline

Class(Var) & 1 & 2 & 3 & 4 & 5\\
O2sat & 90-110 & 70-90 or 110-120 & 50-70 or 120-130 & 30-50 & \lt30 or \gt130\\
BOD5 & \lt3 & 3-6 & 6-9 & 9-15 & \gt15\\
NH3 & \lt0.4 & 0.4-1 & 1-2 & 2-5 & \lt5\\

\hline
\end{tabular}
\caption{Classification of Water Quality Indices}
\label{ClassWQI}
\end{table}




\listroutine{nosmanintot.f}
generates a file of total inorganic Nitrogen as sum of NH3 and 
NOx

\listroutine{nosdif.f}
computer for the chosen state variable, the difference between 
the values at two times step. 


\listroutine{nosdiff.f}
computer the difference between the variable outputs of two simulations 



\listroutine{nosmanibod5.f}
computes the BOD5 values from the CBOD outputs as:

\begin{center}
\begin{verb}
bod5 = cbod*(1. - exp( -5. * par1 ))
\\
        + (64./14.) * nh3 * (1. - exp( -5. * par2 ))
\end{verb}
\end{center}

To run one of the postprocessing routine write the name of the 
routine and enter.




\section{The Sediment Module}

The sediment buffer action on the biogeochemical cycles could 
be very important, especially in the shallow water basins and 
during the storm surge events.

The routine |wsedim| (introduced in April 2004) aims to address 
the resuspention/sinking dynamics of nitrogen and phosphorous.
This routine can be switched on and off as needed by the user, 
setting the |bsedim| parameter |true| or |false| in the |bio3d| 
routine. It is called after the the eutrophication subroutine.

It allows to follow the dynamics of two additional variables, 
OPsed and ONsed that simulate the evolution of Nitrogen and Phosphorous 
detritus in sediment that are not subjected to advection-diffusion 
processes.

These two variables interact with the Nitrogen and Phosphorous 
cycle as decribed by the equations in Table \STone. When 
the |wsedim| subroutine is switched on, OP, ON, NH3 and OPO4 are 
updated at each time step in agreement with those equations.

The resuspension is a linear function of the water velocity calculated 
by the hydrodynamic model at each box, as written in Table \STtwoa. 
The amount of the sinking nutrients depends on specific prossess 
parameters, as given in Table \STtwoa, and on the depth of the underlying 
column.


%\documentclass{report}
%\usepackage{a4}
%\usepackage{shortvrb}
%\begin{document}
%
%\newcommand{\Degree}{${}^{o}$}
%\newcommand{\power}[1]{${}^{#1}$}
%
%\newcommand{\VSPGB}{\vspace{0.2cm}}
%\newcommand{\HSP}{\hspace*{1.5cm}}
%\newcommand{\HHSP}{\hspace*{0.5cm}}
%\newcommand{\GBox}[2]{\parbox{#1 cm}{\VSPGB #2 \VSPGB}}
%\newcommand{\GDBox}[1]{\GBox{5}{#1}}
%\newcommand{\GEBox}[1]{\GBox{7}{#1}}
%\newcommand{\GFBox}[1]{\GBox{7}{#1}}
\newcommand{\GHBox}[1]{\GBox{5.2}{#1}}
%
%\newcommand{\opn}[1]{\, #1 \,}
%\newcommand{\opn}{ {} }
%\newcommand{\opn}[1]{#1}



\begin{table}\centering
\begin{tabular}{lll}
\hline


& & \\
$\frac{\partial S}{\partial t} =Q(S)_{sed}$
& &
General Reactor Equation
\\
& & \\

$Q(NH3)_{sed}= NH3_{res}$
& 3 &
Ammonia NH3 [mg N/L]
\\

$Q(ON)_{sed} = (ON_{res} - ON_{sink})$
& 5 &
Organic Nitrogen ON [mg N/L]
\\

$Q(OPO4)_{sed}= OPO4_{res}$
& 6 &
\GBox{5}{
Inorganic Phosphorous \\
\HHSP OPO4 [mg P/L]
}
\\

$Q(OP)_{sed}= (OP_{res}-OP_{sink})$
& 7 &
Organic Phosphorous OP [mg P/L]
\\

\GBox{5}{
$Q(ON_{sed})= ON_{sink} - ON_{res} $
\HSP $\opn - NH3_{res}$
}
& 1sed &
\GBox{5}{
Sediment Organic Nitrogen \\
\HHSP ON${}_{sed}$ [mg N/L]
}
\\

\GBox{5}{
$Q(OP_{sed})= OP_{sink}- OP_{res}$
\HSP $\opn - OPO4_{res}$
}
& 2sed &
\GBox{5}{
Sediment Organic Phosphorous \\
\HHSP OP${}_{sed}$ [mg N/L]
}
\\


\hline
\end{tabular}
\caption{Sediment Mass balances}
\label{SMassBalance}
\end{table}




\begin{table}\centering
\begin{tabular}{lll}
\hline

\GBox{5}{
$NH3_{res}= Vol_{sed}*KNC_{sed}$
\HSP $\opn *KNT^{(T-T_0)}* ON_{sed}$
}
& 1 &
\GHBox{
Mineralization of organic nitrogen in sediment
}
\\

$ON_{res} = Vol_{sed}*KN_{res}*F_{vel}*ON_{sed}$
& 2 &
\GHBox{
Resuspention of organic nitrogen in sediment
}
\\

\GBox{5}{
$ON_{sink}= Vol*\frac{(1-exp(-dt/\tau_N))}{dt}$
\HSP $\opn *ON*FPON$
}
& 3 &
\GHBox{
Sink of organic nitrogen from the water column
}
\\

\GBox{5}{
$OPO4_{res}= Vol_{sed}*KPC_{sed}$
\HSP $\opn *KPT^{(T-T_0)}*OP_{sed}$
}
& 4 &
\GHBox{
Mineralization of organic phosphorous in sediment
}
\\

$OP_{res}= Vol_{sed} * KP_{res} * F_{vel} * OP_{sed}$
& 5 &
\GHBox{
Resuspention of organic phosphorous in sediment
}
\\

\GBox{5}{
$OP_{sink}= Vol*\frac{(1-exp(-dt/\tau_P))}{dt}$
\HSP $\opn *OP*FPOP$
}
& 6 &
\GHBox{
Sink of organic phosphorous from the water column
}
\\

$\tau_N = H/w_s$
& 7 &
\GHBox{
Time scale for sinking processes of organic N
}
\\

$\tau_P = H/w_s$
& 8 &
\GHBox{
Time scale for sinking processes of organic P
}
\\

\hline
\end{tabular}
\caption{Sediment functional expressions}
\label{SFuncDesc}
\end{table}







\begin{table}\centering
\begin{tabular}{lll}
\hline

$KNC_{sed} = 0.075$
&
Mineralization of sediment ON rate constant
\\

$KNT = 1.08$
&
Mineralization of sediment ON rate temperature constant
\\

$KPC_{sed} = 0.22$
&
Mineralization of sediment OP rate constant
\\

$KPT = 1.08$
&
Mineralization of sediment OP rate temperature constant
\\

$KN_{res} = 0.1$
&
Fraction of sediment depth resuspended/day
\\

$KP_{res} = 0.1$
&
Fraction of sediment depth resuspended/day
\\

$FPON = 0.5$
&
Fraction of particulate organic N
\\

$FPOP = 0.5$
&
Fraction of particulate organic P
\\

$F_{vel} = 1$
&
Velocity coefficient
\\

$w_s = 10$ m/day
&
Sinking velocity
\\

$T_0 = 20$ \Degree C
&
Optimal temperature value
\\

\hline
\end{tabular}
\caption{Sediment parameters}
\label{SParas}
\end{table}




\begin{table}\centering
\begin{tabular}{lll}
\hline

$Vol$
& [m\power{3}] &
volume
\\

$Vol_{sed}$
& [m\power{3}] &
sediment volume
\\

$H$
& [m] &
total depth of water column
\\

$dt$
& [sec] &
time step
\\

\hline
\end{tabular}
\caption{Sediment variables}
\label{SVars}
\end{table}







%\end{document}



%\end{document}


\subsection{Parameters for the Water Quality Module}

\input{S_biopar_h.tex}





\bibliography{abbrev,lag}
\addcontentsline{toc}{chapter}{Bibliography}



\end{document}
